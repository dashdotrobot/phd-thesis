\documentclass{nuthesis}

%%% Packages
\usepackage{amsmath}
\usepackage{graphicx}
\usepackage{subfiles}
\usepackage[symbols,nogroupskip,sort=none,toc=false,numberedsection=autolabel]{glossaries-extra}
\usepackage{xcolor}
\usepackage{import}
\usepackage{todonotes}
\usepackage{tabularx}
\usepackage{multirow}
\usepackage{booktabs}
\usepackage{siunitx}

\sisetup{
	separate-uncertainty = true
}


%%% Setup symbols and math shortcuts
%!TEX root = thesis.tex

%%% Symbols


% Coordinate system
\glsxtrnewsymbol[description={basis vectors of a cylindrical coordinate system at the rim center}]{ebasis}
	{\ensuremath{\mathbf{e}_1, \mathbf{e}_2, \mathbf{e}_3}}

\glsxtrnewsymbol[description={arc length coordinate}]{arc}{\ensuremath{s}}
\glsxtrnewsymbol[description={angular coordinate}]{ang}{\ensuremath{\theta}}

\glsxtrnewsymbol[description={Cartesian coordinates of a point in the rim cross-section relative to its centroid}]{xy}{\ensuremath{x, y}}


% Stress states
\glsxtrnewsymbol[description={quantity in the prestressed configuration}]{Sp}{\ensuremath{()_p}}
\glsxtrnewsymbol[description={change in quantity moving from the prestressed configuration to the deformed configuration}]{delta}{\ensuremath{\delta()}}


% Displacement and strain
\glsxtrnewsymbol[description={vector displacement field}]{uvec}{\ensuremath{\mathbf{u}}}
\glsxtrnewsymbol[description={vector displacement of the rim shear center}]{usvec}{\ensuremath{\mathbf{u}_s}}
\glsxtrnewsymbol[description={vector displacement of the spoke nipple}]{unvec}{\ensuremath{\gls{uvec}_n}}
\glsxtrnewsymbol[description={shear center displacement vector components}]{uvw}{\ensuremath{u,v,w}}
\glsxtrnewsymbol[description={rotation of rim cross-section about its normal vector}]{p}{\ensuremath{\phi}}
\glsxtrnewsymbol[description={augmented rim displacement vector}]{d}{\ensuremath{\mathbf{d}}}

\glsxtrnewsymbol[description={rotation vector of the line of shear centers}]{rot}{\ensuremath{\mathbf{\omega}}}
\glsxtrnewsymbol[description={curvature vector of the line of shear centers}]{curv}{\ensuremath{\mathbf{\kappa}}}

\glsxtrnewsymbol[description={longitudinal strain in the rim}]{elon}{\ensuremath{\varepsilon_{33}}}
\glsxtrnewsymbol[description={longitudinal strain of the centroidal line}]{ec}{\ensuremath{\varepsilon_c}}
\glsxtrnewsymbol[description={torsion-induced shear strain}]{shear}{\ensuremath{\gamma}}


% Geometry
\glsxtrnewsymbol[description={distance between hub flanges}]{wh}{\ensuremath{w_h}}
\glsxtrnewsymbol[description={hub flange diameter}]{dh}{\ensuremath{d_h}}
\glsxtrnewsymbol[description={out-of-plane spoke angle}]{alphas}{\ensuremath{\alpha}}
\glsxtrnewsymbol[description={in-plane spoke angle}]{betas}{\ensuremath{\beta}}

\glsxtrnewsymbol[description={number of spokes}]{ns}{\ensuremath{n_s}}
\glsxtrnewsymbol[description={length of a spoke}]{ls}{\ensuremath{l_s}}
\glsxtrnewsymbol[description={cross-sectional area of a spoke}]{As}{\ensuremath{A_s}}
\glsxtrnewsymbol[description={spoke unit vector pointing from hub to rim}]{nvec}{\ensuremath{\mathbf{n}}}
\glsxtrnewsymbol[description={direction cosines of spoke vector \n}]{ccc}{\ensuremath{c_1, c_2, c_3}}
\glsxtrnewsymbol[description={vector offset from rim shear center to spoke nipple}]{bs}{\ensuremath{\mathbf{b}_s}}
\glsxtrnewsymbol[description={lateral and radial components of \bs}]{bb}{\ensuremath{b_1, b_2}}

\glsxtrnewsymbol[description={rim radius measured at the shear center}]{R}{\ensuremath{R}}
\glsxtrnewsymbol[description={height of the shear center above the centroid}]{yo}{\ensuremath{y_0}}
\glsxtrnewsymbol[description={radii of gyration for lateral and radial bending, respectively}]{rxry}{\ensuremath{r_x, r_y}}
\glsxtrnewsymbol[description={outer radius of circular rim cross-section (see Optimization)}]{r}{\ensuremath{r}}
\glsxtrnewsymbol[description={wall thickness of circular rim cross-section}]{twall}{\ensuremath{t_w}}


% Rim section properties
\glsxtrnewsymbol[description={rim cross-sectional area}]{Arim}{\ensuremath{A}}
\glsxtrnewsymbol[description={rim second moment of area for radial bending}]{Irad}{\ensuremath{I_1}}
\glsxtrnewsymbol[description={rim second moment of area for lateral bending}]{Ilat}{\ensuremath{I_2}}
\glsxtrnewsymbol[description={rim torsion constant}]{Jtor}{\ensuremath{J}}

\glsxtrnewsymbol[description={rim warping function}]{warp}{\ensuremath{\alpha_s}}
\glsxtrnewsymbol[description={rim warping constant}]{Iwarp}{\ensuremath{I_w}}
\glsxtrnewsymbol[description={ratio of torsional stiffness to lateral bending stiffness}]{mu}{\ensuremath{\mu}}
\glsxtrnewsymbol[description={ratio of torsional stiffness to lateral bending stiffness, including warping effects}]{mut}{\ensuremath{\tilde{\mu}}}


% Material properties
\glsxtrnewsymbol[description={Young's modulus and shear modulus of the rim, respectively}]{EGrim}{\ensuremath{E, G}}
\glsxtrnewsymbol[description={Young's modulus of a spoke}]{Espk}{\ensuremath{E_s}}
\glsxtrnewsymbol[description={mass density of the rim}]{rimdens}{\ensuremath{\rho_r}}
\glsxtrnewsymbol[description={mass density of the spokes}]{spkdens}{\ensuremath{\rho_s}}

\glsxtrnewsymbol[description={mass of the rim}]{mrim}{\ensuremath{m_{rim}}}
\glsxtrnewsymbol[description={total mass of the spokes}]{mspk}{\ensuremath{m_{spk}}}
\glsxtrnewsymbol[description={total mass of the rim and spokes}]{M}{\ensuremath{M}}


% Forces
\glsxtrnewsymbol[description={tension in a spoke}]{T}{\ensuremath{T}}
\glsxtrnewsymbol[description={average radial spoke tension per unit rim circumference}]{Tb}{\ensuremath{\bar{T}}}
\glsxtrnewsymbol[description={non-dimensionalized spoke tension}]{t}{\ensuremath{\tau}}
\glsxtrnewsymbol[description={critical tension for rim buckling}]{Tc}{\ensuremath{\gls{T}_c, \gls{Tb}_c, \gls{t}_c}}

\glsxtrnewsymbol[description={force exerted on a spoke by the rim}]{fs}{\ensuremath{\mathbf{f}}}
\glsxtrnewsymbol[description={modal force vector (vector of projections of external forces onto modes)}]{Fext}{\ensuremath{\mathbf{F}_{ext}}}

\glsxtrnewsymbol[description={axial force in the rim}]{Fax}{\ensuremath{F_3}}
\glsxtrnewsymbol[description={radial shear force in the rim}]{Fsh}{\ensuremath{F_2}}

\glsxtrnewsymbol[description={external force}]{P}{\ensuremath{P}}
\glsxtrnewsymbol[description={critical external force to cause a single spoke to buckle}]{Psb}{\ensuremath{P_{sb}}}
\glsxtrnewsymbol[description={critical external force to cause the rim to buckle}]{Pc}{\ensuremath{P_c}}

\glsxtrnewsymbol[description={strain energy}]{U}{\ensuremath{U}}
\glsxtrnewsymbol[description={total potential energy}]{Pi}{\ensuremath{\Pi}}


% Mode-matrix terms
\glsxtrnewsymbol[description={displacement mode coefficients, where $()$ may be $u$, $v$, $w$, or $\p$}]{mc}{\ensuremath{()_0, ()_n^c, ()_n^s}}
\glsxtrnewsymbol[description={mode number}]{n}{\ensuremath{n}}
\glsxtrnewsymbol[description={highest mode included in the mode stiffness approximation}]{N}{\ensuremath{N}}
\glsxtrnewsymbol[description={vector of mode displacement coefficients}]{dm}{\ensuremath{\gls{d}_m}}
\glsxtrnewsymbol[description={matrix relating mode displacement vector to augmented displacement vector}]{B}{\ensuremath{\mathbf{B}}}


% Stiffness
\glsxtrnewsymbol[description={axial stiffness of a spoke}]{Ks}{\ensuremath{K_s}}
\glsxtrnewsymbol[description={spoke stiffness matrix relating the force on a spoke to the spoke nipple displacement}]{kforce}{\ensuremath{\mathbf{k}_f}}
\glsxtrnewsymbol[description={spoke augmented stiffness matrix, including $\p$ terms}]{k}{\ensuremath{\mathbf{k}}}
\glsxtrnewsymbol[description={homogenized spoke stiffness per unit rim circumference}]{kbar}{\ensuremath{\mathbf{\bar{k}}}}
\glsxtrnewsymbol[description={components of \gls{kbar} in cylindrical coordinates}]{kij}{\ensuremath{\bar{k}_{ij}}}

\glsxtrnewsymbol[description={rim modal stiffness matrix}]{KmRim}{\ensuremath{\mathbf{K}_{rim}}}
\glsxtrnewsymbol[description={rim stiffness matrix for the $n$th mode}]{KmnRim}{\ensuremath{\mathbf{K}_n^{rim}}}
\glsxtrnewsymbol[description={spoke system modal stiffness matrix}]{KmSpk}{\ensuremath{\mathbf{K}_{spk}}}
\glsxtrnewsymbol[description={spoke system modal stiffness matrix, smeared-spoke approximation}]{KmSpkSm}{\ensuremath{\bar{\mathbf{K}}_{spk}}}

% Point stiffnesses
\glsxtrnewsymbol[description={radial point-load stiffness of the wheel}]{Krad}{\ensuremath{K_{rad}}}
\glsxtrnewsymbol[description={lateral point-load stiffness of the wheel}]{Klat}{\ensuremath{K_{lat}}}
\glsxtrnewsymbol[description={tangential point-load stiffness of the wheel}]{Ktan}{\ensuremath{K_{tan}}}
\glsxtrnewsymbol[description={scalar lateral mode stiffness of the wheel}]{Knlat}{\ensuremath{K_n}}
\glsxtrnewsymbol[description={scalar lateral mode stiffness of the rim alone}]{KnRim}{\ensuremath{K_n^{rim}}}
\glsxtrnewsymbol[description={stiffness evaluated at zero spoke tension}]{Kzero}{\ensuremath{()^{0}}}

\glsxtrnewsymbol[description={denotes the part of the spoke proportional to $E$, $G$, or \gls{Espk}}]{kmatl}{\ensuremath{()^{matl}}}
\glsxtrnewsymbol[description={denotes the part of the spoke stiffness proportional to $\Tb$}]{kgeom}{\ensuremath{()^{geom}}}


% Non-dimensional groups
\glsxtrnewsymbol[description={non-dimensionalized ratio of spoke stiffness to rim stiffness}]{lr}{\ensuremath{\lambda}}


% Acoustic testing terms
\glsxtrnewsymbol[description={natural frequency of a radial bending mode}]{fnrad}{\ensuremath{f_n^{rad}}}
\glsxtrnewsymbol[description={natural frequency of a lateral bending-twisting mode}]{fnlat}{\ensuremath{f_n^{lat}}}
\glsxtrnewsymbol[description={time}]{time}{\ensuremath{t}}


% Design and optimization terms
\glsxtrnewsymbol[description={fraction of wheel mass in the rim}]{fr}{\ensuremath{f_{rim{}}}}
\glsxtrnewsymbol[description={vector of wheel design parameters}]{X}{\ensuremath{\chi}}
\glsxtrnewsymbol[description={vector of extensive wheel parameters, $(\gls{R}, \gls{M})$}]{Xext}{\ensuremath{\chi_{extent}}}
\glsxtrnewsymbol[description={vector of geometric wheel parameters, $(r/\gls{R}, \gls{wh}/\gls{R}, \gls{dh}/\gls{R})$}]{Xgeom}
	{\ensuremath{\chi_{geom}}}
\glsxtrnewsymbol[description={vector of material wheel parameters, $(E,G,\gls{Espk},\gls{rimdens},\gls{spkdens})$}]{Xmatl}
	{\ensuremath{\chi_{matl}}}

%%% Symbol shortcuts
\newcommand{\R}{\gls{R}}
\newcommand{\eo}{\mathbf{e}_1}
\newcommand{\et}{\mathbf{e}_2}
\newcommand{\eh}{\mathbf{e}_3}

\newcommand{\x}{\ensuremath{x}}
\newcommand{\y}{\ensuremath{y}}

\newcommand{\n}{\gls{nvec}}
\newcommand{\npo}{\gls{nvec}_{\perp 1}}
\newcommand{\npt}{\gls{nvec}_{\perp 2}}

\newcommand{\bs}{\gls{bs}}
\newcommand{\D}{\mathcal{D}}
\newcommand{\ds}[2]{\frac{d^#2#1}{ds^#2}}
\newcommand{\dt}[2]{\frac{d^#2#1}{d\theta^#2}}

\newcommand{\p}{\gls{p}}

\newcommand{\T}{\gls{T}}
\newcommand{\Tb}{\gls{Tb}}
\newcommand{\ts}{\gls{t}}

\newcommand{\kuu}{\bar{k}_{uu}}
\newcommand{\kvv}{\bar{k}_{vv}}
\newcommand{\kww}{\bar{k}_{ww}}
\newcommand{\kpp}{\bar{k}_{\phi\phi}}
\newcommand{\kuv}{\bar{k}_{uv}}
\newcommand{\kuw}{\bar{k}_{uw}}
\newcommand{\kup}{\bar{k}_{u\phi}}
\newcommand{\kvw}{\bar{k}_{vw}}
\newcommand{\kvp}{\bar{k}_{v\phi}}
\newcommand{\kwp}{\bar{k}_{w\phi}}

% Define shortcut commands to \gls parameters to fix alignment of super and subscripts
\newcommand{\KmRim}{\mathbf{K}_{rim}}
\newcommand{\KmSpk}{\mathbf{K}_{spk}}

\newcommand{\matl}[1]{#1^{matl}}
\newcommand{\geom}[1]{#1^{geom}}

\newcommand{\lr}{\lambda}

\newcommand{\E}{\ensuremath{E}}
\newcommand{\G}{\ensuremath{G}}

\newcommand{\EA}{\ensuremath{\E\gls{Arim}}}
\newcommand{\EIr}{\ensuremath{\E\gls{Irad}}}
\newcommand{\EIl}{\ensuremath{\E\gls{Ilat}}}
\newcommand{\EIw}{\ensuremath{\E\gls{Iwarp}}}
\newcommand{\GJ}{\ensuremath{\G\gls{Jtor}}}

\newcommand{\Kbend}{\ensuremath{K_b}}
\newcommand{\Ktors}{\ensuremath{K_t}}

\newcommand{\m}{\gls{mu}}
\newcommand{\mt}{\gls{mut}}

\newcommand{\rx}{\ensuremath{r_x}}
\newcommand{\ry}{\ensuremath{r_y}}
\newcommand{\yo}{\ensuremath{\gls{yo}}}
\newcommand{\ro}{\ensuremath{r_0}}



%%% Custom commands
\providecommand{\rootdir}{.}

\newcommand{\inprogress}{
\begin{center}
{
\color{red}
\noindent\rule{\textwidth}{1pt}
\Large (Section in progress)
\noindent\rule{\textwidth}{1pt}
}
\end{center}
}

\newcommand{\executeiffilenewer}[3]{
% Update pdf_tex if SVG has been updated (or if pdf_tex does not exist)
\ifnum \pdfstrcmp{\pdffilemoddate{#1}}{\pdffilemoddate{#2}} > 0
	{\immediate\write18{#3}}
\fi
}

\newcommand{\includesvg}[2]{%
\executeiffilenewer{#1#2.svg}{#1#2.pdf}%
{inkscape -z -D --file=#1#2.svg %
--export-pdf=#1#2.pdf --export-latex}%
\import{#1}{#2.pdf_tex}
}


% Extra-wide right margin for notes (REMOVE FOR PUBLICATION)
\setlength{\hoffset}{-0.75in}


\title{Reinventing the Wheel:\\Stress Analysis, Stability, and Optimization of the Bicycle Wheel}
\author{Matthew Ford}
\field{Mechanical Engineering}
\campus{EVANSTON, ILLINOIS}

\begin{document}

\maketitle

\copyrightpage

\abstract{
The tension-spoke bicycle wheel owes its stiffness and strength to a cooperative relationship between the rim and the pretensioned spokes. The rim holds the spokes in tension to prevent them from buckling under external loads, while the spokes channel external loads to the hub and prevent the rim from becoming severely distorted. The prestressing process allows the use of very slender spokes, but also makes the rim susceptible to elastic buckling. I aim to uncover the principles governing the deformation and stability of the wheel under internal and external forces.

I establish a theoretical framework for analysis of the wheel as a monosymmetric elastic beam (the rim) anchored by uniaxial elastic truss elements (the spokes). From a general statement of the total energy of the system, I derive a set of coupled, linear, ordinary differential equations describing the deformation of the wheel and illustrate instances in which those equations can be solved analytically. To solve the general equations, approximate the displacement field with a finite set of periodic functions to transform the differential equations to a linear matrix equation. This matrix equation leads to an intuitive model of the bicycle wheel as an infinite array of springs connected in series, where each spring contains a contribution from the rim and the spokes. The series-springs model reveals how the bending and twisting stiffnesses of the rim act also like springs in series, and is therefore dominated by the smaller of the two.\todo{mention stiffness with tension}

Next I derive the condition for elastic stability of the prestressed wheel absent external forces. The buckling criterion reveals the dominant role of the spoke stiffness and arrangement. Under external loads, two competing failure modes govern the elastic stability of the wheel: spoke buckling and rim buckling. High spoke tension inhibits spoke buckling but promotes rim buckling. This trade-off leads to an optimum spoke tension of roughly \SI{50}{\percent} of the critical buckling tension.

Finally I discuss the existence of optimal wheel configurations and properties. By reducing the design space to a single parameter---the mass of the rim divided by the total mass---I find optimal wheels to maximize the lateral stiffness, radial strength, and buckling tension. The existence of an optimal wheel for a given mass, rim radius, and hub width permits investigation of general scaling laws governing stiffness and strength.
}

\tableofcontents


\chapter{Introduction}
\label{chap:introduction}
\subfile{chapters/introduction}


\chapter{Linear stress analysis}
\label{chap:stress_analysis}
\subfile{chapters/stress_analysis}


\chapter{Acoustic characterization of bicycle rims}
\label{chap:acoustic_testing}
\subfile{chapters/acoustic_testing}


\chapter{Flexural-torsional buckling under uniform tension}
\label{chap:tension_buckling}
\subfile{chapters/buckling_tension}


\chapter{Buckling under external loads}
\label{chap:buckling_ext_loads}
\subfile{chapters/buckling_ext_loads}


\chapter{Optimization of bicycle wheels}
\label{chap:optimization}
\subfile{chapters/optimization}


\chapter*{Acknowledgments}

My deepest thanks go to my thesis committee: Oluwaseyi Balogun, James M. (Jim) Papadopoulos, and John Rudnicki. I am especially grateful for the support of my adviser, Oluwaseyi Balogun, who gave me a home in his lab and took a great professional risk by encouraging me to pursue an unconventional thesis topic. This project would not have progressed without the critical eye of Jim Papadopoulos, with whom I exchanged many manuscripts and ideas and whose pioneering work on the dynamics and stability of the bicycle and studies of the mechanics of the wheel ``paved the way'' for this work. I am grateful for the support and advice I received from Professor Jan D. Achenbach, whose generosity with his research funds provided by Sigma Xi, the scientific research honor society, and with his time and wisdom enabled this project to succeed.

I, like many mechanics researchers at Northwestern, am indebted to Dr. Joel Fenner for his consultations on experimental measurement techniques. Scarcely a test fixture or amplifier circuit is constructed in the department without his expert guidance. His boundless curiosity for, and encyclopedic knowledge of topics ranging from vacuum tube technology to pianoforte construction is an inspiration to me.

I am lucky to have found The Recyclery, an educational community bike shop and diverse community of generous and passionate humans. In addition to giving me access to bicycle components and tools, The Recyclery connected me with the broader world of community bike projects.

Many thanks to Professors Michael Beltran and J. Alex Birdwell, and Ellen Owens for the guidance they gave to the Northwestern Bicycle Wheel Tester team through the ME 398 capstone course.

This material is based upon work supported by the National Science Foundation Graduate Research Fellowship Program under Grant No. DGE-1324585. Conference travel was funded by the Ryan Fellowship, made possible by the generous support of Patrick G. and Shirley W. Ryan. Funding for the ME 398 wheel tester was provided by the Mechanical Engineering department at Northwestern.


\bibliographystyle{plain}
\bibliography{references}


\appendix
\addtocontents{toc}{\protect\setcounter{tocdepth}{0}}

\printunsrtglossary[type=symbols, style=long, title={List of symbols}]

\chapter{Acoustic testing additional procedures and results}
\label{app:acoustic_testing}
\subfile{chapters/appendix/acoustic_testing}

\chapter{Stiffness matrix for asymmetric $n$-cross wheel}
\label{app:kbar_asymm}
\subfile{chapters/appendix/spoke_stiffness_ncross_asymm}



\end{document}
