\providecommand{\rootdir}{..}
\documentclass[../thesis.tex]{subfiles}

\begin{document}

In the preceding analysis, we have assumed prior knowledge of $EI_1$, $EI_2$, $GJ$ and $EI_w$. Rim cross-sections typically have complicated shapes with multiple open and closed regions and the exact shape and wall thickness cannot be easily determined without a destructive test. Furthermore, the spoke holes reduce the effective bending stiffness over relevant length scales in a complex manner. Therefore, it is desirable to obtain the rim section properties experimentally and with minimal assumptions or computation.

Pippard and Francis \cite{Pippard1931b,Pippard1932d} undertook the first quantitative experimental investigations of the stiffness of spoked wheels and compared their results with an analytical solution. For the special case of radial loads, they determined the in-plane bending stiffness of bare rims by diametral extension. The rims that they tested were cut from a steel plate and all had rectangular cross-sections of varying aspect ratio and did not resemble the complicated cross-sections of modern rims. Due to the difficulty of determining the out-of-plane bending stiffness and torsional stiffness of a circular beam, their investigation of lateral deformations was limited to theory alone. Burgoyne and Dilmaghanian \cite{Burgoyne1993b} performed experiments on bicycle wheels and compared their results with Pippard's theory. They calculated the radial bending stiffness of the rim from geometric analysis of its cross-section, but their study was limited to radial loads. Gavin \cite{Gavin1996b} noted that the out-of-plane bending stiffness and torsional stiffness are coupled in curved beams and require at least two independent measurements to determine. He performed out-of-plane deflection tests while clamping the rim at two points with various arc lengths. This method requires rigid clamps and neglects warping of the cross-section.

Experimental Modal Analysis (EMA) \cite{Ewins1984} is a technique for inspecting structures to predict the dynamic response, assess the quality of a manufactured product, or monitor the health of an existing structure \cite{Salawu1997}. In one variant of EMA, the structure is impulsively excited and then monitored using one or more accelerometers or contact transducers. With enough transducers, both the natural frequencies and mode shapes may be estimated. If spatial information is not required and the modes of interest have sufficiently high acoustic coupling in air, a microphone may be used to obtain a spectrum, allowing for non-contact measurement.

The ubiquity, connectivity, and computational power of smartphones have inspired applications in non-destructive evaluation (NDE) and structural health monitoring (SHM). The built-in accelerometer has been used to identify natural modes of buildings and bridges \cite{Feng2015}, measure inclination angles \cite{Morgenthal2012}, and detect and quantify seismic events \cite{Kong2016}. Smartphone accelerometers generally have a maximum frequency of 50-100 Hz, and thus are limited to measuring seismic activity or natural modes of large structures.

The microphone picks up where the accelerometer leaves off: the one used in this study has a relatively flat frequency response above 100 Hz. Although the microphone has received limited attention for NDE applications, smartphone microphones have been used for close-range sonar measurements \cite{Morgenthal2012}, detecting roller bearing failures \cite{Grebenik2016}, and measuring bicycle spoke tension \cite{Pepelko2016}. Other potential applications in the audible range include concrete bridge deck inspections, which often rely on the operator's trained ear to detect anomalies, and rapid inspection of automotive assemblies during manufacturing. In this paper we present a novel application using a smartphone microphone for quantitative, model-based NDE of bicycle rims.

We have developed a method for measuring the stiffness of bicycle rims for both in-plane and out-of-plane loads using quantitative model-based EMA. Our method is fast, non-destructive, and can be performed with only simple household tools including a weight scale, a piece of string, and a smartphone. For the purposes of validating the technique we also compare the calculated stiffness from the acoustic test with quasistatic load-displacement tests in both the radial and lateral directions.


\section{Resonant frequencies of a bicycle rim}

A bicycle rim without spokes will resonate at its natural frequencies when struck. These resonant modes are within the audible range and can be easily recorded with a standard smartphone microphone. The modes are classified into radial bending modes (rim moves entirely within its plane) and lateral-torsional modes (rim moves out of its plane). Although both types will be present in an experimental spectrum, they can be preferentially excited by striking the rim at different angles, much like how a percussionist can control the timbre of a gong or drum.

The natural frequencies of the radial bending modes depend on the rim properties as follows \cite{Timoshenko1955}:
  \begin{equation}\label{eq:fn_rad}
  f_n^{rad} = \frac{n(n^2-1)}{\sqrt{n^2+1}} \sqrt{\frac{EI_1}{2\pi R^3M}}
  \end{equation}

where $f_n^{rad}$ is the $n$th harmonic frequency and $M$ is the total mass of the rim. The first mode $n=1$ corresponds to a rigid-body motion with zero frequency. The fundamental vibration mode is $n=2$. Having measured $M$ and $R$ and identified several modes from the frequency spectrum, the in-plane bending stiffness $EI_1$ can be determined by solving Eqn. \ref{eq:fn_rad} and averaging the result from several different modes.

If warping is neglected, the frequencies of the lateral-torsional modes depend on the rim properties as follows \cite{Timoshenko1955}:
  \begin{equation}\label{eq:fn_lat}
  f_n^{lat} = \frac{n(n^2-1)}{\sqrt{\mu n^2+1}} \sqrt{\frac{GJ}{2\pi R^3M}}
  \end{equation}

where $\mu=GJ/EI_{2}$. Unlike Eqn. \ref{eq:fn_rad}, Eqn. \ref{eq:fn_lat} depends on two independent stiffness parameters $\mu$ and $GJ$ which must be determined simultaneously.


\section{Experimental procedure}

\subsection{Acoustic Test}

  \begin{table}
  \caption{Properties of the rims in this study.\label{tb:rims}}
  \begin{tabular}{@{}llcc}
  \hline\noalign{\smallskip}
  \bf{Rim} & Type$^{\rm a}$ & $R$ [mm] & $M$ [g]\\
  \noalign{\smallskip}\hline\noalign{\smallskip}
  Alex ALX295          & DDW & 305 & 480\\
  DT Swiss R460        & DDW & 304 & 459\\
  Sun Ringle CR18 20'' & DW  & 217 & 380\\
  Sun Ringle CR18 700c & DW  & 304 & 540\\
  Alex Y2000 26''      & SW  & 271 & 460\\
  Alex Y2000 700c      & SW  & 302 & 551\\
  Alex X404 27''       & SW  & 307 & 594\\
  \noalign{\smallskip}\hline
  \end{tabular}\\
  $^{\rm a}$Cross-section type: DDW=deep double-wall, DW=double-wall, SW=single-wall.
  \end{table}

\begin{figure}
\centering
\includesvg{\rootdir/figs/acoustic_testing/}{schem_and_fft}
\caption{(a)-(b) Experimental setup for radial and lateral strike test. (c)-(d) Time-domain signals for radial and lateral strike test. (e) Fourier spectrum for radial strike (top), lateral strike (middle), and background noise (bottom).}
\label{fig:schem}
\end{figure}

The impulse responses of seven aluminum rims of unknown properties were obtained by the following procedure: the rim was suspended by a string from the valve stem hole and struck with a screwdriver handle wrapped in rubber. The rim was struck first on the inside circumference and then on the sidewall at a point between two spoke holes approximately 10$^{\circ}$ from the bottom of the wheel to excite as many modes as possible. Audio was recorded with the ``Recorder+'' app on an iPhone SE using the built-in microphone at a sampling rate of 44.1 kHz. The frequency spectrum was estimated by averaging eight spectra calculated using the Fast Fourier Transform with a bandwidth of 1.35 Hz. A noise spectrum was also obtained by recording several seconds of silence in the same room. The frequency response of the built-in microphone was measured in an anechoic chamber (see Online Resource 1).

The peaks with a signal-to-noise ratio greater than 10 were identified and classified as radial or lateral modes depending on their relative magnitude in the two spectra. The frequency of each peak was determined by fitting a Lorentzian function in the neighborhood of the maximum value. The two peaks at 27 Hz and 60 Hz were present in the noise spectrum and therefore rejected.

\subsection{Diametral compression}
The rims were then loaded in diametral compression under displacement control in an Instron MTS (Figure \ref{fig:mech_tests} (b)). The valve hole was placed at 45$^{\circ}$ from the load point where the bending moment is minimized to reduce its effect on the measurement. Castigliano's method gives the deflection of a ring subjected to radial point loads \cite{Timoshenko1961a}:
  \begin{equation}\label{eq:def_rad}
  \delta = \frac{PR^3}{4EI_1} \left(\pi-\frac{8}{\pi} \right)
  \end{equation}

\begin{figure}
  \centering
  \includesvg{\rootdir/figs/acoustic_testing/}{mech_testing}
  \caption{\textbf{(a)} Four-point bending test. A small mirror resting on the rim at 9-o’clock reflects the laser spot onto a grid (to the right, not shown). The un-balanced configuration effectively doubles the lateral displacement at the load point and increases sensitivity. \textbf{(b)} Selected radial load-displacement curves under diametral compression. Black triangles = Alex ALX295, red circles = Alex Y2000 26'', blue squares = Sun CR18 700c.}
  \label{fig:mech_tests}
\end{figure}

\subsection{Four-point bending test}

The lateral stiffness of each bicycle rim was also measured using a four-point bending test (Figure \ref{fig:mech_tests} (a)). The rim was supported at 3- and 9-o'clock by cylindrical rods and constrained against a rigid bracket on the top surface of the rim at 12-o'clock. The rim was then loaded by hanging a weight from the spoke hole (or valve hole) at 6-o'clock. The vertical deflection at 6-o'clock was measured using a dial indicator and the rotation of the cross-section at 9-o'clock was measured by an optical lever with a diode laser.

If warping is neglected and only the strain energy due to lateral bending and uniform torsion are considered, Castigliano's method yields the displacement $u_l$ at the load point and the rotation of the cross-section $\phi_s$ at the left support (see Online Resource 1):
  \begin{equation}\label{eq:u_4p}
  \begin{array}{c}
  u_l = -\left(\frac{PR^3}{2GJ}\right) [(2(3-\pi)+\mu(2-\pi)]\\
  \phi_s = -\left(\frac{PR^3}{8GJ}\right) (1+\mu)(2-\pi)
  \end{array}
  \end{equation}

By simultaneously measuring the deflection and rotation, $GJ$ and $\mu$ can be determined from a single test.

\section{Results and discussion}

\subsection{Radial stiffness}

After identifying the first several mode frequencies in each spectrum, $\sqrt{EI_1/2\pi R^3M}$ was estimated from the fundamental ($n=2$) mode. With knowledge of $R$ and $M$, the radial bending stiffness was determined from Eqn. \ref{eq:fn_rad}.

The results for the radial stiffness $EI_1$ are shown in Fig. \ref{fig:bar}. The error estimates are made on the assumption that the mass and radius are both known to within 1\%. The uncertainty in the frequency is the greater of either the estimated parameter variance from the Lorentzian fit, or the frequency resolution of the spectral average. Multiple modes may be averaged together to estimate $EI_1$, however the deviation from Eqn. \ref{eq:fn_rad} grows steadily larger with higher mode number due to the fact that shorter wavelengths interact with spoke holes and other inhomogeneities.

  % \begin{figure*}
  %   \includegraphics[width=1.0\textwidth]{Fig4.eps}
  %   \caption{Comparison of stiffness parameters calculated from the acoustic test vs. quasistatic load-displacement tests. For $EI_1$ and $GJ$, refer to the left scale. For $\mu$, refer to the right scale.}
  %   \label{fig:bar}
  % \end{figure*}

\subsection{Lateral-torsional stiffness}

Lateral bending and torsion are coupled in out-of-plane deformation modes of circular beams. Therefore, information from multiple modes must be used to calculate $GJ$ and $\mu$. Taking the ratio of two lateral-torsional frequencies and solving for $\mu$ in Eqn. \ref{eq:fn_lat} gives
  \begin{equation}\label{eq:mu}
  \mu = \frac{16 - (f_3^{lat}/f_2^{lat})^2}{9(f_3^{lat}/f_2^{lat})^2 - 64)}
  \end{equation}

After calculating $\mu$, $GJ$ can be calculated from Eqn. \ref{eq:fn_lat} by setting $n=2$:
  \begin{equation}\label{eq:GJ}
  GJ = \left(\frac{4\mu + 1}{18} \right ) \pi R^3M(f_2^{lat})^2
  \end{equation}

Qualitatively, $GJ$ scales the magnitude of the frequencies and $\mu$ scales the spacing between modes. However, the situation is further complicated by the fact that the cross-section of the rim does not remain perfectly planar. This additional warping deformation introduces a length scale into the torsional stiffness which depends on the rim radius and mode number. In this case the effective torsional response involves both $GJ$ and $EI_w$, where $I_w$ is the warping constant.

Results from the acoustic test and four-point bending test are shown Fig. \ref{fig:bar}. The value for $\mu$ is first calculated using Eqn. \ref{eq:mu} and then $GJ$ is determined from Eqn. \ref{eq:fn_lat} using the frequency of the $n=2$ lateral-torsional mode. We choose to fit $GJ$ because it is generally smaller than $EI_{2}$ and therefore dominates the total flexibility.

The error estimates in Fig. \ref{fig:bar} are made on the same assumptions as for $EI_1$. Due to the non-linearity of Eqn. \ref{eq:mu}, error estimates for $\mu$ are calculated using the Monte-Carlo method (see Online Resource 1). The lateral bending stiffness and torsion stiffness are geometrically coupled in lateral deformations. The total lateral-torsional stiffness depends on $EI_{2}$ and $GJ$ as though they were springs connected in series. The smaller stiffness dominates the overall stiffness of series-connected springs. Therefore $GJ$ (generally smaller than $EI_{2}$) can be determined with much higher precision than $EI_{2}$ or $\mu$. Even a small uncertainty on $f_3^{lat}/f_2^{lat}$ results in a large estimated uncertainty on $\mu$ and $EI_{2}$, but not $GJ$.

\subsection{Lateral-torsional mode stiffness}
An acoustic test is sufficient to calculate $GJ$ to within 11\% of the results from the four-point bending test. However, both models assume that warping is negligible. In fact, the acoustic test may be even more accurate than the four-point bending test because it directly measures the mode stiffness of the rim, which includes bending, pure torsion, and warping. In order to account for warping, we derive the frequency equation for lateral-torsional vibrations with an additional term for the warping resistance:

The differential equations of dynamic equilibrium, including warping but neglecting the rotary inertia of the rim cross-section, are
  \begin{subequations}\label{eq:lat_eq}
  \begin{align}
  \frac{EI_{2}}{R^4}\left(\frac{d^4u}{d\theta^4}-R\frac{d^2\phi}{d\theta^2}\right) +
  \frac{EI_w}{R^6} \left(\frac{d^4u}{d\theta^4}+R\frac{d^4\phi}{d\theta^4}\right) -
  \frac{GJ}{R^4}\left(\frac{d^2u}{d\theta^2}+R\frac{d^2\phi}{d\theta^2}\right) +
  \left(\frac{M}{2\pi R}\right) \frac{d^2u}{dt^2} = 0\\
  \frac{EI_{2}}{R^3}\left(\frac{d^2u}{d\theta^2}-R\phi\right) -
  \frac{EI_w}{R^5}\left(\frac{d^4u}{d\theta^4}+R\frac{d^4\phi}{d\theta^4}\right) +
  \frac{GJ}{R^3}\left(\frac{d^2u}{d\theta^2}+R\frac{d^2\phi}{d\theta^2}\right) = 0
  \end{align}
  \end{subequations}

We are seeking free vibrations of the form
  \begin{subequations}\label{eq:harmonic}
  \begin{align}
  u(\theta, t)=u_n e^{in\theta}e^{i\omega t}\\
  \phi(\theta, t)=\phi_n e^{in\theta}e^{i\omega t}
  \end{align}
  \end{subequations}

Inserting Eqns. \ref{eq:harmonic} into Eqns. \ref{eq:lat_eq} yields a linear system of the form $\mathbf{A}\cdot [u_n, \phi_n]^T=\mathbf{0}$. Non-trivial solutions exist when the determinant of the matrix $\mathbf{A}$ vanishes. Using this condition to solve for the angular frequency $\omega$ yields the frequency equation:
  \begin{equation}\label{eq:omega}
  \omega^2 = \frac{2\pi n^2(n^2-1)^2EI_{2}\left(GJ + \frac{EI_w}{R^2}n^2 \right)}
                  {MR^3 \left( EI_{2} + GJn^2 + \frac{EI_w}{R^2}n^4\right)}
  \end{equation}

Exploiting the analogy with the simple harmonic oscillator, for which $\omega^2=K/M$, allows us to calculate an effective rim stiffness of the $n$th mode:
  \begin{equation}\label{eq:keff}
  K_{rim,n} = 2 \left( \frac{R^3}{\pi n^2(n^2-1)^2\left( GJ + \frac{EI_w}{R^2}n^2\right)} +
    \frac{R^3}{\pi(n^2-1)^2EI_{2}} \right)^{-1}
  \end{equation}

Comparing Eqn. \ref{eq:keff} with Eqns. 34 and 35 in reference \cite{Ford2016a}, it's clear that the effective stiffness is twice the series combination of the rim bending stiffness and torsion stiffness. Even if $EI_{2}$ and $GJ$ cannot be reliably determined independently, the series combination can be directly determined from the relation $f_n^{lat} = (2\pi)^{-1}\sqrt{K_{rim,n}/M}$.

  % \begin{figure}[h!]
  %   \includegraphics[width=1.0\columnwidth]{Fig5.eps}
  %   \caption{Comparison of mode stiffness from the acoustic test (x-axis) and the mode stiffness calculated from Eqn. \ref{eq:keff} and the stiffness parameters determined from the four-point bending test. The error increases at higher mode numbers, likely due to the length-scale dependence of the warping stiffness.}
  %   \label{fig:keff}
  % \end{figure}


\section{Undergraduate lab activity}


\end{document}
