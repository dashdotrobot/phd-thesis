%!TEX root = ../../thesis.tex
\providecommand{\rootdir}{../..}
\documentclass[\rootdir/thesis.tex]{subfiles}

\begin{document}

As shown in Equation \eqref{eq:EulerLagrange}, the linear equilibrium equations for the rim alone factor into two independent sets of equations for in-plane $(v-w)$ displacements out-of-plane $u-\phi$ displacements. 

\subsection{Wheels with offset spokes: ``fat bikes''}



\subsection{Asymmetric wheels}

An inwards radial force at a spoke gives rise to both a radial displacement in the direction of the load and a lateral displacement opposite the hub flange to which the spoke is connected. In a symmetric wheel, this reaction is equal and opposite at opposing spokes and this radial-lateral coupling disappears when the spoke stiffness is homogenized. However, if the spoke inclination angle is different one side of the wheel, these reactions will be equal but not opposite, giving rise to a net radial--lateral coupling.

Rear wheels and wheels for disk brakes are commonly built with one hub flange closer to the rim centerline than the other due to the need to create space for the gear cluster or the disk to fit between the frame and the hub. The out-of-plane spoke angle is steeper on the side of the wheel with the gears or disk. Additionally, the spokes on the steep side must be tighter by a factor of $c_1^l/c_1^r$. For a typical wheel, this factor is on the order of 1.5, but can be higher depending on the flange width and number of sprockets.

It is often erroneously assumed that the lateral stiffness is different in the left and right directions on such a wheel. Experiments by Rinard have demonstrated that the stiffness is indistinguishable \cite{Rinard}, but it should also be clear from a theoretical perspective that the stiffness is the instantaneous \emph{slope} of the load-displacement curve, and that the spokes on each side give the same stiffness in tension as in ``compression'' (or more precisely, loss of tension).

\end{document}
