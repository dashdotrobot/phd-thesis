%!TEX root = ../../thesis.tex
\providecommand{\rootdir}{../..}
\documentclass[\rootdir/thesis.tex]{subfiles}

\begin{document}

\subsection{Deformation of the rim}
\label{sec:rim_strain_energy}

The rim is modeled as a circular beam with a constant, thin-walled cross-section having an axis of symmetry in the plane of the wheel. I adopt the standard Euler-Bernoulli assumptions:

\begin{enumerate}
	\item{The material behavior is linear-elastic.}\label{assum:elastic}
	\item{The cross-section is rigid with respect to in-plane deformation, except for out-of-plane warping deformation.}\label{assum:rigid}
	\item{Shear deformations can be neglected, except that associated with uniform torsion.}\label{assum:no_shear}
	\item{Displacements and rotations are infinitesimal.}\label{assum:infinitesimal}
\end{enumerate}

We first compute the continuum displacement field in the rim based on assumptions (\ref{assum:rigid}) and (\ref{assum:no_shear}). Due to assumption (\ref{assum:rigid}), the displacement at any point in the cross-section is given by a rigid-body displacement of a suitable reference point, a rigid-body rotation about that reference point, followed by a normal displacement given by the rate-of-twist curvature multiplied by the normalized warping function \cite{Timoshenko1961}. For a monosymmetric beam, the most convenient reference point is the shear center, which is the unique point in the rim cross-section at which an applied shear load produces no twist\footnote{For an introduction to torsion of thin-walled beams, see Barber \cite{Barber2011}.}. Due to assumption (\ref{assum:infinitesimal}), these operations can be applied in any order.

After deformation, each point in the body displaces by \gls{uvec} and the basis vectors $\eo$, $\et$, $\eh$ rotate through angles $\gls{rot}_1$, $\gls{rot}_2$, $\gls{rot}_3$. The displacement vector at a point $(\x, \y, 0)$ in the cross-section is given by
\begin{equation}
\label{eq:u}
\gls{uvec} = \gls{usvec} + \left[\left(\p' - \frac{u'}{\R}\right)\gls{warp}\right]\eh +
\begin{bmatrix}
0        & -\gls{rot}_3& \gls{rot}_2\\
\gls{rot}_3 & 0        & -\gls{rot}_1\\
-\gls{rot}_2& \gls{rot}_1 & 0
\end{bmatrix}
\begin{bmatrix}\x\\\y - \gls{yo}\\0\end{bmatrix}
\end{equation}

where $\gls{usvec} = u\eo + v\et + w\eh$ is the displacement of the shear center, \gls{warp} is the normalized warping function defined at the shear center, and \gls{yo} is the height of the shear center relative to the centroid. The infinitesimal rotation angles are given by
\begin{subequations}
\label{eq:omega}
\begin{align}
\gls{rot}_1 = v' + w/\R\\
\gls{rot}_2 = u'\\
\gls{rot}_3 = \p
\end{align}
\end{subequations}

The deformation curvature \gls{curv} is found by differentiating the rotation vector \gls{rot} with respect to \gls{arc}, making use of the Frenet-Serret formulas: $\et' = -\eh/\R$ and $\eh' = \et/\R$. The result is:
\begin{subequations}
\label{eq:kappa}
\begin{align}
\kappa_1 = \left( v'' + \frac{w'}{\R} \right) \label{eq:kappa1}\\
\kappa_2 = \left( u'' + \frac{\p}{\R} \right) \label{eq:kappa2}\\
\kappa_3 = \left( \p' - \frac{u'}{\R} \right) \label{eq:kappa3}
\end{align}
\end{subequations}

We recognize these three components as the in-plane bending curvature, out-of-plane bending curvature, and twist. For understanding rim deformation, it is worth remarking that $\p=$ constant creates pure bending (ring eversion), while $u'=$ constant creates pure torsion (analogous to a helical spring) \cite{PapadopoulosPriv}.

\subsubsection{Strain-displacement relations}

The longitudinal strain at each point in the body is computed from the displacement field \eqref{eq:u}. In cylindrical coordinates:
\begin{align}
\label{eq:strain}
\begin{split}
\gls{elon} &= u_3' - \frac{1}{\R} u_2\\
           &= w' - \frac{v}{\R} - \x\left(u'' + \frac{\p}{\R}\right) +
                 	(\y-\gls{yo})\left(v'' + \frac{w'}{\R}\right) +
                 	\left(\p'' - \frac{u''}{\R}\right)\gls{warp}
\end{split}
\end{align}

The longitudinal strain distribution in \eqref{eq:strain} is identical to the linear part of the longitudinal strain derived by Trahair and Papangelis \cite{Trahair1987}, and Pi, et. al. \cite{Pi1995}. In deriving \eqref{eq:strain}, it is assumed that the initial curvature $1/\R$ is constant across the cross-section. For most bicycle rims, for which the ratio of rim radius to cross-section height typically exceeds 20, this is an excellent approximation. For very deep rims, \eqref{eq:strain} must be multiplied by the curvature factor $\R/(\R+\y)$, which greatly complicates integration of the section \cite{Kang1994,Lim2004,Ryu2012}.

The non-vanishing shear strain associated with uniform torsion is given by \cite{Pi1995,Kang1994}:
\begin{equation}
\label{eq:shear_strain}
\gls{shear} = 2 \xi \left(\p' - \frac{u'}{\R}\right)
\end{equation}

where $\xi$ is the normal distance from the midplane of the thin-walled section. The shear direction is directed normal to the thickness direction of the local section. Equation \eqref{eq:shear_strain} is appropriate for cross-sections assembled from multiple open and closed thin-walled profiles (which includes the vast majority of bicycle rims). An expression suitable for general, symmetrical bodies neglecting the curvature correction $\R/(\R+\y)$ is given by Pi, et. al. \cite{Bradford2006b}:
\begin{subequations}
\begin{align}
\gamma_{31} &= -\left(\y + \frac{\partial \gls{warp}}{\partial \x}\right)\left(\p' - \frac{u'}{\R}\right)\\
\gamma_{32} &= \left(\x + \frac{\partial \gls{warp}}{\partial \y}\right)\left(\p' - \frac{u'}{\R}\right)
\end{align}
\end{subequations}

In general, determining the warping function $\gls{warp}$ for an arbitrary cross-section is difficult and must be obtained numerically \cite{Timoshenko1961}.


\subsubsection{Strain energy}
Making use of assumption (\ref{assum:elastic}), the strain energy in the rim due to the linearized displacement field is given by
\begin{equation}
\label{eq:U_rim_gen}
U_{rim} = \frac{1}{2}\int_0^{2\pi \R} \int_{\gls{Arim}} (\E\gls{elon}^2 + \G\gamma^2) \,d\gls{Arim}\,\, d\gls{arc}
\end{equation}

where $\E$ and $\G$ are the Young's modulus and shear modulus, respectively. Substituting \eqref{eq:strain} and \eqref{eq:shear_strain} into \eqref{eq:U_rim_gen} and integrating over the rim cross-section yields
\begin{multline}
\label{eq:U_rim_uvw}
U_{rim} = \frac{1}{2}\int_0^{2\pi \R}
	\EA  \left(w' - \frac{v}{\R} - \gls{yo}\left(v'' + \frac{w'}{\R}\right) \right)^2 +
	\EIr \left(v'' + \frac{w'}{\R}\right)^2 +\\
	\EIl \left(u'' + \frac{\p}{\R}\right)^2 +
	\EIw \left(\p'' - \frac{u''}{\R}\right)^2 +
	\GJ  \left(\p' - \frac{u'}{\R}\right)^2 d\gls{arc}
\end{multline}

Equation \eqref{eq:U_rim_uvw} is derived with the help of the following relations:
\begin{gather}
\int \x\, d\gls{Arim} = \int \y\, d\gls{Arim} = \int \x\y\, d\gls{Arim} = 0\\
\int d\gls{Arim} = \gls{Arim},\,\,\, \int \x^2\, d\gls{Arim} = \gls{Ilat},\,\,\, \int \y^2\, d\gls{Arim} = \gls{Irad},\,\,\,
\int 4\xi^2\, d\gls{Arim} = \gls{Jtor},\,\,\, \int \gls{warp}^2\, d\gls{Arim} = \gls{Iwarp}
\end{gather}

Noting that the longitudinal strain at the centroid is given by $\gls{ec} = w' - v/\R - \gls{yo}(v'' + w'/\R)$ and substituting the curvatures \eqref{eq:kappa} into \eqref{eq:U_rim_uvw}, we obtain
\begin{equation}
\label{eq:U_rim}
U_{rim} = \frac{1}{2}\int_0^{2\pi \R} \EA\gls{ec}^2 + \EIr\kappa_1^2 + \EIl\kappa_2^2 + \GJ\kappa_3^2 + \EIw(\kappa_3')^2\, d\gls{arc}
\end{equation}

\subsubsection{Strain energy in a general deformed configuration}

Any general deformation of the bicycle wheel may be represented as a transition from an unstressed state $\mathcal{S}_0$ to a prestressed state $\mathcal{S}_p$, and then to a deformed state $\mathcal{S}_d$. The total displacement field is given by
\begin{equation}
\label{eq:u_total}
\gls{uvec} = \gls{uvec}_p + \delta\gls{uvec}
\end{equation}

where $\gls{uvec}_p$ is the displacement field for $\mathcal{S}_0 \rightarrow \mathcal{S}_p$ and $\delta\gls{uvec}$ is the displacement field for $\mathcal{S}_p \rightarrow \mathcal{S}_d$. Inserting \eqref{eq:u_total} into \eqref{eq:U_rim} and adopting the same notation conventions for the prestressed and deformed configurations, we obtain
\begin{equation}
\label{eq:U_rim_total}
U_{rim} = U_{rim}^p + \delta U_{rim}^{p\delta} + U_{rim}^{\delta}
\end{equation}

where $U_{rim}^p$ is the strain energy due to $\gls{uvec}_p$ alone, $U_{rim}^{\delta}$ is the strain energy due to $\delta\gls{uvec}$ alone, and the cross-term is defined as
\begin{equation}
\label{eq:U_rim_pd}
\delta U_{rim}^{p\delta} = \int_0^{2\pi \R}
	\EA\gls{ec}^p \delta\gls{ec} +
	\EIr\kappa_1^p\delta\kappa_1 +
	\EIl\kappa_2^p\delta\kappa_2 +
	\GJ\kappa_3^p\delta\kappa_3 +
	EI_w(\kappa_3^p)'\delta\kappa_3'\, d\gls{arc}
\end{equation}

This term is the first variation of the strain energy $U_{rim}$ in the prestressed configuration with respect to a virtual displacement $\delta\gls{uvec}$.

\end{document}