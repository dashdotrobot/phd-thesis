\providecommand{\rootdir}{../..}
\documentclass[\rootdir/thesis.tex]{subfiles}

\begin{document}

The radial-tangential equations are
\begin{subequations}
\begin{align}
\label{eq:rad_tan_2}
(EI_1 + EAy_0^2)\left( \ds{v}{4} + \frac{1}{R}\ds{w}{3} \right) -
    \frac{EA}{R}\left( \frac{dw}{ds} - \frac{v}{R} +y_0\left(2\ds{v}{2} + \frac{1}{R}\frac{dw}{ds} -
    R\ds{w}{3}\right) \right) +\bar{k}_{vv}v &= f_v\\
-\left( \frac{EA_1}{R} + EAy_0\left( 1 + \frac{y_0}{R} \right) \right)
    \left(\ds{v}{3} + \frac{1}{R}\ds{w}{2} \right) -
    EA\left(1+\frac{y_0}{R} \right) \left(\ds{w}{2} - \frac{1}{R}\frac{dv}{ds} \right) + \bar{k}_{ww}w &= f_w
\end{align}
\end{subequations}

Combining these two equations (equivalent to taking the determinant of the $v,w$ submatrix in Equation \eqref{eq:rad_tan_2}) results in a single governing equation for $v$:
\begin{multline}
\label{eq:rad_tan_ry}
\dt{v}{6} + \left[2-\lambda_{ww}\left(\left(\frac{y_0}{R}\right)^2 +
                                        \left(\frac{r_y}{R}\right)^2 \right) \right] \dt{v}{4}\\
          + \left[1+\lambda_{vv}\left(\left(1+\frac{y_0}{R}\right)^2+\left(\frac{r_y}{R}\right)^2\right)
                   +2\lambda_{ww}\left(\frac{y_0}{R}\right)\right]\dt{v}{2}
          - \lambda_{ww}\left[1+\lambda_{vv}\left(\frac{r_y}{R}\right)^2\right] v = 0
\end{multline}

where $\lambda_{vv}=\bar{k}_{vv}R^4/EI_1$ and $\lambda_{ww}=\bar{k}_{ww}R^4/EI_1$. The tangential spoke stiffness $\bar{k}_{ww}$ is related to the projection of the spoke stiffness along the tangential direction. For practical wheels, $\bar{k}_{ww}$ is at least 2 orders of magnitude smaller than $\bar{k}_{vv}$. Analytical solutions to Equation \eqref{eq:rad_tan_ry} are possible because the roots of the characteristic equation come in three pairs, $\pm r_i$.

Further simplification is possible for most practical cases by noting that $y_0,r_y \ll R$. These conditions are equivalent to assuming that the beam is doubly-symmetric, and that extension of the centerline can be neglected, respectively.
\begin{equation}
\label{eq:rad_tan}
\dt{v}{6} + 2\dt{v}{4}+(1+\lambda_{vv})\dt{v}{2} - \lambda_{ww}v=0
\end{equation}

Equation \eqref{eq:rad_tan} is the same as Pippard’s result obtained from equilibrium of a differential element of the rim24. The boundary conditions and solution procedure is identical for Equations \eqref{eq:rad_tan} and \eqref{eq:rad_tan_ry}.

\subsubsection{Loading case I: radial point load}

The reaction force from the road on the wheel may be represented by a radial point load at $\theta=0$. The radial displacement closely resembles the classical solution of a point load acting on a beam supported by an elastic foundation\cite{Hetenyi}. The radial deformation is confined to a narrow arc around the load point in which spokes lose tension, bounded by an ``overshoot'' region where spokes tensions increase and reach a maximum within about 30$^{\circ}$ of the load point. Far from the load, spoke tensions generally change by a very small amount on the order of 5\% of the applied load. This has led some authors to claim that ``the hub stands on the spokes beneath it,'' despite the counter-intuitive image this conjures\cite{Brandt}. Others insist that the hub ``hangs from the spokes above it'' due to the fact that the spoke tensions above the hub are higher than those below it.  Both statements are mathematically equivalent, and it is clear that the lower spokes play the most significant dynamic role in supporting the bicycle and are most prone to loosening or buckling under load.

\begin{figure}
\centering
\includesvg{\rootdir/figs/stress_analysis/}{rad_tan_grid}
\caption{(a)-(c) Deformation of a wheel subject to radial point load, normalized by $P/\pi R\bar{k}_{vv}$. For the dark lines $\lambda_{vv}=1000$ and for the light lines $\lambda_{vv}=10$. (d)-(f) Deformation of a wheel subject to a tangential point load.}
\label{fig:rad_tan_grid}
\end{figure}

If, as is generally the case for practical wheels, $\lambda_{ww} \ll \lambda_{ww}$, or more rigorously for the case of purely radial spokes, Equation \eqref{eq:rad_tan} simplifies to
\begin{equation}
\label{eq:rad}
\dt{v}{4} + 2\dt{v}{2} + (1+\lambda_{vv})v=0
\end{equation}

Since all the derivatives have even order, a relatively simple analytical solution to \eqref{eq:rad} exists. The most relevant loading case for the bicycle wheel is approximated by a point load $P$ located at $\theta=0$ corresponding to the reaction force from the road. In this case the radial displacement is given by
\begin{equation}
\label{eq:rad_soln}
v = \frac{PR^3}{4abEI_1} \left( \frac{2ab}{\pi\eta^2} + \frac{b\sinh{a\theta}\cos{b\theta}}{\eta} 
                               -\frac{a\cosh{a\theta}\sin{b\theta}}{\eta}
                               +A\cosh{a\theta}\cos{b\theta} + B\sinh{a\theta}\sin{b\theta}\right)
\end{equation}

where
\begin{gather*}
\eta=\sqrt{\lambda_{vv} + 1}, \,\,\,\, a=\sqrt{\frac{\eta-1}{2}}, \,\,\,\, b=\sqrt{\frac{\eta+1}{2}}\\
A = -\frac{a\sin{2\pi b} + b\sinh{2\pi a}}{2\eta(\sinh^2{\pi a} + \sin^2{\pi b})}\\
B =  \frac{a\sinh{2\pi a} - b\sin{2\pi b}}{2\eta(\sinh^2{\pi a} + \sin^2{\pi b})}
\end{gather*}

This solution for radial spokes was first given in slightly different form by Smith\cite{Smith} in 1901 and later by Pippard\cite{Pippard} in 1931, who also performed experiments on spoked wheels. The non-dimensionalized radial stiffness depends only on the stiffness ratio $\lambda_{vv}$:
\begin{equation}
\label{eq:Krad}
\frac{K_{rad}}{\pi R \bar{k}_{vv}} =
    \frac{1}{\pi\lambda_{vv}} \left(\frac{1}{2\pi\eta^2}
                                    -\frac{b\sinh{a\pi}\cosh{a\pi} + a\sin{b\pi}\cos{b\pi}}
                                     {4ab\eta (\sinh^2{a\pi} + \sin^2{b\pi})} \right)^{-1}
\end{equation}

\subsubsection{Loading case II: tangential point load}

During acceleration (or braking using disc brakes), the reaction force from the road has a component in the tangential direction. The tangential displacement is primarily controlled by the stiffness $\bar{k}_{ww}$, while the ratio $\lambda_{ww}/\lambda_{vv}$ controls the degree to which radial displacement is also involved.

A very satisfactory approximation to the problem of a tangential load can be obtained by noting that $\lambda_{vv}\gg\lambda_{ww}$ and therefore the radial displacement is very small compared with the tangential displacement. Under this approximation, the rim rotates about the axle as a rigid body and the tangential stiffness is simply
\begin{equation}
\label{eq:Ktan}
K_{tan} = 2\pi R \kww
\end{equation}

\subsubsection{The role of spoke tension in supporting radial and tangential loads}

The model considered here depends on the assumption that a pretensioned spoke can equally support tension or compression (or rather, loss of tension) and that the elastic properties remain constant. In order for this assumption to be valid, all of the spokes must maintain a non-zero tension at all times. This condition may be violated if an excessive load is applied to the wheel.

Figure \ref{fig:radtan_Tinf} shows the change in spoke tension under different loading scenarios. In this example, the most critical spoke supports 40\% of the applied load, while the load sharing fractions for the nearest-neighbor and next-nearest-neighbor spokes are about 24\% and 3\%, respectively. A properly tensioned wheel should not admit spokes go slack under typical loads.

\begin{figure}[h]
\centering
\includesvg{\rootdir/figs/stress_analysis/}{rad_tan_Tinf}
\caption{Change in spoke tension under different loading scenarios: (left column) unit radial load, (center column) unit tangential load, (right column) 500 N radial load and 50 N tangential load. Each row corresponds to a different spoke pattern. A wheel with radial spokes cannot support tangential loads without significant non-linear deformation.}
\label{fig:radtan_Tinf}
\end{figure}

On a typical wheel with ``cross'' laced spokes, half of the spokes are inclined forward in the plane of the wheel, while the other half are inclined backwards. These are referred to as ``pushing'' and ``pulling'' or ``leading'' and ``trailing'' spokes due to their behavior under load. Pulling spokes increase their tension under acceleration torque while pushing spokes decrease their tension, as shown in Figure \ref{fig:radtan_Tinf} (b). Under a combined radial load and acceleration torque, the primary factor causing spokes to slacken is the radial load, while the primary factor causing spokes to tighten is the tangential load. Therefore, both types of loads should be accounted for when making fatigue calculations.

\end{document}