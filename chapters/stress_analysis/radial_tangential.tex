%!TEX root = ../../thesis.tex
\providecommand{\rootdir}{../..}
\documentclass[\rootdir/thesis.tex]{subfiles}

\begin{document}

The radial-tangential equations are
\begin{subequations}
\begin{align}
\label{eq:rad_tan_2}
(\EIr + \EA \gls{yo}^2)\left( \ds{v}{4} + \frac{1}{\R}\ds{w}{3} \right) -
    \frac{\EA}{\R}\left( \frac{dw}{ds} - \frac{v}{\R} +\gls{yo}\left(2\ds{v}{2} + \frac{1}{\R}\frac{dw}{ds} -
    \R\ds{w}{3}\right) \right) +\bar{k}_{vv}v &= f_v\\
-\left( \frac{\EIr}{\R} + \EA \gls{yo}\left( 1 + \frac{\gls{yo}}{\R} \right) \right)
    \left(\ds{v}{3} + \frac{1}{\R}\ds{w}{2} \right) -
    \EA\left(1+\frac{\gls{yo}}{\R} \right) \left(\ds{w}{2} - \frac{1}{\R}\frac{dv}{ds} \right) + \bar{k}_{ww}w &= f_w
\end{align}
\end{subequations}

Combining these two equations (equivalent to taking the determinant of the $v,w$ submatrix in Eqn. \eqref{eq:rad_tan_2}) results in a single governing equation for $v$:
\begin{multline}
\label{eq:rad_tan_ry}
\dt{v}{6} + \left[2-\lr_{ww}\left(\left(\frac{\gls{yo}}{\R}\right)^2 +
                                        \left(\frac{\ry}{\R}\right)^2 \right) \right] \dt{v}{4}\\
          + \left[1+\lr_{vv}\left(\left(1+\frac{\gls{yo}}{\R}\right)^2+\left(\frac{\ry}{\R}\right)^2\right)
                   +2\lr_{ww}\left(\frac{\gls{yo}}{\R}\right)\right]\dt{v}{2}
          - \lr_{ww}\left[1+\lr_{vv}\left(\frac{\ry}{\R}\right)^2\right] v = 0
\end{multline}

where $\lr_{vv}=\kvv\R^4/\EIr$ and $\lr_{ww}=\kww\R^4/\EIr$. The tangential spoke stiffness $\kww$ is related to the projection of the spoke stiffness along the tangential direction. For practical wheels, $\kww$ is at least 2 orders of magnitude smaller than $\kvv$. Analytical solutions to Eqn. \eqref{eq:rad_tan_ry} are possible because the roots of the characteristic equation come in three pairs, $\pm r_i$.

Further simplification is possible for most practical cases by noting that $\gls{yo},\ry \ll \R$. These conditions are equivalent to assuming that the beam is doubly-symmetric, and that extension of the centerline can be neglected, respectively.
\begin{equation}
\label{eq:rad_tan}
\dt{v}{6} + 2\dt{v}{4}+(1+\lr_{vv})\dt{v}{2} - \lr_{ww}v=0
\end{equation}

Equation \eqref{eq:rad_tan} is the same as Pippard’s result obtained from equilibrium of a differential element of the rim \cite{Pippard1932b}. The boundary conditions and solution procedure is identical for \eqref{eq:rad_tan} and \eqref{eq:rad_tan_ry}.

\subsection{Loading case I: radial point load}

The reaction force from the road on the wheel may be represented by a radial point load at $\gls{ang}=0$. The radial displacement closely resembles the classical solution of a point load acting on a beam supported by an elastic foundation \cite{Hetenyi1946}. The rim bends inwards in a narrow arc near the load and squashes outwards on either side of this region due to the tendency of the rim to maintain its total circumference constant\footnote{The spoke tensions on either side of the affected length can increase significantly if the bottommost lose tension. In one test on a 20'' wheel, the spoke nipples in this region failed, ejecting the spokes like arrows from a bow. No one was injured.}. Far from the load, spoke tensions generally change by a very small amount on the order of \SI{5}{\percent} of the applied load. This has led some to claim that ``the hub stands on the spokes beneath it,'' despite the counter-intuitive image this conjures \cite{Brandt1993,Forester1980}. Others insist that the hub ``hangs from the spokes above it'' due to the fact that the spoke tensions above the hub are higher than those below it \cite{Fine1998}. Both statements are mathematically equivalent, but it is clear that the lower spokes play the most significant dynamic role in supporting the bicycle and are most prone to loosening or buckling under load.

\begin{figure}
\centering
\includesvg{\rootdir/figs/stress_analysis/}{rad_tan_grid}
\caption{(a)-(c) Deformation of a wheel subject to radial point load, normalized by $P/\pi\R\kvv$. For the dark lines $\lr_{vv}=1000$ and for the light lines $\lr_{vv}=10$. (d)-(f) Deformation of a wheel subject to a tangential point load.}
\label{fig:rad_tan_grid}
\end{figure}

If, as is generally the case for practical wheels, $\lr_{ww} \ll \lr_{ww}$, Eqn. \eqref{eq:rad_tan} simplifies to
\begin{equation}
\label{eq:rad}
\dt{v}{4} + 2\dt{v}{2} + (1+\lr_{vv})v=0
\end{equation}

Since all the derivatives have even order, a relatively simple analytical solution to \eqref{eq:rad} exists. The radial displacement under a point load $P$ at $\gls{ang}=0$ is given by
\begin{equation}
\label{eq:rad_soln}
v = \frac{P\R^3}{4ab\EIr} \left( \frac{2ab}{\pi\eta^2} + \frac{b\sinh{a\gls{ang}}\cos{b\gls{ang}}}{\eta} 
                               -\frac{a\cosh{a\gls{ang}}\sin{b\gls{ang}}}{\eta}
                               +A\cosh{a\gls{ang}}\cos{b\gls{ang}} + B\sinh{a\gls{ang}}\sin{b\gls{ang}}\right)
\end{equation}

where
\begin{gather*}
\eta=\sqrt{\lr_{vv} + 1}, \,\,\,\, a=\sqrt{\frac{\eta-1}{2}}, \,\,\,\, b=\sqrt{\frac{\eta+1}{2}}\\
A = -\frac{a\sin{2\pi b} + b\sinh{2\pi a}}{2\eta(\sinh^2{\pi a} + \sin^2{\pi b})}\\
B =  \frac{a\sinh{2\pi a} - b\sin{2\pi b}}{2\eta(\sinh^2{\pi a} + \sin^2{\pi b})}
\end{gather*}

This solution for radial spokes was first given in slightly different form by Smith \cite{Smith1901} in 1901 and later by Pippard \cite{Pippard1931} and Hetenyi \cite{Hetenyi1946}. The radial stiffness is
\begin{equation}
\label{eq:Krad}
\gls{Krad} =
    \frac{\pi\R\kvv}{\pi\lr_{vv}} \left(\frac{1}{2\pi\eta^2}
                                        -\frac{b\sinh{a\pi}\cosh{a\pi} + a\sin{b\pi}\cos{b\pi}}
                                              {4ab\eta (\sinh^2{a\pi} + \sin^2{b\pi})} \right)^{-1}
\end{equation}

\subsection{Loading case II: tangential point load}

During acceleration (or deceleration with disc brakes\footnote{With rim brakes, the torque from the road force is reacted by the force applied by the brake pads. The spokes at the front of the wheel lose tension while the spokes at the back gain tension.}), the reaction force from the road has a component in the tangential direction. The tangential displacement is primarily controlled by the stiffness $\kww$, while the ratio $\lr_{ww}/\lr_{vv}$ controls the degree to which radial displacement is also involved.

A very satisfactory approximation to the problem of a tangential load can be obtained by noting that $\lr_{vv}\gg\lr_{ww}$ and therefore the radial displacement is very small compared with the tangential displacement. Under this approximation, the rim rotates about the axle as a rigid body and the tangential stiffness is
\begin{equation}
\label{eq:Ktan}
\gls{Ktan} = 2\pi\R \kww
\end{equation}

\subsection{The role of spoke tension in supporting in-plane loads}

\begin{figure}
\centering
\includesvg{\rootdir/figs/stress_analysis/}{rad_tan_Tinf}
\caption{Change in spoke tension under different loading scenarios: unit radial load (left column), unit tangential load (center column), \SI{500}{N} radial load and \SI{50}{N} tangential load (right column). Each row corresponds to a different spoke pattern. The red bar represents the spoke at the load point. (A wheel with radial spokes cannot support tangential loads without significant non-linear deformation, therefore those plots have been omitted.)}
\label{fig:radtan_Tinf}
\end{figure}

The model considered here depends on the assumption that a pretensioned spoke can equally support tension or compression (or rather, loss of tension). In order for this assumption to be valid, all of the spokes must maintain positive tension at all times. This condition may be violated if an excessive load is applied to the wheel.

Figure \ref{fig:radtan_Tinf} shows the change in spoke tension for a typical road bike wheel\footnote{See Appendix \ref{app:std_research_wheel} for complete wheel properties.} under different loading scenarios. The spokes are given an initial pretension of \SI{800}{N}. Under a radial load, the most critical spoke supports \SI{44}{\percent} of the applied load, while the load sharing fractions for the nearest-neighbor and next-nearest-neighbor spokes are about \SI{24}{\percent} and \SI{3}{\percent}, respectively. In a properly tensioned wheel spokes should not go slack under typical loads.

On a typical wheel with tangent spokes, half of the spokes are inclined forward in the plane of the wheel, while the other half are inclined backwards. These are referred to as ``pushing'' and ``pulling'' or ``leading'' and ``trailing'' spokes due to their behavior under torque. Pulling spokes increase their tension under acceleration torque while pushing spokes decrease their tension, as shown in Fig. \ref{fig:radtan_Tinf} (middle column). Under a combined radial load and acceleration torque, the primary factor causing spokes to slacken is the radial load, while the primary factor causing spokes to tighten is the tangential load. Therefore, both types of loads should be accounted for when making fatigue calculations.

\end{document}
