%!TEX root = ../../thesis.tex
\providecommand{\rootdir}{../..}
\documentclass[\rootdir/thesis.tex]{subfiles}

\begin{document}

Under lateral loads, the rim bends and twists into a non-planar shape. The wheel is considerably more flexible in the lateral direction than in the radial direction due to the small lateral projection of the spokes. When the rim undergoes lateral deformation, the potential energy of the compressive load induced by the spoke pretension is reduced. Due to the large lateral compliance, this reduction in potential energy can be significant compared to the increase in strain energy due to lateral bending and twisting. This leads to larger lateral deflections with the possibility of lateral-torsional instability at a sufficiently high spoke tension.

The lateral-torsional equations of \eqref{eq:EulerLagrange} are
\begin{subequations}
\label{eq:lat_ode_2}
\begin{align}
\begin{split}
  \EIl\left( \ds{u}{4} + \frac{1}{\R}\ds{\p}{2} \right)
  - \frac{\EIw}{\R^2}\left( \R\ds{\p}{4} - \ds{u}{4} \right)
  + \frac{\GJ}{\R^2}\left( \R\ds{\p}{2} - \ds{u}{2} \right)\\
  + \R\Tb\left( \ds{u}{2} + \yo\ds{\p}{2}
                  -\frac{\ro^2}{\R^2}\left(\R\ds{\p}{2}
                                          -\ds{u}{2}\right) \right)
  + \kuu u + \kup \p = f_u
\end{split}\\
\begin{split}
  \frac{\EIl}{\R}\left( \ds{u}{2} + \frac{1}{\R}\p \right)
  + \EIw\left( \ds{\p}{4} - \frac{1}{\R}\ds{u}{4} \right)
  - \GJ\left( \ds{\p}{2} - \frac{1}{\R}\ds{u}{2} \right)\\
  + \R\Tb\left(\ro^2\left(\ds{\p}{2} - \frac{1}{\R}\ds{u}{2}\right)
                  +\yo\left( \ds{u}{2} + \frac{\p}{\R} \right)\right)
  + \kpp \p + \kup u = m
\end{split}
\end{align}
\end{subequations}

This coupled system leads to an eighth-order ordinary differential equation with constant coefficients and all even order derivatives. Analytical formulae for the roots are possible, but impractical due to the need to find roots of a quartic characteristic polynomial. Neglecting warping stiffness and defining the non-dimensional groups $\lr_{uu}=\kuu \R^4/\EIl$, $\m=\GJ/\EIl$, and $\ts=\R^3\Tb/\EIl$, Equations \eqref{eq:lat_ode_2} become:
\begin{multline}
\label{eq:lat_ode_full}
  \left(1 - \frac{\ts}{\m}\frac{\ro^2}{\R^2}\right)\dt{u}{6}\\
  +\left(2 - \frac{\lr_{\p\p}}{\m} + \ts + \ts\frac{\yo}{\R}\left(2 + \frac{1}{\m}\right)
         -2\frac{\ts}{\m}\left(\frac{\ro^2}{\R^2}\right)
         -\frac{\ts^2}{\m}\left(\frac{\ro^2}{\R^2} - \frac{\yo^2}{\R^2} +
                               \frac{\ro^2\yo}{\R^3}\right) \right) \dt{u}{4}\\
  +\left(1 + \lr_{uu} + \lr_{\p\p} + 2\lr_{u\p}\left(1+\frac{1}{\m}\right)
         -\frac{\ts}{\m}\left(1 + \lr_{\p\p}
                             -\frac{\yo}{\R}(\m + 2\lr_{u\p})
                             +\frac{\ro^2}{\R^2}\left(1 + \lr_{uu} + \lr_{\p\p} + 2\lr_{u\p}\right)\right)\right.\\
         \left.-\frac{\ts^2\yo}{\m \R}\left(1 + \frac{\ro^2}{\R^2}\right)\right)\dt{u}{2}\\
  -\frac{1}{\m}\left(\lr_{uu}\left(1+t\frac{\yo}{\R}\right)
                      +\lr_{uu}\lr_{\p\p} - \lr_{u\p}^2\right)u = 0
\end{multline}

The terms arising from the Wagner moment are vanishly small. The ratio $\ro^2/\R^2$ is on the order of $10^{-4}\sim10^{-3}$ for a typical rim, while $\ts$ is of order $1\sim10$. Neglecting second-order quantities $\ro^2,\yo^2$ compared to $\R^2$, Equation \eqref{eq:lat_ode_full} becomes:
\begin{multline}
\label{eq:lat_ode}
  \dt{u}{6}
  +\left(2 - \frac{\lr_{\p\p}}{\m} + \ts + \ts\frac{\yo}{\R}\left(2 + \frac{1}{\m}\right)
         \right) \dt{u}{4}\\
  +\left(1 + \lr_{uu} + \lr_{\p\p} + 2\lr_{u\p}\left(1+\frac{1}{\m}\right)
         -\frac{\ts}{\m}\left(1 + \lr_{\p\p}
                             -\frac{\yo}{\R}(\m + 2\lr_{u\p})
                             \right)
         -\frac{\ts^2\yo}{\m \R}\right)\dt{u}{2}\\
  -\frac{1}{\m}\left(\lr_{uu}\left(1+\ts\frac{\yo}{\R}\right)
                      +\lr_{uu}\lr_{\p\p} - \lr_{u\p}^2\right)u = 0
\end{multline}

Dropping the terms in Equation \eqref{eq:lat_ode} involving $\ts, \yo, \lr_{u\p}, \lr_{\p\p}$, one recovers the differential equation derived by Pippard \cite{Pippard1932a}. The dimensionless tension $\ts$ can be interpreted as the spoke tension divided by the spoke tension which would create a compressive stress in the rim equal to the buckling load of a straight fixed-fixed column of length $2\pi \R$. The stiffness ratio $\lr_{uu}=\matl{\lr_{uu}}+\geom{\lr_{uu}}$ is composed of two parts: $\matl{\lr_{uu}}$, a term proportional to the elastic stiffness of the spokes and $\geom{\lr_{uu}}$, a term proportional to the spoke tension. For most bicycle wheels, $\geom{\lr_{uu}}$ can be obtained to a very good approximation as $\lr_{tens} = \R^2\gls{ls}\Tb/\EIl \approx \ts$.

\subsection{Solutions to \eqref{eq:lat_ode_2} by the equivalent springs method.}
\label{sec:equiv_springs}

Equation \eqref{eq:lat_ode} is a 6\textsuperscript{th}-order linear ordinary differential equation with constant coefficients. Since all the derivatives have even order, the roots of the characteristic equation can be solved analytically by solving a cubic equation. However, a straightforward approximation of arbitrary accuracy which preserves warping stiffness is possible using the mode matrix method described in Section \ref{sec:ModeMatrix}. Furthermore, if coupling between in-plane and out-of-plane deformations is precisely zero or neglected (i.e. $\kuv = \kuw = 0$), the spoke offset vector is sufficiently small ($\bs=0$), and the rim shear center offset, $\yo$, is neglected, then the lateral stiffnness of the wheel can be modeled as a system of equivalent springs with clear physical interpretations.

Under a point load, $P$, applied at $\gls{ang}=0$, the modal approximation \eqref{eq:u_Bd} to the lateral displacement becomes
\begin{equation}
\label{eq:u_kequiv}
u = u_0 + \sum_{n=1}^N u_n \cos{n\gls{ang}}
\end{equation}

The lateral stiffness $K_{lat}=P/u$ is found by combining the individual mode stiffnesses $P/u_n$ using the series-spring rule:
\begin{equation}
\label{eq:Klat_series}
\frac{1}{\gls{Klat}} = \frac{1}{K_0} + \frac{1}{K_1} + \frac{1}{K_2} + \dots
\end{equation}

The mode stiffnesses are found by solving the appropriate block of the mode matrix $\gls{KmRim} + \gls{KmSpk}$ with the simplifying assumptions described above:

\begin{subequations}
\begin{align}
K_0 &= 2\pi\R\kuu\\
K_{n\geq1} & = \pi\R\kuu + \frac{K_{bend}K_{tors}}{K_{bend} + K_{tors}} - \pi n^2 \Tb
\label{eq:Kn}
\end{align}
\end{subequations}

where
\begin{align*}
K_{bend} &= \frac{\pi \EIl}{R^3} (n^2-1)^2\\
K_{tors} &= \frac{\pi (\GJ + \EIw n^2/R^2)}n^2(n^2-1)^2
\end{align*}

This model corresponds to the diagram shown in Figure \ref{fig:lat_mode_springs}. The $n=0$ mode corresponds to a rigid-body displacement of the rim along the $\eo$ direction. The $n=1$ mode corresponds to a rigid-body rotation of the rim about an axis in the plane of the wheel passing through the hub. The $n=2$ mode corresponds to the well-known ``taco'' that a wheel typically takes on when buckled. Each mode stiffness is composed of 3 springs in parallel: (1) a spring representing the stiffness of the spokes, (2) a spring representing the stiffness of the rim, itself composed of two springs in series for the effective bending and torsional stiffnesses, and (3) a destabilizing spring representing the tendency of the wheel to buckle under excessive tension. Because the rim stiffness increases dramatically with $n$, a satisfactory approximation can usually be obtained by only including 3 or 4 terms in the series \eqref{eq:Klat_series}.

\begin{figure}[h]
\centering
\includesvg{\rootdir/figs/stress_analysis/}{lat_mode_springs}
\caption{Equivalent spring model. \textbf{(a)} Mode stiffnesses represented as springs. \textbf{(b)} Illustrated mode shapes for the first three modes.}
\label{fig:lat_mode_springs}
\end{figure}

A few observations can be gleaned from the equivalent spring model: First, the bending and torsion stiffness of the rim combine like springs connected in series, and is therefore dominated by the smaller spring constant. If the torsional stiffness $\GJ$ is significantly less than the lateral bending stiffness $\EIl$ (e.g. in a single-wall rim or a wide ``fat-bike'' rim), the wheel stiffness will be dictated by the torsional stiffness and the lateral stiffness will have an insignificant effect. Next, the equivalent springs model makes clear the dominant role that the spoke system plays in lateral stiffness. The first two modes represent rigid-body motions of the rim and only involve the spoke stiffness. If the rim is made infinitely stiff (compared to the spokes), a rigorous upper-bound for the spoke stiffness is given by
\begin{equation}
\label{eq:Klat_stiff_rim}
\max{\gls{Klat}} = \frac{2}{3} \pi \R \kuu
\end{equation}

Bicycle wheels are often marketed on their stiffness, which is prized for its presumed benefits to performance and durability. However, as modern rims have become stiffer, wheel manufacturers have followed a trend towards fewer spokes as a way to save weight, reduce drag, and cut costs \cite{Brown2011}. How might a modern wheel compare with a typical road wheel from the 1970's?

\begin{table}[h]
\caption{Example wheel properties.\label{tab:wheels}}
\begin{tabular}{@{}lcccc}
\hline\noalign{\smallskip}
\bf{Wheel} & $\GJ$ & $\EIl$ & $\pi\R\kuu$ [N/mm] & \gls{Klat} [N/mm]\\
\noalign{\smallskip}\hline\noalign{\smallskip}
Modern & 94.0 & 206 & 304 & 121\\
Vintage & 16.8 & 158 & 461 & 147\\
\noalign{\smallskip}\hline
\end{tabular}
\end{table}

As an example calculation, let us consider two hypothetical front wheels with the same hub width (\SI{60}{mm}): (a) a modern racing bicycle wheel constructed from the Alex ALX295 rim (a modern deep double-wall rim) with 24 lightweight 1.7/2.0 mm spokes, and (b) a vintage road bicycle wheel constructed from the Alex X404-27'' rim (a shallow single-wall rim) with 36 \SI{2}{mm} spokes. The rim properties and spoke system stiffnesses are given in Table \ref{tab:wheels}. Equation \eqref{eq:Klat_series} gives a theoretical lateral stiffness of \SI{121}{N/mm} for wheel (a) and a stiffness of \SI{147}{N/mm} for wheel (b). The stiffer (and heavier) spokes in wheel (b) make up for its relatively flexible rim.

\subsection{Lateral stiffness vs. spoke tension}

A common misconception among cyclists holds that increasing spoke tension results in a stiffer wheel. This theory likely stems from an intuitive association between ``tight'' and ``stiff,'' and possibly from the fact that a spoke, when plucked, produces a pitch proportional to its tension \cite{Allen1997}. However, the conventional wisdom among wheelbuilders holds that spoke tension has no effect on stiffness \emph{provided that spokes do not go slack under external loads.} \cite{Rinard,Kopecky2013,Hjertberg2014}. Although it's true that the \emph{axial stiffness} of a single spoke does not change with tension, the effective stiffness in a direction not parallel to the spoke axis does depend on tension.

Spoke tension appears in the equilibrium equations in two ways: (1) the ``tension-stiffness'' component of the spoke stiffess---the phenomenon responsible for the stiffness of guitar strings---and (2) the tendency of the rim to buckle under the compressive load induced by the spoke tension. These two effects are in opposition and roughly balance out at moderate spoke tensions. However, at sufficiently high tension, the negative stiffness term $-\pi n^2 \Tb$ begins to dominate and cause the lateral stiffness to decrease. (At a critical tension, the lateral stiffness vanishes entirely and the wheel buckles into a non-planar shape. This phenomenon will be explored in Chapter \ref{chap:tension_buckling}).

\begin{figure}[h]
\centering
\includesvg{\rootdir/figs/stress_analysis/}{Klat_tension}
\caption{Influence of spoke tension on lateral stiffness. \textbf{(a)} Comparison of lateral stiffness vs. spoke tension calculated from non-linear finite-element simulations (ABAQUS Standard), Equation \eqref{eq:Klat_series}, and my own experiments. \textbf{(b)} Experimental results by Damon Rinard \cite{Rinard}. \textbf{(c)} Deformed shape of the rim under a unit load under increasing spoke tensions. $T=\,\,$0, 500, 1000, and 1500 N. \textbf{(d)-(e)} Experimental setup. \textbf{(f)} Load-displacement curves for the six tensions.}
\label{fig:Klat_tension}
\end{figure}

\todo{Obtain permission to reprint Damon Rinard's results.}
Damon Rinard measured the lateral stiffness of a wheel at different tensions and concluded that tension has no significant impact on stiffness, unless the tension is so low that the spokes on the loaded side of the rim buckle \cite{Rinard}. Rinard measured stiffness by hanging a \SI{25}{lb} weight and recording the deflection with a dial indicator. He did not measure or report the tensions for each configuration, but rather reported the number of quarter turns of the spoke nipple below ``full tension.'' Although the relationship between turns and tension depends on wheel parameters that he did not report, it can be assumed that the relation is linear.

Rinard's results are converted and re-plotted in Figure \ref{fig:Klat_tension} (b) (his original data was reported as deflections rather than stiffness). At low tension, his measured stiffness drops by about \SI{50}{\percent} due to buckling of spokes under his relatively large test load (\SI{25}{lb}). At higher tensions, his measured stiffness decreases measurably. It is difficult to increase the spoke tension in a typical wheel to much more than \SI{50}{\percent} of the buckling tension due to increasing friction at the spoke nipple and the magnification of geometric imperfections in the rim which make it difficult or impossible. Thus the most likely explaination for the conventional wisdom that spoke tension does not affect stiffness is that no one has tested rims at sufficiently high tensions to see more than a modest effect.

\todo{Conduct my own experiments on lateral stiffness}
I measured the lateral stiffness of a wheel at six different spoke tensions. The hub was clamped into a custom fixture mounted to a rigid optical breadboard. A normal bicycle wheel could have significant flexibility in the ball bearings, so I used a custom research hub consisting of steel flanges screwed onto a 3/4-inch threaded axle\footnote{The research hub was designed by a student capstone design team advised by Jim Papadopoulos \cite{Alim2016}.}. The load was applied by hanging weights from the rim. The displacement was measured using a dial indicator positioned on a pivoting bar at a point \SI{10}{cm} away from the load point. The bar magnified the displacement and enabled the use of smaller loads. The load-displacement curves are shown ir Figure \ref{fig:Klat_tension} (f).

The experimental results are shown in Figure \ref{fig:Klat_tension} (a), compared against Equation \eqref{eq:Klat_series} and ABAQUS simulations. The stiffness recorded at \SI{1067}{N} spoke tension (\SI{76.5}{N/mm}) is about \SI{20}{\percent} lower than the maximum recorded stiffness (\SI{95.5}{N/mm}). At higher tensions, the spoke nipples are difficult to turn and the rim begins to take on a distorted shape. To my knowledge, these are the first reported experiments on lateral stiffness which are quantitatively compared with theory and simulations.


\subsection{Wheels with offset spokes: ``fat bikes''}
\todo{Wheels with offset spokes}
\inprogress

% where $U_n$ and $V_n$ are the strain energy and the virtual work due to rim compression for the $n$\textsuperscript{th} mode, and $P$ is the applied load. The strain energy and virtual work decompose additively due to the orthogonality of the Fourier basis functions around the unit circle. The mode coefficients $u_n, \p_n$ are those which bring Equation \eqref{eq:TotPot_RR} to a global minimum. Since \eqref{eq:TotPot_RR} is quadratic in $u_n, \p_n$ it is sufficient to require
% \begin{subequations}
% \label{eq:RR_min}
% \begin{align}
% \frac{\partial \Pi}{\partial u_n} &= 0\\
% \frac{\partial \Pi}{\partial \p_n} &= 0
% \end{align}
% \end{subequations}

% \begin{equation}
% \begin{cases}
% \begin{matrix}
% 2\pi R \kuu u_0 + 2\pi R \kup \p_0=P\\
% \kup u_0 + \left(\kpp + \frac{\EIl}{R}\right)\p_0=0
% \end{matrix}, & (n=0)\\
% \begin{matrix}
% \left(\frac{\EIl\pi}{R^3}n^4 + \frac{\tilde{\GJ}\pi}{R^3}n^2 + \pi R\kuu -\frac{n^2\bar{T}}{2}\right)u_n
% 	-\left(\frac{\EIl\pi}{R^2}n^2 + \frac{\tilde{\GJ}\pi}{R^2}n^2-\pi R\kup\right)\p_n =P\\
% \left(\frac{\EIl\pi}{R^2}n^2 + \frac{\tilde{\GJ}\pi}{R^2}n^2 - \pi R\kup \right)u_n
%  - \left(\frac{\EIl\pi}{R} + \frac{\tilde{\GJ}\pi}{R}n^2 + \pi R\kpp\right)\p_n =0
% \end{matrix}, & (n\geq 1)
% \end{cases}
% \end{equation}


\end{document}