\providecommand{\rootdir}{../..}
\documentclass[\rootdir/thesis.tex]{subfiles}

\begin{document}

We can easily obtain approximate solutions to \eqref{eq:EulerLagrange} of arbitrary accuracy while preserving the full details of coupling between $u,v,w,\phi$ by the Rayleigh-Ritz method. Rather than requiring that the first variation of the total potential \eqref{eq:TotPot} be precisely zero, we construct convenient approximations to the deformation variables $u,v,w,\phi$ and minimize the total potential.

We approximate the deformation variables with a finite Fourier series:
\begin{align}
\label{eq:FourierDef}
u &= u_0 + \sum_{n=1}{N} u_n^c \cos{n\theta} + u_n^s \sin{n\theta}\\
v &= v_0 + \sum_{n=1}{N} v_n^c \cos{n\theta} + v_n^s \sin{n\theta}\\
w &= w_0 + \sum_{n=1}{N} w_n^c \cos{n\theta} + w_n^s \sin{n\theta}\\
\phi &= \phi_0 + \sum_{n=1}{N} \phi_n^c \cos{n\theta} + \phi_n^s \sin{n\theta}
\end{align}

Increasing the maximum mode number $N$ results in higher accuracy. The deformation mode coefficients are collected into a single vector of length $4+8N$:
\begin{equation}
\mathbf{d} = [u_0,v_0,w_0,\phi_0,u_1^c,u_1^s,v_1^c,v_1^s,w_1^c,w_1^s,\phi_1^c,\phi_1^s,u_2^c,u_2^s,\dots]^T
\end{equation}

The displacement vector $[u,v,w,\phi]$ at a point $\theta$ is given by
\begin{equation}
\mathbf{u} = \mathbf{B}(\theta)\mathbf{d}
\end{equation}

where
\begin{equation}
\setcounter{MaxMatrixCols}{16}
\mathbf{B}(\theta) =
\begin{bmatrix}
1 & 0 & 0 & 0 & c\theta & s\theta & 0 & 0 & 0 & 0 & 0 & 0 & c2\theta & s2\theta & \dots\\
0 & 1 & 0 & 0 & 0 & 0 & c\theta & s\theta & 0 & 0 & 0 & 0 & 0 & 0 & \dots\\
0 & 0 & 1 & 0 & 0 & 0 & 0 & 0 & c\theta & s\theta & 0 & 0 & 0 & 0 & \dots\\
0 & 0 & 0 & 1 & 0 & 0 & 0 & 0 & 0 & 0 & c\theta & s\theta & 0 & 0 & \dots
\end{bmatrix}
\end{equation}

where $c\theta=\cos{\theta}, s\theta=\sin{\theta}$, etc.

Inserting the series approximations \eqref{eq:FourierDef} into \eqref{eq:U_rim} and integrating yields a quadratic form for the strain energy in the rim:
\begin{equation}
U_{rim} = \frac{1}{2} \mathbf{d}^T \mathbf{K}_{rim} \mathbf{d}
\end{equation}

where $\mathbf{K}_{rim}$ is the rim mode stiffness matrix. Since the Fourier basis functions are orthogonal on the unit circle, $\mathbf{K}_{rim}$ has the block diagonal structure:
\begin{equation}
\label{eq:K_rim_mode}
\mathbf{K}_{rim} =
\begin{bmatrix}
\mathbf{K}_0^{rim} & & &\\
& \mathbf{K}_1^{rim} & &\\
& & \ddots &\\
& & & \mathbf{K}_N^{rim}
\end{bmatrix}
\end{equation}

The zero-mode matrix is
\begin{equation}
\mathbf{K}_0^{rim} =
\begin{bmatrix}
0 & 0 & 0 & 0\\
0 & 2\pi \frac{EA}{R} & 0 & 0\\
0 & 0 & 0 & 0\\
0 & 0 & 0 & 2\pi \frac{EI_2}{R}
\end{bmatrix}
\end{equation}

The subsequent mode matrices take the form
\begin{equation}
\mathbf{K}_{n\geq 1}^{rim} =
\begin{bmatrix}
k_{uu} & 0 & 0 & 0 & 0 & 0 & -k_{u\phi} & 0\\
0 & k_{uu} & 0 & 0 & 0 & 0 & 0 & -k_{u\phi}\\
0 & 0 & k_{vv} & 0 & 0 & -k_{vw} & 0 & 0\\
0 & 0 & 0 & k_{vv} & k_{vw} & 0 & 0 & 0\\
0 & 0 & -k_{vw} & 0 & 0 & k_{ww} & 0 & 0\\
-k_{u\phi} & 0 & 0 & 0 & 0 & 0 & k_{\phi\phi} & 0\\
0 & -k_{u\phi} & 0 & 0 & 0 & 0 & 0 & k_{\phi\phi}
\end{bmatrix}
\end{equation}

\todo{Add compression terms?}
\begin{align*}
k_{uu} &= \frac{\pi EI_2}{R^3}n^4 + \frac{\pi EI_w}{R^5}n^4 + \frac{\pi GJ}{R^3}n^2 - ...\\
k_{vv} &= \frac{\pi EI_1}{R^3}n^4 + \frac{\pi EA}{R}\left(1 + \frac{y_0}{R}n^2 \right)^2\\
k_{ww} &= \frac{\pi EI_1}{R^3}n^2 + \frac{\pi EA n^2}{R}\left(1 + \frac{y_0}{R} \right)^2\\
k_{vw} &= \frac{\pi EI_1}{R^3}n^3 + \frac{\pi EA n}{R}\left(1 + \frac{y_0}{R}(1+n^2) + \frac{y_0^2}{R^2}n^2\right)\\
k_{\phi\phi} &= \frac{\pi EI_2}{R} + \frac{\pi EI_w}{R^3}n^4 + \frac{\pi GJ}{R}n^2 + ...\\
k_{u\phi} &= \frac{\pi EI_2}{R^2}n^2 + \frac{\pi EI_w}{R^4}n^4 + \frac{\pi GJ}{R^2}n^2 + ...
\end{align*}

\end{document}
