%!TEX root = ../../thesis.tex
\providecommand{\rootdir}{../..}
\documentclass[\rootdir/thesis.tex]{subfiles}

\begin{document}

One can easily obtain approximate solutions to \eqref{eq:TotPot} of arbitrary accuracy while preserving the full details of coupling between $u,v,w,\p$ by the Rayleigh-Ritz method. Rather than requiring that the first variation of the total potential \eqref{eq:TotPot} be precisely zero, we construct convenient approximations to the deformation variables $u,v,w,\p$ and minimize the resulting approximate total potential function.

We approximate the deformation variables with a finite Fourier series:
\begin{align}
\label{eq:mm_FourierDef}
u &= u_0 + \sum_{n=1}^{\gls{N}} u_n^c \cos{n\gls{ang}} + u_n^s \sin{n\gls{ang}}\\
v &= v_0 + \sum_{n=1}^{\gls{N}} v_n^c \cos{n\gls{ang}} + v_n^s \sin{n\gls{ang}}\\
w &= w_0 + \sum_{n=1}^{\gls{N}} w_n^c \cos{n\gls{ang}} + w_n^s \sin{n\gls{ang}}\\
\p &= \p_0 + \sum_{n=1}^{\gls{N}} \p_n^c \cos{n\gls{ang}} + \p_n^s \sin{n\gls{ang}}
\end{align}

Increasing the maximum mode number \gls{N} results in higher accuracy. The deformation mode coefficients are collected into a single vector of length $4+8\gls{N}$:
\begin{equation}
\gls{dm} = [u_0,v_0,w_0,\p_0,u_1^c,u_1^s,v_1^c,v_1^s,w_1^c,w_1^s,\p_1^c,\p_1^s,u_2^c,u_2^s,\dots]^T
\end{equation}

The augmented displacement vector $\gls{d}=[u,v,w,\p]^T$ at a point $\gls{ang}$ is given by
\begin{equation}
\label{eq:u_Bd}
\gls{d} = \gls{B}(\gls{ang})\gls{dm}
\end{equation}

where
\begin{equation}
\setcounter{MaxMatrixCols}{16}
\gls{B}(\gls{ang}) =
\begin{bmatrix}
1 & 0 & 0 & 0 & c\gls{ang} & s\gls{ang} & 0 & 0 & 0 & 0 & 0 & 0 & c2\gls{ang} & s2\gls{ang} & \dots\\
0 & 1 & 0 & 0 & 0 & 0 & c\gls{ang} & s\gls{ang} & 0 & 0 & 0 & 0 & 0 & 0 & \dots\\
0 & 0 & 1 & 0 & 0 & 0 & 0 & 0 & c\gls{ang} & s\gls{ang} & 0 & 0 & 0 & 0 & \dots\\
0 & 0 & 0 & 1 & 0 & 0 & 0 & 0 & 0 & 0 & c\gls{ang} & s\gls{ang} & 0 & 0 & \dots
\end{bmatrix}
\end{equation}

where $c\gls{ang}=\cos{\gls{ang}}, s\gls{ang}=\sin{\gls{ang}}$, etc. Inserting the series approximations \eqref{eq:mm_FourierDef} into \eqref{eq:U_rim} and \eqref{eq:V_rim} and integrating yields a quadratic form for the strain energy and virtual work of internal forces in the rim:
\begin{equation}
\label{eq:mm_U_rim}
\Pi_{rim} = \frac{1}{2} \gls{dm}^T \gls{KmRim} \gls{dm}
\end{equation}

where $\gls{KmRim}$ is the rim mode stiffness matrix. Since the Fourier basis functions are orthogonal on the unit circle, $\gls{KmRim}$ has the block diagonal structure:
\begin{equation}
\label{eq:mm_K_rim}
\gls{KmRim} =
\begin{bmatrix}
\mathbf{K}_0^{rim} & & &\\
& \mathbf{K}_1^{rim} & &\\
& & \ddots &\\
& & & \mathbf{K}_{\gls{N}}^{rim}
\end{bmatrix}
\end{equation}

The zero-mode matrix is
\begin{equation}
\mathbf{K}_0^{rim} =
\begin{bmatrix}
0 & 0 & 0 & 0\\
0 & 2\pi \frac{\EA}{\R} & 0 & 0\\
0 & 0 & 0 & 0\\
0 & 0 & 0 & 2\pi \frac{\EIl}{\R} + 2\pi \R\Tb\gls{yo}
\end{bmatrix}
\end{equation}

The subsequent mode matrices take the form
\begin{equation}
\mathbf{K}_{n\geq 1}^{rim} =
\begin{bmatrix}
k_{uu} & 0 & 0 & 0 & 0 & 0 & -k_{u\p} & 0\\
0 & k_{uu} & 0 & 0 & 0 & 0 & 0 & -k_{u\p}\\
0 & 0 & k_{vv} & 0 & 0 & -k_{vw} & 0 & 0\\
0 & 0 & 0 & k_{vv} & k_{vw} & 0 & 0 & 0\\
0 & 0 & -k_{vw} & 0 & 0 & k_{ww} & 0 & 0\\
-k_{u\p} & 0 & 0 & 0 & 0 & 0 & k_{\p\p} & 0\\
0 & -k_{u\p} & 0 & 0 & 0 & 0 & 0 & k_{\p\p}
\end{bmatrix}
\end{equation}

\begin{align*}
k_{uu} &= \frac{\pi \EIl}{\R^3}n^4 + \frac{\pi \EIw}{\R^5}n^4 + \frac{\pi \GJ}{\R^3}n^2 - \pi n^2 \Tb\left(1 + \frac{r_0^2}{\R^2}\right)\\
k_{vv} &= \frac{\pi \EIr}{\R^3}n^4 + \frac{\pi \EA}{\R}\left(1 + \frac{y_0}{\R}n^2 \right)^2\\
k_{ww} &= \frac{\pi \EIr}{\R^3}n^2 + \frac{\pi \EA n^2}{\R}\left(1 + \frac{\gls{yo}}{\R} \right)^2\\
k_{vw} &= \frac{\pi \EIr}{\R^3}n^3 + \frac{\pi \EA n}{\R}\left(1 + \frac{\gls{yo}}{\R}(1+n^2) + \frac{y_0^2}{R^2}n^2\right)\\
k_{\p\p} &= \frac{\pi \EIl}{\R} + \frac{\pi \EIw}{\R^3}n^4 + \frac{\pi \GJ}{\R}n^2 + \pi \R^2\Tb\left(\frac{\gls{yo}}{\R} - \frac{r_0^2}{\R^2}n^2\right)\\
k_{u\p} &= \frac{\pi \EIl}{\R^2}n^2 + \frac{\pi \EIw}{\R^4}n^4 + \frac{\pi \GJ}{\R^2}n^2 + \pi n^2 \R \Tb\left(\frac{\gls{yo}}{\R} - \frac{r_0^2}{\R^2}n^2\right)
\end{align*}

Inserting \eqref{eq:u_Bd} into \eqref{eq:U_spokes} yields the strain energy stored in the spokes:
\begin{align}
\label{eq:mm_U_spokes}
\begin{split}
U_{spokes} &= \frac{1}{2} \sum_{i=1}^{\gls{ns}} \gls{d}_i^T \gls{k}_i \gls{d}_i\\
&= \frac{1}{2} \sum_{i=1}^{\gls{ns}} \gls{dm}^T \left(\gls{B}_i^T \gls{k}_i \gls{B}_i \right)\gls{dm}\\
&= \frac{1}{2} \gls{dm}^T \left(\sum_{i=1}^{\gls{ns}} \gls{B}_i^T \gls{k}_i \gls{B}_i \right) \gls{dm}\\
&= \frac{1}{2} \gls{dm}^T \gls{KmSpk} \gls{dm}
\end{split}
\end{align}

Due the discrete nature of the spokes, the spoke stiffness matrix $\gls{KmSpk}$ has non-zero elements outside of the block diagonal shown in Eqn. \eqref{eq:mm_K_rim}. If, on the other hand, the smeared-spokes approximation to the strain energy \eqref{eq:U_spokes} is used, then the modes decouple and the strain energy is given by
\begin{equation}
\label{eq:U_spokes_smeared}
\bar{U}_{spokes} = \frac{1}{2} \gls{dm}^T \gls{KmSpkSm} \gls{dm}
\end{equation}

where the zero-mode block (upper-left 4x4 matrix) is equal to $2\pi \R\gls{kbar}$. The remaining 8x8 blocks are given by the relation $\gls{KmSpkSm}(2i-1, 2j-1) = \gls{KmSpkSm}(2i, 2j) = \pi \R \bar{k}_{ij}$.

If the external loads are given as a series of point loads (and couples), the virtual work of external loads is obtained in a straightforward manner:
\begin{align}
\label{eq:mm_F_ext}
\begin{split}
V_{ext} &= \sum_{i=1}^{n_f} \mathbf{f}_{ext}^i \cdot \gls{d}(\gls{ang}_i) \\
&= \sum_{i=1}^{n_f} \mathbf{f}_{ext}^i \cdot \gls{B}_i \gls{dm}\\
&= \left(\sum_{i=1}^{n_f} \mathbf{f}_{ext}^i \gls{B}_i \right) \gls{dm}\\
&= \gls{Fext}\gls{dm}
\end{split}
\end{align}

Combining Eqns. \eqref{eq:mm_U_rim}, \eqref{eq:mm_U_spokes}, and \eqref{eq:mm_F_ext}, the total potential energy is
\begin{equation}
\label{eq:mm_TotPot}
\Pi = \frac{1}{2} \gls{dm}^T \left( \gls{KmRim} + \gls{KmSpk} \right) \gls{dm} - \gls{Fext}\gls{dm}
\end{equation}

Minimizing the total potential energy \eqref{eq:mm_TotPot} with respect to the mode coefficients $\gls{dm}$ yields the modal Rayleigh-Ritz equations:
\begin{equation}
\label{eq:mm_Kd_f}
\left( \gls{KmRim} + \gls{KmSpk} \right) \gls{dm} = \gls{Fext}
\end{equation}

Equation \eqref{eq:mm_Kd_f} suggests an analogy with the finite-element method, in which the displacement field is approximated with appropriately constructed shape functions which interpolate the displacements at discrete points throughout the body. By contrast, the mode-matrix method described here approximates the displacement field with a finite set of functions chosen such that the strain energy is approximately additively decomposed. If the smeared-spokes approximation \eqref{eq:U_spokes_smeared} is used, the stiffness matrix is guaranteed to have a sparse, block-diagonal form, while still retaining possible coupling (through the spoke geometry) between in-plane and out-of-plane deformations.

\end{document}
