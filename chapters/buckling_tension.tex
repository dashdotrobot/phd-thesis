%!TEX root = ../thesis.tex
\providecommand{\rootdir}{..}
\documentclass[\rootdir/thesis.tex]{subfiles}

\begin{document}

If the spokes are tensioned beyond a critical value, the circular shape becomes unstable and the rim will buckle into a non-planar shape. The post-buckling configuration is generally stable and the original shape of the wheel can be recovered by uniformly reducing the spoke tensions. Despite its implications for wheel strength, the buckling problem has never received a rigorous treatment. Jobst Brandt alludes to buckling in his practical manual for wheelbuilders\cite{Brandt1993}:

\begin{quote}
\emph{``If the wheel becomes untrue in two large waves during stress relieving, the maximum, safe tension has been exceeded. Approach this tension carefully to avoid major rim distortions. When the wheel loses alignment from stress relieving, loosen all spokes a half turn before retruing the wheel.''}
\end{quote}

The ``stress relieving'' referred to here is the practice of laterally loading the rim (or the spokes directly) so that the spokes temporarily increase their tension, which is presumed also to cause plastic flow in the spoke elbows so as to lower residual stresses in the service condition, and hence reduces fatigue.

Flexural-torsional buckling of rings can be treated as a special case of buckling of arches, where the included angle is allowed to go to $2\pi$ with appropriately periodic boundary conditions. Timoshenko and Gere\cite{Timoshenko1961} gave a formula for the critical load for a ring with doubly-symmetric cross-section subjected to a line load directed towards the ring's center The theory of flexural-torsional buckling of monosymmetric arches (bicycle rim sections generally have only one plane of symmetry) was broadly formalized by Trahair and Papangelis\cite{Trahair1987} using the virtual work approach to derive the equilibrium and stability equations. Their theory has been extended to treat arches with continuous \cite{Pi2002} or discrete \cite{Bradford2002} elastic supports and elastic end restraints\cite{Guo2014}.

The problem of the prestressed bicycle wheel is unique for a number of reasons. First, the load for buckling due to spoke over-tension is internal to the structure. Second, the spokes act both as elastic restraints resisting buckling and as prestressing elements causing buckling. Third, the lateral, radial, tangential, and torsional restraining actions of the spokes are commonly coupled – e.g. lateral deflection at a spoke may produce a mix of those reactions on the rim section. These considerations extend to other structural systems. Large observation wheels such as the London Eye\cite{} and the Singapore Flyer\cite{} resemble bicycle wheels and achieve lateral stability due to the bracing angle of prestressed cables, and must be designed against flexural-torsional buckling. At the biological scale, the cellular fragmentation process which leads to platelet formation may also be driven by flexural-torsional buckling instability of a ring of bundled actin fibers loaded by an elastic membrane which both promotes buckling and provides elastic restraint\cite{}.


\section{Elastic stability criterion}
It was previous shown (Section \ref{sec:ModeMatrix}) that the deformed shape of the wheel under external loads could be found by solving the mode matrix equation \eqref{eq:mm_Kd_f}. One form of the elastic stability criterion states that for any admissable deformation, the second variation of the total potential \eqref{eq:mm_TotPot} must be positive. Since any rim deformation must be periodic and continuous, the modes in \eqref{eq:mm_FourierDef} form a set of kinematically admissable deformations. Therefore the buckling tension can be defined as the tension at which \eqref{eq:mm_Kd_f} admits nontrivial solutions when $\mathbf{F}_{ext}=\mathbf{0}.$\todo{Check wording of precise stability criterion}

The spoke stiffness comprises a term proportional to the elastic stiffness, $EA$, of the spokes and a term proportional to the spoke tension, $T$. Additionally, the rim stiffness comprises a term depending on the bending, torsion, and warping constants, and a term proportional to the net radial spoke tension per unit length, $\bar{T}$. In order to make the dependence on $\bar{T}$ explicit, we write the buckling criterion as follows:
\begin{equation}
\label{eq:Tc_crit}
\det{\left[\mathbf{K}_{rim}^{el} + \mathbf{K}_{spokes}^{el} + \bar{T}(\mathbf{K}_{spokes}^{tens} - \mathbf{K}_{rim}^{tens})\right]} = 0
\end{equation}

If the complete details of discrete spokes and possible radial/lateral coupling are retained, a numerical solution to \eqref{eq:Tc_crit} can be obtained by including enough modes such that the relevant length scales (e.g. the distance between spokes) are correctly approximated, and then either numerically solving the characteristic polynomial of \eqref{eq:Tc_crit}, or iteratively increasing $\bar{T}$ until the determinant is minimized. Under more restrictive assumptions, analytical solutions are possible, as shown in Table \ref{tab:BucklingSolutions}.

\begin{table}[h]
\caption{Simplifying assumptions for elastic buckling criterion.}
\label{tab:BucklingSolutions}
\begin{tabularx}{\textwidth}{p{1.2cm}Xccc}
% \hline\noalign{\smallskip}
\toprule
\centering Smeared spokes & \centering Symmetry & Rim & Complexity & Analytical solution\\
\midrule
\centering No  & \centering None & Monosymmetric$^{\rm b}$ & $\det$ of (4+8N)x(4+8N) matrix & None\\
\centering Yes & \centering None & Monosymmetric & $\det$ of 4x4 matrix & Impractical\\
\centering Yes & \centering Left-right$^{\rm a}$& Monosymmetric& Quadratic equation for $\bar{T}_c$ & Solution of \eqref{eq:Tc_quad}\\
\centering Yes & \centering Left-right& Bi-symmetric$^{\rm c}$ & Linear equation for $\bar{T}_c$ & \eqref{eq:Tcn_lin}\\
\centering Yes & \centering Left-right, no spoke offset & Bi-symmetric & Linear equation for $\bar{T}_c$ & \eqref{eq:tcn_lin_nophi}\\
\bottomrule
\multicolumn{5}{p{\textwidth}}{
$^{\rm a}$The distance to the hub flange and spoke arrangement are identical on the left and right sides, which implies no radial/lateral coupling.
}\\
\multicolumn{5}{p{\textwidth}}{
$^{\rm b}$Monosymmetry implies $y_0\neq 0$
}\\
\multicolumn{5}{p{\textwidth}}{
$^{\rm c}$Bi-symmetry implies $y=0$, i.e. the shear center and centroid coincide.
}\\
\bottomrule
\end{tabularx}
\end{table}

\subsection{Smeared spokes approximation}

If the smeared-spokes approximation is used, the mode stiffness matrix $\mathbf{K}_{rim} + \mathbf{\bar{K}}_{spokes}$ has a block diagonal form, with each block corresponding to a different mode $n$ in \eqref{eq:mm_FourierDef}. The first two modes $n=0$ and $n=1$ are rigid-body motions and do not admit buckling. The buckling criterion then becomes

\begin{equation}
\label{eq:Tc_crit_smeared}
\det{\left[\mathbf{K}_{rim,n}^{el} + \mathbf{\bar{K}}_{spokes,n}^{el} + \bar{T}(\mathbf{\bar{K}}_{spokes,n}^{tens} - \mathbf{K}_{rim,n}^{tens})\right]} = 0
\end{equation}

In its most general form, Equation \eqref{eq:Tc_crit} results in a quadratic equation for $\bar{T}$ which can be solved to find the critical tension $\bar{T}_{c,n}$ for a given mode $n$. The critical buckling tension $\bar{T}_c$ for the wheel is the minimum $\bar{T}_{c,n}$ over all integer modes.

The elastic stiffness $EA$ is much less than the spoke tension. Therefore, the tension components of the spoke stiffness matrix $\bar{k}_{ij}^{tens}$ can generally be neglected compared with $\bar{k}_{ij}^{el}$, except for $\kuu$ due to the small lateral projection of the spoke vector. In this section we will neglect the tension component of all stiffness parameters except for $\kuu$.

\subsubsection{No radial/lateral coupling}
If the radial/lateral spoke coupling terms are neglected ($\kuv=\kuw=\kvp=\kwp=0$), the buckling criterion \eqref{eq:Tc_crit} reduces to
\begin{multline}
\label{eq:Tc_quad}
\left(R^2\kuu^{tens}y_0 - n^2Ry_0 \right)\bar{T}^2\\
-\left[\frac{EI_2}{R}\left(n^2 - 3n^4 \frac{y_0}{R} + R\kuu^{tens}\right)
       +\frac{\widetilde{GJ}}{R}\left(n^4 - \frac{y_0}{R}(2n^4 - n^2) - R\kuu^{tens}n^2\right) \right.\\
       \left. +R\kpp (n^2 - R\kuu^{tens}) + 2Ry_0\kup n^2 - R^2\kuu y_0 \vphantom{\frac12} \right]\bar{T}\\
+\left[\frac{EI_2 \widetilde{GJ} n^2}{R^4}(n^2-1)^2
       +EI_2\left(\kuu + 2\frac{\kup}{R}n^2 + \frac{\kpp}{R^2}n^4\right) \right.\\
       \left. +\widetilde{GJ}\left(\kuu n^2 + 2\frac{\kup}{R}n^2 + \frac{\kpp}{R^2}n^2\right)
       +R^2(\kuu\kpp - \kup^2)\right] = 0
\end{multline}

where $\widetilde{GJ} = GJ + EI_wn^2/R^2$ is the effective torsional stiffness.

\subsubsection{No radial/lateral coupling, bi-symmetric rim}
The quadratic term in \eqref{eq:Tc_quad} is proportional to $y_0$. If the rim cross-section is assumed to be symmetric across both the $\eo$ and $\et$ axes, then the buckling criterion \eqref{eq:Tc_quad} reduces to a linear equation for $\bar{T}$. Using the non-dimensional paremeters defined in Section \ref{sec:Lateral}, the non-dimensionalized critical buckling tension is given by
\begin{equation}
\label{eq:Tcn_lin}
t_{c,n} = \left(\frac{1}{n^2 - R\kuu^{tens}}\right)
\left(\lambda_{uu}
      +\frac{(n^4 + \mu n^2)\lambda_{\phi\phi}
             +2n^2(\mu + 1)\lambda_{u\phi} - \lambda_{u\phi}^2
             +\mu n^2(n^2-1)^2}
        {1 + \mu n^2 + \lambda_{\phi\phi}}\right)
\end{equation}

The first term $(n^2 - R\kuu^{tens})^{-1}$ accounts for the fact the spokes (and therefore the direction of the applied tension at the rim) rotates under a buckling displacement. In classical buckling analysis, this accounts for the difference between buckling under dead loads (e.g. gravity) and directed loads (e.g. tensioned cables).

\inprogress

\subsubsection{No radial/lateral coupling, bi-symmetric rim, no spoke offset}
If the spokes are assumed to connect to the rim through the shear center, the lateral/torsional coupling terms vanish, i.e. $\lambda_{u\pi}=\lambda_{\phi\phi}=0$. Thus, the critical buckling tension is given by

\begin{equation}
\label{eq:tcn_lin_nophi}
t_{c,n} = \left(\frac{1}{n^2 - R\kuu^{tens}}\right)
\left(\lambda_{uu}
      +\frac{\mu n^2(n^2-1)^2}
        {1 + \mu n^2}\right)
\end{equation}

The equivalent spring model described in Section \ref{sec:equiv_springs} provides some insight into Equation \eqref{eq:tcn_lin_nophi}. The buckling tension can be found by solving Equation \eqref{eq:Kn} such that $K_{n\geq 1} = 0$.
\begin{equation}
\bar{T}_{c,n} = \frac{1}{\pi}\left(\frac{1}{n^2 - R\kuu^{tens}}\right) K_n(T=0)
\end{equation}

where $K_n(T=0)$ is given by Equation \eqref{eq:Kn}, evaluated at zero tension. As with the lateral mode stiffness, the spoke stiffness plays a significant role in determining the buckling tension \emph{for a given mode, $n$}. However, increasing the spoke stiffness significantly relative to the rim stiffness will favor higher modes $n > 2$ and lead to a discontinuous, non-linear increase in $\bar{T}_c$.

\section{Limiting solutions of \eqref{eq:tcn_lin_nophi} for special cases}

The critical tension $t_{c,n}$ for a given mode depends on $n$, $\mu$, and $\lambda_{uu}^{el}$. Therefore, the preferred buckling mode (minimum $t_{c,n}$ over all integers $n$) only depends on $\mu$ and $\lambda_{uu}^{el}$. Figure \ref{fig:tc_lambda_mu} (a) shows a map of the preferred buckling modes and their respective shapes. In the following sections we will consider several special cases of Equation \eqref{eq:tcn_lin_nophi}.

\begin{figure}[h]
\centering
\includesvg{\rootdir/figs/buckling_tension/}{tc_lambda-mu}
\caption{\textbf{(a)} Parameter map of preferred buckling mode. Normalized buckling tension $t_c$ is given by the color scale on the right. \textbf{(b)-(c)} Comparison between Equation \eqref{eq:tcn_lin_nophi} (black line), ABAQUS simulation results (blue stars), and power law approximations: \eqref{eq:Tc_hi_k} for (b) and \eqref{eq:tc_lomu} for (c). Parameters held constant are $\mu=0.38$ for (b) and $\lambda_{uu}^{el}=10$ for (c).}
\label{fig:tc_lambda_mu}
\end{figure}

\subsection{Low torsional stiffness}
As the ratio $GJ/EI_2$ tends towards zero, the mode number $n$ tends towards infinity. When the mode number is large, the buckling mode can then be found by approximating the discrete mode number $n$ with a continuous variable $\bar{n}$. We will adopt the following ansatz: (a) $\bar{n}^2 \gg 1$, and (b) $\mu\bar{n}^2 \ll 1$. Under these conditions, Equation \eqref{eq:tcn_lin_nophi} becomes
\begin{equation}
\label{eq:tcn_lomu}
t_{c,n} = \left(\mu \bar{n}^4 + \frac{\lambda_{uu}}{\bar{n}^2}\right)
\end{equation}

Minimizing \eqref{eq:tcn_lomu} with respect to $\bar{n}$ yields
\begin{equation}
\label{eq:n_lomu}
\bar{n}=\left(\frac{\lambda_{uu}}{2\mu}\right)^{1/6}
\end{equation}

Inserting \eqref{eq:n_lomu} into \eqref{eq:tcn_lomu}, we obtain the critical buckling tension, independent of mode number:
\begin{equation}
\label{eq:tc_lomu}
t_c = \left(\frac{1}{2^{2/3}} + 2^{1/3}\right)\mu^{1/3}\lambda_{uu}^{2/3}
\end{equation}

Substituting definitions for dimensionless quantities $t,\mu,\lambda_{uu}$, we obtain a critical buckling tension which is independent of lateral bending stiffness, $EI_2$:
\begin{equation}
\label{eq:Tc_lo_mu}
\bar{T}_c = 1.89 \left( \frac{GJ}{R} \right)^{1/3} \kuu^{2/3}
\end{equation}

Equation \eqref{eq:Tc_lo_mu} further illustrates the consequences of over-designing the bending stiffness while neglecting the torsional stiffness. Since the bending and torsional stiffnesses act like equivalent springs connected in series, the stiffness---and therefore the buckling resistance---will be dominated by the smaller of the two. Thus a very wide, but very shallow rim (e.g. a ``fat-bike'' rim) will be entirely dominated by its torsional stiffness.

\subsection{Moderate torsion stiffness, stiff spoke system}
Modern rims are often constructed from hollow extruded aluminum profiles. As a result they have high torsional resistance $GJ$ and negligible warping coefficient $EI_w$. Therefore \todo{$\tilde{\mu}$ may not be defined}$\tilde{\mu} = \mu$ and $GJ \sim EI_2$. If the spoke stiffness is much higher than the rim stiffness, i.e. $\lambda_{uu} \gg 1$, then we can make a similar argument as in the previous section. Now we will accept as an ansatz that $n^2 \gg 1$ and $\mu \sim 1$. Estimating the discrete variable $n$ with a continuous analog $\bar{n}$ Equation \eqref{eq:tcn_lin_nophi} becomes
\begin{equation}
\label{eq:tcn_hi_k}
t_{c,n} = \left(\bar{n}^2 + \frac{\lambda_{uu}}{\bar{n}^2}\right)
\end{equation}

Minimiizing \eqref{eq:tcn_hi_k} with respect to $\bar{n}$ gives the scaling law $\bar{n}=(\lambda_{uu})^{1/4}$. Inserting into \eqref{eq:tcn_hi_k} gives $t_c = 2(\lambda_{uu})^{1/2}$. In terms of the net radial tension per unit length, $\bar{T}$, this gives
\begin{equation}
\label{eq:Tc_hi_k}
\bar{T}_c = 2 \left(\frac{\kuu^{el}EI_2}{R^2}\right)^{1/2}
\end{equation}

Noting that the axial force in the rim is $N_r=R\bar{T}$, we recognize \eqref{eq:Tc_hi_k} as the critical buckling load for an infinite beam on an elastic foundation given by Hetenyi \cite{Hetenyi1946}. The rim buckles as if it were a straight beam since $\lambda_{uu}$ implies that the rim radius is large compared to the characteristic length of the beam on an elastic foundation.

\subsection{All spokes lie in the plane of the wheel}

If all the spokes were laced to the same flange of the hub, the spokes will all lie in the plane of the rim, implying $\kuu^{el}=0$. In this case, the rim will always buckle into the $n=2$ mode and Equation \eqref{eq:tcn_lin_nophi} simplifies to
\begin{equation}
t_c = \left(\frac{9\mu}{1 + 4\mu}\right)\left(\frac{4}{4 - R\kuu^{tens}}\right)
\end{equation}

If the spokes meet at the center of the rim (i.e. a hub of zero diameter), then $R=l_s$ and the critical reduced tension becomes
\begin{equation}
\label{eq:tc_Hencky}
t_c = \frac{12\mu}{1 + 4\mu}
\end{equation}

This is precisely the result obtained by Hencky for the critical distributed radial load for a thin ring\cite{Timoshenko1961}. Timoshenko obtained a slightly different result which differs from Equation \eqref{eq:tc_Hencky} by a factor of  $3/4$, by considering dead loads which do not change direction during buckling. This case illustrates the importance of considering the ``tension-stiffness'' when deriving the spoke forces. The change in direction of the spoke force after deflection produces a small restoring force on the rim which increases the buckling load by a factor of $4/(4 - R\kuu^{tens}) \approx 33\%$. Wheels with larger $R/l_s$, and thus a larger change in spoke angle for a given lateral deflection---for example high-flange hubs or hub motors---will receive an even larger benefit from the tension-stiffness term.

\section{Buckling of asymmetric wheels}
\inprogress

\section{Finite-element buckling calculations}
\inprogress

\section{Buckling of real wheels}

Several practical considerations make it impossible to reach the critical tension in a real wheel: (1) the buckling tension can be higher than the yield point of the spoke itself, (2) friction at the spoke nipple/rim interface becomes too great to overcome with a spoke wrench, and (3) the rim will go out of true laterally at around 50\% of the critical tension due to imperfections in the rim.

Despite these difficulties, the mode stiffness model described in Section \ref{sec:equiv_springs} provides a method of estimating the critical buckling tension by directly measuring the stiffness of a single mode ($n=2$) as a function of spoke tension. Under the assumptions described in \ref{sec:equiv_springs}, the stiffness of the $n$th mode is:
\begin{equation}
\label{eq:Kn_buckling}
K_n(T) = K_n^{rim} + \pi R (\bar{k}_{uu}^{el} + \bar{T} \bar{k}_{uu}^{tens}) - \pi n^2 \bar{T}
\end{equation}

The linear dependence of $\kuu$ on $\bar{T}$ is made explicit in Eqn. \eqref{eq:Kn_buckling} to show that $K_n$ also depends linearly on $\bar{T}$. By measuring the second mode stiffness (using a 4-point bending arrangement) at several spoke tensions and then extrapolating the data to $K_2 = 0$, the critical tension for the second mode can be obtained.

\subsection{Method}

I made measurements on five different wheels, all constructed with the same rim and spokes, but with varying distance between hub flanges. For each wheel, I made several measurements at increasing spoke tensions until the wheel started to buckle enough to create a significant difference between left and right spoke tensions at the anti-nodes of the rim.

\subsubsection{Wheel construction}

I built five different wheels using the same rim and spokes, but with varying distance between hub flanges. The rim was a Sun CR18 700C, a narrow road rim with a very shallow double-wall construction and spoke nipple eyelets drilled for 36 spokes. The spokes were Wheelsmith double-butted spokes with an end diameter of 2.0 mm and 1.70 mm along the swaged section. The hub was a custom-built adjustable-width hub designed by a Northeastern University capstone team advised by Jim Papadopoulos. Spoke tensions were determined using a WheelFanatyk analog tensiometer and readings were interpolated from the calibration table supplied with the instrument.

All wheel configurations were radially spoked with all inbound spokes. I made no attempt to keep the wheel laterally or radially true within any specification, but the rim was symmetrically dished and I ensured that no individual spoke deviated in tension by more than 10\% of the average tension. This eventually became impossible when the average spoke tension exceeded about 50\% of the critical tension and the wheel started to distort into a taco shape.

\subsubsection{Four-point bending test}

Each wheel was supported from the bottom at the 3-, and 9-o'clock positions, and loaded by hanging weights at the 12-o'clock position (Figure \ref{fig:4pt_bend_setup} (a)). The rim was oriented so that the sleeve joint and the valve hole fell midway between supports where the bending moment is minimized. Due to the symmetry of the boundary conditions, the work done against the odd modes ($n=1, 3, ...$) and even modes divisible by 4 ($n=4, 8, ...$) is identically zero. Since the hub is unsupported, the zero mode is also eliminated. Due to the rapid increase in mode stiffness with $n$, the $n=2$ mode accounts for about 95\% of the strain energy, while the other 5\% is spread across the remaining modes ($n=6, 10, ...$). Figure \ref{fig:4pt_bend_setup} (b) shows the measured lateral displacement around the rim for a point load located at $\theta=-\pi$. After subtracting the rigid body rotation, the measured displacement closely matches $\cos{2\theta}$. The second mode stiffness is related to the load-displacement slope of the four-point bend test by $K_2 = 16(P/u_l)$, where $P$ is the applied load and $u_l$ is the deflection of the load point.

\begin{figure}[h]
\centering
\includesvg{\rootdir/figs/buckling_tension/}{4pt_bend}
\caption{Four-point bend test. \textbf{(a)} Experimental setup showing load fixture, applied load, and measurement point. \textbf{(b)} Measured lateral displacement around the rim when loaded at $\theta=-\pi$. The symmetric part of the displacement is found by subtracting $(u_l/2)\cos{\theta}$ from the measured displacement.}
\label{fig:4pt_bend_setup}
\end{figure}

\subsection{Results}

As predicted by Equation \eqref{eq:Kn_buckling}, the measured mode stiffness decreases with applied tension (Figure \ref{fig:K2_results} (a)). The dominant role of the spoke system in determining the wheel stiffness is apparent in the dramatic increase in stiffness from the narrowest hub (40 mm) to the widest hub (80 mm). Although the stiffness cannot be measured at zero tension due to buckling of spokes, the extrapolated zero-tension stiffness was obtained from linear fits to each wheel. The zero-tension stiffness increases linearly with $\kuu$. A theoretical wheel with zero hub width and zero tension has a stiffness equal to the stiffness of the rim alone. The rim stiffness for the CR18-700 rim, 71 N/mm, was measured using the technique described in Chapter \ref{chap:acoustic_testing} (shown as a red dashed line in Figure \ref{fig:K2_results} (b)).

\begin{figure}[h]
\centering
\includesvg{\rootdir/figs/buckling_tension/}{K2_results}
\caption{Four-point bend test results. \textbf{(a)} Measured mode stiffness vs. spoke tension for five different hub flange spacings. \textbf{(b)} Extrapolated mode stiffness at zero tension vs. the spoke system stiffness. The bounds on all linear fits correspond to the 95\% confidence interval.}
\label{fig:K2_results}
\end{figure}

There are two notable discrepancies between the experimental results and the stiffness predicted by Equation \eqref{eq:Kn_buckling}. First, the material component of the spoke stiffness (proportional to $K_s$ and independent of tension) is lower than the theoretical stiffness by about 35\%. The extapolated zero-tension stiffness should increase commensurately with $\pi R\kuu$ (black dashed line in Figure \ref{fig:K2_results} (b)). This is possibly due to the fact that the J-bend spokes used in this study are able to deform elastically near the spoke due to the loose fit of the spoke elbow in the hub flange.

Second, the dimensionless slope $K_2$ vs. $\bar{T}$ is $-6.3 \pm 0.25$, while the slope predicted by Equation \eqref{eq:Kn_buckling} is approximately $-3\pi = -9.42$. The positive difference suggests that the increased tension causes the spokes to recover some stiffness, possibly by increasing the contact force between the spoke elbow and the edges of the hole in the hub flange. The details of the stiffness of J-bend spokes under tension is the subject of an ongoing research project.

\end{document}
