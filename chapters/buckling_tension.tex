%!TEX root = ../thesis.tex
\providecommand{\rootdir}{..}
\documentclass[\rootdir/thesis.tex]{subfiles}

\begin{document}

If the spokes are tensioned beyond a critical value, the circular shape becomes unstable and the rim will buckle into a non-planar shape. The post-buckling configuration is generally stable and the original shape of the wheel can be recovered by uniformly reducing the spoke tensions. Despite its implications for wheel strength, the buckling problem has never received a rigorous treatment. Jobst Brandt alludes to buckling in his practical manual for wheelbuilders \cite{Brandt1993}:

\begin{quote}
\emph{``If the wheel becomes untrue in two large waves during stress relieving, the maximum, safe tension has been exceeded. Approach this tension carefully to avoid major rim distortions. When the wheel loses alignment from stress relieving, loosen all spokes a half turn before retruing the wheel.''}
\end{quote}

The ``stress relieving'' referred to here is the practice of laterally loading the rim (or the spokes directly) so that the spokes temporarily increase their tension, which is presumed also to cause plastic flow in the spoke elbows so as to lower residual stresses in the service condition, and hence reduces fatigue.

Flexural-torsional buckling of rings can be treated as a special case of buckling of arches, where the included angle is allowed to go to $2\pi$ with appropriately periodic boundary conditions. Timoshenko and Gere \cite{Timoshenko1961} gave a formula for the critical load for a ring with doubly-symmetric cross-section subjected to a line load directed towards the ring's center The theory of flexural-torsional buckling of monosymmetric arches (bicycle rim sections generally have only one plane of symmetry) was broadly formalized by Trahair and Papangelis \cite{Trahair1987} using the virtual work approach to derive the equilibrium and stability equations. Their theory has been extended to treat arches with continuous \cite{Pi2002} or discrete \cite{Bradford2002} elastic supports and elastic end restraints \cite{Guo2014}.

The problem of the prestressed bicycle wheel is unique for a number of reasons. First, the load for buckling due to spoke over-tension is internal to the structure. Second, the spokes act both as elastic restraints resisting buckling and as prestressing elements promoting buckling. Third, the lateral, radial, tangential, and torsional restraining actions of the spokes are commonly coupled – e.g. lateral deflection at a spoke may produce a mix of those reactions on the rim section. These considerations extend to other structural systems. Large observation wheels such as the London Eye \cite{Mann2001} and the Singapore Flyer \cite{Allsop2009} resemble bicycle wheels and achieve lateral stability due to the bracing angle of prestressed cables, and must be designed against flexural-torsional buckling. At the biological scale, the cellular fragmentation process which leads to platelet formation may also be driven by flexural-torsional buckling instability of a ring of bundled actin fibers loaded by an elastic membrane which both promotes buckling and provides elastic restraint \cite{Stroberg2016}.


\section{Elastic stability criterion}
It was previous shown (Section \ref{sec:ModeMatrix}) that the deformed shape of the wheel under external loads could be found by solving the mode matrix equation \eqref{eq:mm_Kd_f}. One form of the elastic stability criterion states that for any admissible deformation, the second variation of the total potential \eqref{eq:mm_TotPot} must be positive. Since any rim deformation must be periodic and continuous, the modes in \eqref{eq:mm_FourierDef} form a set of kinematically admissible deformations. Therefore the buckling tension can be defined as the tension at which \eqref{eq:mm_Kd_f} admits nontrivial solutions when $\gls{Fext}=\mathbf{0}.$\todo{Check wording of precise stability criterion}

The spoke stiffness comprises a term proportional to the elastic stiffness, $\gls{Ks}$, of the spokes and a term proportional to the spoke tension, $\T$. Additionally, the rim stiffness comprises a term depending on the bending, torsion, and warping constants, and a term proportional to the net radial spoke tension per unit length, $\Tb$. In order to make the dependence on $\Tb$ explicit, we write the buckling criterion as follows:
\begin{equation}
\label{eq:Tc_crit}
\det{\left[\matl{\KmRim} + \matl{\KmSpk} + \Tb(\geom{\KmSpk} - \geom{\KmRim})\right]} = 0
\end{equation}

If the complete details of discrete spokes and possible radial/lateral coupling are retained, a numerical solution to \eqref{eq:Tc_crit} can be obtained by including enough modes such that the relevant length scales (e.g. the distance between spokes) are correctly approximated, and then either numerically solving the characteristic polynomial of \eqref{eq:Tc_crit}, or iteratively increasing $\Tb$ until the determinant is minimized. Under more restrictive assumptions, analytical solutions are possible, as shown in Table \ref{tab:BucklingSolutions}.

\begin{table}[h]
\caption{Simplifying assumptions for elastic buckling criterion.}
\label{tab:BucklingSolutions}
\begin{tabularx}{\textwidth}{p{1.2cm}Xccc}
% \hline\noalign{\smallskip}
\toprule
\centering Smeared spokes & \centering Symmetry & Rim & Complexity & Analytical solution\\
\midrule
\centering No  & \centering None & Monosymmetric$^{\rm b}$ & $\det$ of (4+8N)x(4+8N) matrix & None\\
\centering Yes & \centering None & Monosymmetric & $\det$ of 4x4 matrix & Impractical\\
\centering Yes & \centering Left-right$^{\rm a}$& Monosymmetric& Quadratic equation for $\Tb_c$ & Solution of \eqref{eq:Tc_quad}\\
\centering Yes & \centering Left-right& Bi-symmetric$^{\rm c}$ & Linear equation for $\Tb_c$ & \eqref{eq:Tcn_lin}\\
\centering Yes & \centering Left-right, no spoke offset & Bi-symmetric & Linear equation for $\Tb_c$ & \eqref{eq:tcn_lin_nophi}\\
\bottomrule
\multicolumn{5}{p{\textwidth}}{
$^{\rm a}$The distance to the hub flange and spoke arrangement are identical on the left and right sides, which implies no radial/lateral coupling.
}\\
\multicolumn{5}{p{\textwidth}}{
$^{\rm b}$Monosymmetry implies $\yo\neq 0$
}\\
\multicolumn{5}{p{\textwidth}}{
$^{\rm c}$Bi-symmetry implies $\yo=0$, i.e. the shear center and centroid coincide.
}\\
\bottomrule
\end{tabularx}
\end{table}

\subsection{Smeared spokes approximation}

If the smeared-spokes approximation is used, the mode stiffness matrix $\KmRim + \gls{KmSpkSm}$ has a block diagonal form, with each block corresponding to a different mode $n$ in \eqref{eq:mm_FourierDef}. The first two modes $n=0$ and $n=1$ are rigid-body motions and do not admit buckling. The buckling criterion then becomes

\begin{equation}
\label{eq:Tc_crit_smeared}
\det{\left[\matl{\mathbf{K}_{rim,n}} + \matl{\mathbf{\bar{K}}_{spk,n}} + \Tb(\geom{\mathbf{\bar{K}}_{spk,n}} - \geom{\mathbf{K}_{rim,n}})\right]} = 0
\end{equation}

In its most general form, Equation \eqref{eq:Tc_crit} results in a quadratic equation for $\Tb$ which can be solved to find the critical tension $\Tb_{c,n}$ for a given mode $n$. The critical buckling tension $\Tb_c$ for the wheel is the minimum $\Tb_{c,n}$ over all integer modes.

The spoke elastic stiffness $\gls{Ks}$ is much less than the spoke tension. Therefore, the tension components of the spoke stiffness matrix $\geom{\bar{k}_{ij}}$ can generally be neglected compared with $\matl{\bar{k}_{ij}}$, except for $\kuu$ due to the small lateral projection of the spoke vector. In this section we will neglect the tension component of all stiffness parameters except for $\kuu$.

\subsubsection*{No radial/lateral coupling}
If the radial/lateral spoke coupling terms are neglected ($\kuv=\kuw=\kvp=\kwp=0$), the buckling criterion \eqref{eq:Tc_crit} reduces to
\begin{multline}
\label{eq:Tc_quad}
\left(\R^2\geom{\kuu}\yo - n^2\R\yo \right)\Tb^2\\
-\left[\frac{\EIr}{\R}\left(n^2 - 3n^4 \frac{\yo}{\R} + \R\geom{\kuu}\right)
       +\frac{\widetilde{\GJ}}{\R}\left(n^4 - \frac{\yo}{\R}(2n^4 - n^2) - \R\geom{\kuu}n^2\right) \right.\\
       \left. +\R\kpp (n^2 - \R\geom{\kuu}) + 2\R\yo\kup n^2 - \R^2\kuu \yo \vphantom{\frac12} \right]\Tb\\
+\left[\frac{\EIr \widetilde{\GJ} n^2}{\R^4}(n^2-1)^2
       +\EIr\left(\kuu + 2\frac{\kup}{\R}n^2 + \frac{\kpp}{\R^2}n^4\right) \right.\\
       \left. +\widetilde{\GJ}\left(\kuu n^2 + 2\frac{\kup}{\R}n^2 + \frac{\kpp}{\R^2}n^2\right)
       +\R^2(\kuu\kpp - \kup^2)\right] = 0
\end{multline}

where $\widetilde{\GJ} = \GJ + \EIw n^2/\R^2$ is the effective torsional stiffness.

\subsubsection*{No radial/lateral coupling, bi-symmetric rim}
The quadratic term in \eqref{eq:Tc_quad} is proportional to $\yo$. If the rim cross-section is assumed to be symmetric across both the $\eo$ and $\et$ axes, then the buckling criterion \eqref{eq:Tc_quad} reduces to a linear equation for $\Tb$. Using the non-dimensional parameters defined in Section \ref{sec:Lateral}, the non-dimensionalized critical buckling tension is given by
\begin{equation}
\label{eq:Tcn_lin}
\ts_{c,n} = \left(\frac{1}{n^2 - \R\geom{\kuu}}\right)
\left(\lr_{uu}
      +\frac{(n^4 + \m n^2)\lr_{\p\p}
             +2n^2(\m + 1)\lr_{u\p} - \lr_{u\p}^2
             +\m n^2(n^2-1)^2}
        {1 + \m n^2 + \lr_{\p\p}}\right)
\end{equation}

The first term $(n^2 - \R\geom{\kuu})^{-1}$ accounts for the fact the spokes (and therefore the direction of the applied tension at the rim) rotates under a buckling displacement. In classical buckling analysis, this accounts for the difference between buckling under dead loads (e.g. gravity) and directed loads (e.g. tensioned cables). For typical wheels, $\R\geom{\kuu} \approx 1$.

\subsubsection*{No radial/lateral coupling, bi-symmetric rim, no spoke offset}
If the spokes are assumed to connect to the rim through the shear center, the lateral/torsional coupling terms vanish, i.e. $\lr_{u\pi}=\lr_{\p\p}=0$. The critical buckling tension is given by

\begin{equation}
\label{eq:tcn_lin_nophi}
\ts_{c,n} = \left(\frac{1}{n^2 - \R\geom{\kuu}}\right)
\left(\lr_{uu}
      +\frac{\m n^2(n^2-1)^2}
        {1 + \m n^2}\right)
\end{equation}

The critical tension $\ts_{c,n}$ for a given mode depends on $n$, $\m$, and $\matl{\lr_{uu}}$. Therefore, the preferred buckling mode (minimum $\ts_{c,n}$ over all integers $n$) only depends on $\m$ and $\matl{\lr_{uu}}$. Figure \ref{fig:tc_lambda_mu} (a) shows a map of the preferred buckling modes and their respective shapes. Much like the related problem of the straight beam on an elastic foundation, the bicycle wheel buckling problem exhibits a length scale which depends on the ratio of beam stiffness to foundation stiffness. In an infinite beam, the wavelength of the buckling mode varies continuously with stiffness ratio. This relationship is used to measure the elastic modulii of thin-films \cite{Suo Hutchinson or Huang}.\todo{Add citation for wrinkling of thin films on elastic substrates} In the bicycle wheel problem, the buckling wavelength must ``fit'' around the rim, i.e. the circumference must be an integer multiple of the wavelength. This constraint gives rise to the discrete mode transitions shown in Figure \ref{fig:tc_lambda_mu}.

\begin{figure}
\centering
\includesvg{\rootdir/figs/buckling_tension/}{tc_lambda-mu}
\caption{\textbf{(a)} Parameter map of preferred buckling mode. Normalized buckling tension $\ts_c$ is given by the color scale on the right. \textbf{(b)-(c)} Comparison between Equation \eqref{eq:tcn_lin_nophi} (black line), ABAQUS simulation results (blue stars), and power law approximations: \eqref{eq:Tc_hi_k} for (b) and \eqref{eq:tc_lomu} for (c). Parameters held constant are $\m=0.38$ for (b) and $\matl{\lr_{uu}}=10$ for (c).}
\label{fig:tc_lambda_mu}
\end{figure}

The equivalent spring model described in Section \ref{sec:equiv_springs} provides some insight into Equation \eqref{eq:tcn_lin_nophi}. The buckling tension can be found by solving Equation \eqref{eq:Kn_n} such that $K_{n\geq 1} = 0$.
\begin{equation}
\Tb_{c,n} = \frac{1}{\pi}\left(\frac{1}{n^2 - \R\geom{\kuu}}\right) K_n^0
\end{equation}

where $K_n^0$ is given by Equation \eqref{eq:Kn_n}, evaluated at zero tension. As with the lateral mode stiffness, the spoke stiffness plays a significant role in determining the buckling tension \emph{for a given mode, $n$}.

Why, with an infinite series of potential buckling modes available, do cyclists only seem to observe the ubiquitous ``taco'' ($n=2$) buckling shape? There are several possible explanations: first, it may be that wheels in common use fall into the upper left portion of Figure \ref{fig:tc_lambda_mu}. This is especially likely for modern double-wall rims which are considerably stiffer compared with their spoke systems than single-wall rims. As an example, a 32-hole Mavic A119 rim laced to a \SI{50}{mm} wide rim would need a spoke diameter of almost \SI{2.5}{mm} to cross the boundary into the $n=3$ region. Second, the buckling tension for such a wheel would be unreasonably high---the wheel in the previous example has a buckling tension of \SI{3.6}{kN}, or about 3.7 times the maximum recommended tension for most rims (\SI{100}{kgf}).

A third reason has to do with the sub-critical behavior. As shown in Section \ref{sec:equiv_springs}, the stiffness of each mode decreases linearly with spoke tension. At zero tension, the $n=2$ mode is always less stiff than the $n=3$ mode, etc. Even a wheel whose stiffness ratio places it in the $n=3$ region of Figure \ref{fig:tc_lambda_mu} has $K_2<K_3$ for tensions below a significant fraction of the buckling tension. As the spoke tension is incrementally increased in such a wheel, geometric imperfections in the rim with wavelengths equal to $n=2$ will be magnified by the decreasing $K_2$ stiffness. The rim will become severely distorted long before the tension is brought sufficiently high to see the $n=3$ mode appear.


\section{Finite-element buckling calculations}

I validated the theoretical predictions of Equation \eqref{eq:tcn_lin_nophi} against non-linear finite-element calculations implemented in ABAQUS 6.14. The spokes and rim were both modeled using 2-node linear beam elements including shear flexibility. A controlled tensioning strain was applied to the spokes starting from zero strain and increasing up to \SI{150}{\percent} of the strain at which buckling was expected to occur. The rim was given a geometric imperfection in the lateral coordinate (a ``wobble'') including several modes of the form $\cos{n\gls{ang}}$. The solution was obtained via an implicit integration including non-linear geometric effects. The average spoke tension was recorded at each loading step. The buckling tension was determined as the point at which the tension vs. tensioning strain curve deviated by more than \SI{2}{\percent} from linearity. Results are plotted against theory in Figure \ref{fig:tc_lambda_mu} (b) and (c).

\begin{figure}
\centering
\includesvg{\rootdir/figs/buckling_tension/}{tension_buckling_dynamic}
\caption{Non-linear post-buckling behavior under spoke tensioning. \textbf{(a)} Tension diagram and \textbf{(b)} energy diagram showing a bifurcation instability at I and unstable collapse at II. \textbf{(c)}-\textbf{(f)} Stages of the tensioning-detensioning cycle showing uniform tensioning (0-I) accompanied by radial shrinkage of the rim, bifurcation buckling (I-II) accompanied by lateral-torsional deformation, detensioning of spokes (III-IV) along a collapsed equilibrium branch, and recovery of the planar shape (IV-0).}
\label{fig:tension_buckling_dynamic}
\end{figure}

In addition to global buckling, the wheel can exhibit local buckling if any of the spokes lose tension during deformation. The initial post-buckling behavior of the bicycle wheel under tension is stable. After bifurcation, further tensioning of the spokes deforms the rim at roughly constant or slowly rising average tension. Since the buckling load of a spoke is much smaller than the initial tension, it exerts essentially no force on the rim when buckled. Therefore, local buckling of the spokes leads to a loss of stiffness which can cause global collapse. During collapse, strain energy stored during tensioning is released and the system finds a new minimum energy configuration.

Figure \ref{fig:tension_buckling_dynamic} illustrates a dynamic finite-element simulation of a complete tensioning-detensioning cycle. The spoke tensions increase uniformly until the equilibrium path bifurcates at point I. The rim follows the lower-energy path I $\rightarrow$ II by buckling and twisting out of its initial plane. As the spokes are tightened further, the rim deforms to accommodate the change in length. Although the average spoke tension remains constant, individual spokes increase or decrease their tension depending on which side of the rim they are on. At II, the spoke at the peak of each wave buckles which reduces the overall stiffness of the structure causing it to collapse to III.

The collapse at point II may also involve a mode transition. Depending on the properties of the wheel, the lowest bifurcation mode may be $n>2$. A rim with no spokes will always buckle into the lowest mode $n=2$, however the spoke stiffness may stabilize higher modes. When spokes buckle, the stiffness drops suddenly causing the system to prefer the $n=2$ mode.


\section{Closed-form solutions of \eqref{eq:tcn_lin_nophi} for special cases}

The critical tension in Equation \ref{eq:tcn_lin_nophi} is not in closed form due to the need to minimize $\ts_{c,n}$ with respect to $n$. In the following sections we will consider several special cases with approximate solutions which transform Equation \eqref{eq:tcn_lin_nophi} into a closed-form power law.

\subsection{Low torsional stiffness}
As the ratio $\GJ/\EIr$ tends towards zero, the mode number $n$ tends towards infinity. When the mode number is large, the buckling mode can then be found by approximating the discrete mode number $n$ with a continuous variable $\bar{n}$. We will adopt the following ansatz: (a) $\bar{n}^2 \gg 1$, and (b) $\m\bar{n}^2 \ll 1$. Under these conditions, Equation \eqref{eq:tcn_lin_nophi} becomes
\begin{equation}
\label{eq:tcn_lomu}
\ts_{c,n} = \left(\m \bar{n}^4 + \frac{\lr_{uu}}{\bar{n}^2}\right)
\end{equation}

Minimizing \eqref{eq:tcn_lomu} with respect to $\bar{n}$ yields
\begin{equation}
\label{eq:n_lomu}
\bar{n}=\left(\frac{\lr_{uu}}{2\m}\right)^{1/6}
\end{equation}

Inserting \eqref{eq:n_lomu} into \eqref{eq:tcn_lomu}, we obtain the critical buckling tension, independent of mode number:
\begin{equation}
\label{eq:tc_lomu}
\ts_c = \left(\frac{1}{2^{2/3}} + 2^{1/3}\right)\m^{1/3}\lr_{uu}^{2/3}
\end{equation}

Substituting definitions for dimensionless quantities $\ts,\m,\lr_{uu}$, we obtain a critical buckling tension which is independent of lateral bending stiffness, $\EIr$:
\begin{equation}
\label{eq:Tc_lo_mu}
\Tb_c = 1.89 \left( \frac{\GJ}{\R} \right)^{1/3} \kuu^{2/3}
\end{equation}

Equation \eqref{eq:Tc_lo_mu} further illustrates the consequences of over-designing the bending stiffness while neglecting the torsional stiffness. Since the bending and torsional stiffnesses act like equivalent springs connected in series, the stiffness---and therefore the buckling resistance---will be dominated by the smaller of the two. Thus a very wide, but very shallow rim (e.g. a ``fat-bike'' rim) will be entirely dominated by its torsional stiffness.

\subsection{Moderate torsion stiffness, stiff spoke system}
Modern rims are often constructed from hollow extruded aluminum profiles. As a result they have high torsional resistance $\GJ$ and negligible warping coefficient $\EIw$. Therefore $\EIw=0$ and $\GJ \sim \EIr$. If the spoke stiffness is much higher than the rim stiffness, i.e. $\lr_{uu} \gg 1$, then we can make a similar argument as in the previous section. Now we will accept as an ansatz that $n^2 \gg 1$ and $\m \sim 1$. Estimating the discrete variable $n$ with a continuous analog $\bar{n}$ Equation \eqref{eq:tcn_lin_nophi} becomes
\begin{equation}
\label{eq:tcn_hi_k}
\ts_{c,n} = \left(\bar{n}^2 + \frac{\lr_{uu}}{\bar{n}^2}\right)
\end{equation}

Minimiizing \eqref{eq:tcn_hi_k} with respect to $\bar{n}$ gives the scaling law $\bar{n}=(\lr_{uu})^{1/4}$. Inserting into \eqref{eq:tcn_hi_k} gives $\ts_c = 2(\lr_{uu})^{1/2}$. In terms of the net radial tension per unit length, $\Tb$, this gives
\begin{equation}
\label{eq:Tc_hi_k}
\Tb_c = 2 \left(\frac{\matl{\kuu}\EIr}{\R^2}\right)^{1/2}
\end{equation}

Noting that the axial force in the rim is $\gls{Fax}=\R\Tb$, we recognize \eqref{eq:Tc_hi_k} as the critical buckling load for an infinite beam on an elastic foundation given by Hetenyi \cite{Hetenyi1946}. The rim buckles as if it were a straight beam since $\lr_{uu}$ implies that the rim radius is large compared to the characteristic length of the beam on an elastic foundation.

\subsection{All spokes lie in the plane of the wheel}

If all the spokes were laced to the same flange of the hub, the spokes will all lie in the plane of the rim, implying $\matl{\kuu}=0$. In this case, the rim will always buckle into the $n=2$ mode and Equation \eqref{eq:tcn_lin_nophi} simplifies to
\begin{equation}
\ts_c = \left(\frac{9\m}{1 + 4\m}\right)\left(\frac{4}{4 - \R\geom{\kuu}}\right)
\end{equation}

If the spokes meet at the center of the rim (i.e. a hub of zero diameter), then $\R=\gls{ls}$ and the critical reduced tension becomes
\begin{equation}
\label{eq:tc_Hencky}
\ts_c = \frac{12\m}{1 + 4\m}
\end{equation}

This is precisely the result obtained by Hencky for the critical distributed radial load for a thin ring \cite{Timoshenko1961}. Timoshenko obtained a slightly different result which differs from Equation \eqref{eq:tc_Hencky} by a factor of  $3/4$, by considering dead loads which do not change direction during buckling. This case illustrates the importance of considering the ``tension-stiffness'' when deriving the spoke forces. The change in direction of the spoke force after deflection produces a small restoring force on the rim which increases the buckling load by a factor of $4/(4 - \R\geom{\kuu})$, or about \SI{33}{\percent}. Wheels with larger $\R/\gls{ls}$, and thus a larger change in spoke angle for a given lateral deflection---for example high-flange hubs or hub motors---will receive an even larger benefit from the tension-stiffness term.

\section{Buckling of real wheels}
\label{sec:K2_tension_buckling}

Several practical considerations make it impossible to reach the critical tension in a real wheel: (1) the buckling tension can be higher than the yield point of the spoke itself, (2) friction at the spoke nipple/rim interface becomes too great to overcome with a spoke wrench, and (3) the rim will go out of true laterally at around \SI{50}{\percent} of the critical tension due to imperfections in the rim.

Despite these difficulties, the mode stiffness model described in Section \ref{sec:equiv_springs} provides a method of estimating the critical buckling tension by directly measuring the stiffness of a single mode ($n=2$) as a function of spoke tension. Under the assumptions described in \ref{sec:equiv_springs}, the stiffness of the $n$th mode is:
\begin{equation}
\label{eq:Kn_buckling}
\gls{Knlat}(\T) = \gls{KnRim} + \pi \R (\matl{\kuu} + \Tb \geom{\kuu}) - \pi n^2 \Tb
\end{equation}

The linear dependence of $\kuu$ on $\Tb$ is made explicit in Eqn. \eqref{eq:Kn_buckling} to show that $K_n$ also depends linearly on $\Tb$. By measuring the second mode stiffness (using a 4-point bending arrangement) at several spoke tensions and then extrapolating the data to $K_2 = 0$, the critical tension for the second mode can be obtained.

\subsection{Method}

I made measurements on five different wheels, all constructed with the same rim and spokes, but with varying distance between hub flanges. For each wheel, I made several measurements at increasing spoke tensions until the wheel started to buckle enough to create a significant difference between left and right spoke tensions at the anti-nodes of the rim.

\subsubsection{Wheel construction}

I built five different wheels using the same rim and spokes, but with varying distance between hub flanges. The rim was a Sun CR18 700C, a narrow road rim with a very shallow double-wall construction and spoke nipple eyelets drilled for 36 spokes. The spokes were Wheelsmith double-butted spokes with an end diameter of 2.0 mm and 1.70 mm along the swaged section. The hub was a custom-built adjustable-width hub designed by a Northeastern University capstone team advised by Jim Papadopoulos.

All wheel configurations were radially spoked with all inbound spokes. I made no attempt to keep the wheel laterally or radially true within any specification, but the rim was symmetrically dished and I ensured that no individual spoke deviated in tension by more than \SI{10}{\percent} of the average tension. This eventually became impossible when the average spoke tension exceeded about \SI{50}{\percent} of the critical tension and the wheel started to distort into a taco shape.

\subsubsection{Four-point bending test}

\begin{figure}[t]
\centering
\includesvg{\rootdir/figs/buckling_tension/}{4pt_bend}
\caption{Four-point bend test. \textbf{(a)} Experimental setup showing load fixture, applied load, and measurement point. \textbf{(b)} Measured lateral displacement around the rim when loaded at $\gls{ang}=-\pi$. The symmetric part of the displacement is found by subtracting $(u_l/2)\cos{\gls{ang}}$ from the measured displacement.}
\label{fig:4pt_bend_setup}
\end{figure}

Each wheel was supported from the bottom at the 3-, and 9-o'clock positions, and loaded by hanging weights at the 12-o'clock position (Figure \ref{fig:4pt_bend_setup} (a)). The rim was oriented so that the sleeve joint and the valve hole fell midway between supports where the bending moment is minimized. Due to the symmetry of the boundary conditions, the work done against the odd modes ($n=1, 3, ...$) and even modes divisible by 4 ($n=4, 8, ...$) is identically zero. Since the hub is unsupported, the zero mode is also eliminated. Due to the rapid increase in mode stiffness with $n$, the $n=2$ mode accounts for about \SI{95}{\percent} of the strain energy, while the other \SI{5}{\percent} is spread across the remaining modes ($n=6, 10, ...$). Figure \ref{fig:4pt_bend_setup} (b) shows the measured lateral displacement around the rim for a point load located at $\gls{ang}=-\pi$. After subtracting the rigid body rotation, the measured displacement closely matches $\cos{2\gls{ang}}$. The second mode stiffness is related to the load-displacement slope of the four-point bend test by $K_2 = 16(P/u_l)$, where $P$ is the applied load and $u_l$ is the deflection of the load point.

\subsection{Results}
\label{sec:K2_T_results}

\begin{figure}[t]
\centering
\includesvg{\rootdir/figs/buckling_tension/}{K2_results}
\caption{Four-point bend test results. \textbf{(a)} Measured mode stiffness vs. spoke tension for five different hub flange spacings. \textbf{(b)} Extrapolated mode stiffness at zero tension vs. the spoke system stiffness. The red dashed line is the rim mode stiffness, while the black dashed line is the theoretical mode stiffness from Equation \eqref{eq:Kn_buckling}. The bounds on all linear fits correspond to the \SI{95}{\percent} confidence interval.}
\label{fig:K2_results}
\end{figure}

As predicted by Equation \eqref{eq:Kn_buckling}, the measured mode stiffness decreases with applied tension (Figure \ref{fig:K2_results} (a)). The dominant role of the spoke system in determining the wheel stiffness is apparent in the dramatic increase in stiffness from the narrowest hub (40 mm) to the widest hub (80 mm). Although the stiffness cannot be measured at zero tension due to buckling of spokes, the extrapolated zero-tension stiffness was obtained from linear fits to each wheel. The zero-tension stiffness increases linearly with $\kuu$. A theoretical wheel with zero hub width and zero tension has a stiffness equal to the stiffness of the rim alone. The rim stiffness for the CR18-700 rim, \SI{71}{N/mm}, was measured using the technique described in Chapter \ref{chap:acoustic_testing} (shown as a red dashed line in Figure \ref{fig:K2_results} (b)).

There are two notable discrepancies between the experimental results and the stiffness predicted by Equation \eqref{eq:Kn_buckling}. First, the material component of the spoke stiffness (proportional to $\gls{Ks}$ and independent of tension) is lower than the theoretical stiffness by about \SI{35}{\percent}. The extrapolated zero-tension stiffness should increase commensurately with $\pi R\kuu$ (black dashed line in Figure \ref{fig:K2_results} (b)). This is possibly due to the fact that the J-bend spokes used in this study are able to deform elastically near the spoke due to the loose fit of the spoke elbow in the hub flange.

Second, the dimensionless slope $K_2$ vs. $\Tb$ is \num{-6.3+-0.25}, while the slope predicted by Equation \eqref{eq:Kn_buckling} is approximately $-3\pi = -9.42$. The positive difference suggests that increasing the tension causes the spokes to recover some stiffness, possibly by increasing the contact force between the spoke elbow and the edges of the hole in the hub flange. The details of the stiffness of J-bend spokes under tension is the subject of an ongoing research project.

\end{document}
