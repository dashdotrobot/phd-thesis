\providecommand{\rootdir}{..}
\documentclass[\rootdir/thesis.tex]{subfiles}

\begin{document}

Most modes of mechanical failure of the wheel are progressive and preventable: gradual loosening of spokes leads to misalignment, repetitive stressing of the spokes leads to localized failure by fatigue, wear on the braking surfaces thins the rim sidewalls and increases the likelihood of tire blow-outs. The wheel can also fail suddenly under excessive loads by buckling. Many cyclists know this failure mode as the ``taco,'' due to the tendency of the wheel to form a saddle-like shape and fold in on itself.

\missingfigure[figwidth=6.5in,figheight=3in]{Photo of wheel taco}

Thus far we have only considered buckling under internal forces (which is primarily a concern of the wheelbuilder, not the rider), and small deformations of the wheel under external loads such that the structural response remains linear. There are two sources of nonlinearity under large deformations: (1) material nonlinearity, in which the constitutive law of the material itself is nonlinear---e.g. plasticity--- and (2) geometric nonlinearity, in which the deformation of the structure is large enough such that the small-angle approximation is no longer valid. Here we will assume that material nonlinearity is not present. A typical bicycle wheel is a slender structure capable of large deformations with small strains. Furthermore, the onset of plasticity will depend on the exact shape of the rim cross-section.

\section{Buckling under lateral force}

The wheel is considerably weaker in the lateral direction than the radial direction due to the small out-of-plane bracing angle. Relatively small side loads can buckle spokes or cause the entire rim to collapse. Lateral failure occurs as a result of two complimentary failure mechanisms: buckling of spokes due to insufficient tension, and buckling of the rim due to excessive spoke tension. Increasing the spoke tension increases the lateral displacement required to de-tension spokes, but decreases the lateral stiffness due to the compressive axial stresses in the rim.

To illustrate these two competing failure mechanisms, I performed finite-element calculations in ABAQUS 6.13 on a bicycle wheel with parameters corresponding to a typical lightweight road bike wheel. The hub is rigidly fixed, while the lateral load and lateral displacement of a point on the rim are simultaneously calculated using the Riks algorithm. Arc-length methods like the Riks algorithm are capable of tracing load-displacement curves which exhibit both snap-through instabilities (unstable under load control) and snap-back instabilities (unstable under displacement control).

\begin{figure}
\centering
\includesvg{\rootdir/figs/buckling_ext_loads/}{lat_buckling} 
\caption{Buckling under lateral force. \textbf{(a)} Snapshot from ABAQUS simulation. The bottom node is displaced in the z-direction. The spokes opposite the side of the applied load (red) have lost tension. \textbf{(b)} Equilibrium load-displacement curves for a vintage road bike wheel for several spoke tensions. Lightest: $T=0.1T_c$, darkest: $T=0.6T_c$. \textbf{(c)} Failure diagram for the same wheel showing the region of no spoke buckling (green), buckled spokes but positive stiffness (yellow), and wheel collapse (red). The markers are from ABAQUS simulations and the dashed line is from Eqn. \eqref{eq:Pc_lat}.}
\label{fig:lat_buckling}
\end{figure}

Figure \ref{fig:lat_buckling} (b) shows the calculated load-displacement curves for the same road bike wheel at six different tensions corresponding to $T=0.1T_c, 0.2T_c,...,0.6T_c$. The initial behavior is linear until the onset of spoke buckling. After the peak load, there is a load drop with a corresponding release of potential energy stored in the pretensioned spokes and the rim buckles into a taco shape. The magnitude of the load-drop (and corresponding snap-back instability) increases with spoke tension. Beyond about $0.5 T_c$, the load-displacement curve crosses the zero-axis, indicating the existence of a buckled state which can be maintained with no external load. Many a hapless wheelbuilder has discovered this unexpected equilibrium solution by vigorously stress-relieving a wheel. It is likely that Jobst Brandt was referring to this state in his practical advice quoted at the beginning of Chapter \ref{tension_buckling}.

The competition between failure mechanisms of spoke buckling and rim collapse leads to a failure diagram like that shown in Figure \ref{fig:lat_buckling} (c). The lateral displacement required to buckle spokes is
\begin{equation}
u_{sb} = \frac{T}{K_s c_1}
\end{equation}

where $K_s$ is the axial stiffness of a spoke and $c_1$ is the direction cosine of the spoke in the $\eo$ direction. Substituting the lateral load and tension-dependent lateral stiffness and eliminating $u_{sb}$, we obtain the lateral force required to buckle one spoke:
\begin{equation}
\label{eq:Pc_lat}
P_{sb} = \frac{K_{lat}(T) T}{K_s c_1}
\end{equation}

The failure diagram shown in Figure \ref{fig:lat_buckling} (c) suggests a simple rule-of-thumb for the optimum spoke tension for supporting side loads: A rough but satisfactory approximation to the tension-dependent lateral stiffness for a wheel is $K_{lat} (T)=K_{lat} (0)(1-T/T_c)$, where $K_{lat} (0)$ is the theoretical lateral stiffness at zero spoke tension. Inserting this approximation into Eqn. \eqref{eq:Pc_lat} and maximizing the lateral force gives an optimum spoke tension of $T_{opt} = 0.5 T_c$. For typical wheels, this is just below the spoke tension which admits equilibrium buckled states, and just below the tension at which the pre-buckling of the rim due to imperfections becomes intolerable. We therefore propose that Brandt’s tension criterion, developed through practical experience, in fact has a firm theoretical foundation.

\section{Buckling under radial force}

The buckled shape of the wheel under a radial load is very similar to the buckled shape under lateral load. At a critical radial load the wheel takes on a non-planar shape by lateral bending and twisting. The subsequent post-buckling stiffness is generally zero or negative (unstable), which leads to collapse under dead loads. In a typically wheel, radial loads cause spokes to buckle in a narrow region beneath the hub before global buckling. Therefore, the pre-buckled state is highly non-linear. Furthermore, the pre-buckling displacements can be quite large and so cannot necessarily be neglected in a simple bifurcation analysis.

For these reasons, a theoretical prediction of the critical radial load is an extremely challenging task. Nevertheless, a satisfactory approximation may be obtained by separately considering the two competing failure modes: spoke buckling and rim buckling. In this section, I will derive an approximate formula for the radial load to cause lateral buckling based on the following assumptions:

\begin{enumerate}
\item{The maximum radial load does not depend significantly on spoke tension.}
\item{The buckling mode shape is identical to the deformed shape of the wheel under a small lateral load.}
\item{During buckling, the load point moves along a circular trajectory centered at the center of the rim in the plane defined by the basis vectors $\eo$ and $\et$.}
\item{The radial load at which spokes begin to buckle is proportional to spoke tension.}
\end{enumerate}

\subsection{Buckling under radial load and effect of spoke tension}
...

\subsection{Rim buckling with laterally constrained spokes: hinged-column model}
...

\subsection{Onset of spoke buckling}
...

\subsection{Competing failure mode model}
...


\end{document}
