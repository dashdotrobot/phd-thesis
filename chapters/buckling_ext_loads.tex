%!TEX root = ../thesis.tex
\providecommand{\rootdir}{..}
\documentclass[\rootdir/thesis.tex]{subfiles}

\begin{document}

\begin{figure}[b]
\centering
\includesvg{\rootdir/figs/buckling_ext_loads/}{bike_with_taco_wheel}
\caption{A bicycle with a buckled wheel, spotted on Northwestern's campus.}
\label{fig:bike_taco_wheel}
\end{figure}

Most modes of mechanical failure of the wheel are progressive and preventable: gradual loosening of spokes leads to misalignment, repetitive stressing of the spokes leads to localized failure by fatigue, wear from rim brakes thins the rim sidewalls and increases the likelihood of tire blow-outs. The wheel can also fail suddenly under excessive loads by buckling. Many cyclists know this failure mode as a ``taco,'' due to the tendency of the wheel to form a saddle-like shape and fold in on itself.

Thus far we have only considered buckling under internal forces (which is primarily a concern of the wheelbuilder, not the rider), and small deformations of the wheel under external loads such that the structural response remains linear. There are two sources of nonlinearity under large deformations: (1) material nonlinearity, in which the constitutive law of the material itself is nonlinear---e.g. plasticity---and (2) geometric nonlinearity, in which the deformation of the structure is large enough such that the small-angle approximation is no longer valid. Here we will assume that material nonlinearity is not present. A typical bicycle rim is a slender structure capable of large deformations with small strains. Furthermore, the onset of plasticity will depend on the exact shape of the rim cross-section.

For validation and illustration, I present non-linear finite-element simulations on seven hypothetical wheels. The wheel properties are given in Appendix \ref{app:wheel_library}. The ``high wheel'' represents the kind of wheel that would be found on an Ordinary, or ``penny-farthing,'' bicycle popular in the 1880s. The ``small'' wheel might be found on an adult folding bicycle or a youth bicycle. All of the properties are hypothetical and are not meant to correspond to any particular wheel or product.


\section{Buckling under lateral force}
\label{sec:buckling_lat_force}

The wheel is considerably weaker in the lateral direction than the radial direction due to the small out-of-plane bracing angle. Relatively small side loads can buckle spokes or cause the entire rim to collapse. Lateral failure occurs as a result of two complimentary failure mechanisms: buckling of spokes due to insufficient tension, and buckling of the rim due to excessive spoke tension. Increasing the spoke tension increases the lateral displacement required to de-tension spokes, but decreases the lateral stiffness due to the compressive axial stresses in the rim.

To illustrate these two competing failure mechanisms, I performed finite-element simulations in ABAQUS 6.13 using the ``vintage road'' wheel parameters. The hub is rigidly fixed, while the lateral load and lateral displacement of a point on the rim are simultaneously calculated using a version of the Riks algorithm \cite{Crisfield1981}. Arc-length methods like the Riks algorithm are capable of tracing load-displacement curves which exhibit both snap-through instabilities (unstable under load control) and snap-back instabilities (unstable under displacement control).

\begin{figure}
\centering
\includesvg{\rootdir/figs/buckling_ext_loads/}{lat_buckling} 
\caption{Buckling under lateral force. \textbf{(a)} Snapshot from ABAQUS simulation. The bottom node is displaced in the z-direction. The spokes opposite the side of the applied load (red) have lost tension. \textbf{(b)} Equilibrium load-displacement curves for the ``vintage road'' bike wheel at several spoke tensions. Lightest: $\T=0.1\T_c$, darkest: $\T=0.7\T_c$. \textbf{(c)} Failure diagram for the same wheel showing the region of no spoke buckling (green), buckled spokes but positive stiffness (yellow), and wheel collapse (red). The markers are from ABAQUS simulations and the dashed line is from Eqn. \eqref{eq:Pc_lat}.}
\label{fig:lat_buckling}
\end{figure}

Figure \ref{fig:lat_buckling} (b) shows the calculated load-displacement curves for the same wheel at seven different tensions corresponding to $\T=0.1\T_c, 0.2\T_c,...,0.7\T_c$. The initial behavior is linear until the onset of spoke buckling. After the peak load, there is a load drop with a corresponding release of potential energy stored during the prestressing process and the rim buckles into a taco shape. The magnitude of the load-drop (and corresponding snap-back instability) increases with spoke tension. Beyond a critical tension, the load-displacement curve crosses the zero-axis, indicating the existence of a buckled state which can be maintained with no external load. Many a hapless wheelbuilder has discovered this equilibrium solution by vigorously stress-relieving a wheel at high tension. It is likely that Jobst Brandt was referring to this state in his practical advice quoted at the beginning of Chapter \ref{chap:tension_buckling}.

The competition between failure mechanisms of spoke buckling and rim collapse leads to a failure diagram like that shown in Fig. \ref{fig:lat_buckling} (c). The lateral displacement required to buckle spokes is
\begin{equation}
u_{sb} = \frac{\T}{\gls{Ks} c_1}
\end{equation}

where $\gls{Ks}$ is the axial stiffness of a spoke and $c_1$ is the direction cosine of the spoke in the $\eo$ (lateral) direction. Converting the buckling displacement $u_{sb}$ to a load using the tension-dependent lateral stiffness, $K_{lat}(\T)$, we obtain the lateral load to buckle a single spoke:
\begin{equation}
\label{eq:Pc_lat}
\gls{Psb} = \frac{\gls{Klat}(\T) \T}{\gls{Ks} c_1}
\end{equation}

The failure diagram shown in Fig. \ref{fig:lat_buckling} (c) suggests a simple rule-of-thumb for the optimum spoke tension for supporting side loads: A rough but satisfactory approximation to the tension-dependent lateral stiffness for a wheel is $\gls{Klat} (\T)=K_{lat}^0 (0)(1-\T/\T_c)$\footnote{The approximation is identical to the ``knock-down factor'' or ``amplification factor'' used in the design of beams or columns which carry both lateral and compressive axial loads \cite{Timoshenko1961}. Although it is almost exact for a straight beam-column laterally loaded at its midpoint, it is only approximately true, and conservative, for the bicycle wheel. See Fig. \ref{fig:Klat_tension} for an illustration of $\gls{Klat}$ vs. $\T$ for a typical wheel.}, where $K_{lat}^0$ is the theoretical lateral stiffness at zero spoke tension. Inserting this approximation into Eqn. \eqref{eq:Pc_lat} and maximizing the lateral force gives an optimum spoke tension of $\T_{opt} = 0.5 \T_c$. For typical wheels, this is just below the spoke tension which admits equilibrium buckled states, and just below the tension at which the pre-buckling of the rim due to imperfections becomes intolerable. I therefore propose that Brandt's tension criterion, developed through practical experience, in fact has a theoretical basis.


\section{Buckling under radial force}

The buckled shape of the wheel under radial load is very similar to the buckled shape under lateral load. At a critical radial load the wheel takes on a non-planar shape by lateral bending and twisting. The subsequent post-buckling stiffness is generally zero or negative (unstable), which leads to collapse under dead loads. In a typically wheel, radial load causes spokes to buckle in a narrow region beneath the hub before global buckling. Therefore, the pre-buckling structural response is non-linear. Furthermore, the pre-buckling displacements can be quite large and so cannot necessarily be neglected.

For these reasons a theoretical prediction of the critical radial load is an extremely challenging task. Nevertheless, a satisfactory approximation may be obtained by separately considering the two competing failure modes: spoke buckling and rim buckling. In this section I will derive an approximate formula \cite{Ford2017} for the radial strength based on the following assumptions:

\begin{enumerate}
\item{The maximum radial load does not depend significantly on spoke tension.}\label{assum:no_T_effect}
\item{The buckling mode shape is identical to the deformed shape of the wheel under a small lateral load.}\label{assum:lat_shape}
\item{During buckling, the point on the rim where the load is applied moves along a circular trajectory whose center lies at the center of the wheel (inside the hub) in the plane defined by the basis vectors $\eo$ and $\et$.}\label{assum:circ_path}
\item{The radial load at which spokes begin to buckle is proportional to spoke tension.}\label{assum:Psb_T}
\end{enumerate}

\subsection{Effect of spoke tension}

\begin{figure}[t]
\centering
\includesvg{\rootdir/figs/buckling_ext_loads/}{rad_buckling_beam}
\caption{ABAQUS simulations of radial buckling at different spoke tensions. \textbf{(a)} Load-displacement curves for the vintage road bike wheel at different spoke tensions. \textbf{(b)} Buckling load, normalized by buckling load at $\T=0$ vs. normalized spoke tension.}
\label{fig:rad_buckling_beam}
\end{figure}

A range of typical load-displacement curves for symmetric wheels under radial load are shown in Fig. \ref{fig:rad_buckling_beam} (a). Initially, the load is proportional to the radial displacement. At a first critical load, \gls{Psb}, depending on the spoke tension, the spoke or spokes directly underneath the hub lose tension causing a sudden change in stiffness. The load continues to rise as nearby spokes participate in balancing the load. At a second critical load, $\gls{Pc}$, the lateral stiffness of the wheel (now reduced due to the buckled spokes) is no longer sufficient to maintain a planar shape and the rim begins to deflect laterally. Beyond this point, the post-buckling stiffness is negative, leading to unstable collapse under load control. If the spoke tension is sufficiently high, there is a second collapse point at which the inward pull of the spokes on the already-distorted rim causes it to collapse into a fully-developed taco shape. This collapsed shape may remain after the radial load is removed, even if the material behavior is fully elastic.

Increasing the spoke tension increases $P_{sb}$ because the spokes can lose more tension before going slack. However, it also reduces the lateral stiffness. These two effects very roughly balance each other so that changing the spoke tension does not have a large effect on the peak radial load (Fig. \ref{fig:rad_buckling_beam} (b)). A low-tension wheel buckles at a much higher radial displacement, but at roughly the same force as a high-tension wheel. This trade-off is the basis of assumption (\ref{assum:no_T_effect}).

\subsection{Rim buckling with laterally constrained spokes}

If the spokes are laterally constrained so that they do not buckle, the structural response remains linear up to the point of rim buckling. At a critical load, the rim undergoes a bifurcation instability and the lateral displacement increases sharply. The critical load decreases with tension, as shown in Fig. \ref{fig:rad_buckling_truss} (c).

\subsubsection*{Single-degree-of-freedom rim buckling model}

\begin{figure}[t]
\centering
\includesvg{\rootdir/figs/buckling_ext_loads/}{rad_buckling_truss} 
\caption{ABAQUS simulations of radial buckling with laterally restrained spokes. \textbf{(a)} Radial load vs. lateral displacement for cheap MTB wheel at several tensions. The lateral displacement follows a classical bifurcation buckling path, dependent on spoke tension. \textbf{(b)} For a wide range of wheel types and spoke tensions, the critical radial load is well approximated by $\gls{Pc} = K_{lat}R$. Markers and colors same as Fig. \ref{fig:rad_buckling_beam} (b). \textbf{(c)} Critical radial load normalized by critical radial load for the same wheel at $T=0$.}
\label{fig:rad_buckling_truss}
\end{figure}

Since spoke buckling no longer plays a role when the spokes are laterally constrained, we can isolate the rim buckling failure mode. Under the assumptions outlined above, an analytical solution for the bifurcation load is possible.

As the rim deflects laterally, the radial load projects a small lateral component onto the rim. Since the lateral stiffness is much less than the radial stiffness, I assume (\ref{assum:lat_shape}) that the effect of the radial displacement on the deformed lateral shape is negligible, and that the deformed shape of the rim is identical to the deformed shape under a pure lateral load. Therefore the deformed shape is fully characterized by a single parameter: the lateral deflection of the load point, $u_l$. A direct result of this assumption is that the increase in strain energy under a virtual displacement $\delta u_l$ is exactly
\begin{equation}
\label{eq:U_rim_1dof}
U = \frac{1}{2} \gls{Klat} \delta u_l^2
\end{equation}

If the extension of the rim centerline during buckling is assumed to be zero, the rim must pull in radially as it deflects laterally. I assume (\ref{assum:circ_path}) that the buckling path of the load point follows a circular trajectory shown in Fig. \ref{fig:hinged_column_model} (b). During buckling, the load $P$ moves through a virtual displacement $\delta v_l$. The corresponding reduction in potential energy of external loads is
\begin{equation}
\label{eq:V_rim_1dof}
V = -P\delta v_l = -P \delta u_l \left(\\R - \sqrt{R^2 - \delta u_l^2}\right)
\approx -P \left(\frac{\delta u_l^2}{2\R}\right)
\end{equation}

Taking the second variation of the total potential energy $U-V$ yields the stability criterion:
\begin{equation}
\label{eq:ddPi_1dof}
\delta^2 \Pi = \gls{Klat} - \frac{P}{\R}
\end{equation}

The critical bifurcation load is
\begin{equation}
\label{eq:Pc_1dof}
\gls{Pc} = \gls{Klat}\R
\end{equation}

Finite-element results are compared against Eqn. \eqref{eq:Pc_1dof} in Fig. \ref{fig:rad_buckling_truss} (b). The single degree-of-freedom buckling model gives a satisfactory approximation over almost two orders of magnitude. Increasing the spoke tension reduces the lateral stiffness (see Section \ref{sec:Lateral}) in a predictable manner. The buckling load as a function of tension can be approximated with a simple linear model (Fig. \ref{fig:rad_buckling_truss} (c), dashed line):
\begin{equation}
\label{eq:P_c_T}
\frac{\gls{Pc}}{K_{lat}^0 R} = 1 - \frac{\T}{\T_c}
\end{equation}

where $K_{lat}^0$ is the lateral stiffness at zero spoke tension.

\subsubsection*{Hinged-column model}

\begin{figure}[t]	
\centering
\includesvg{\rootdir/figs/buckling_ext_loads/}{hinged_column_model}
\caption{\textbf{(a)-(b)} Single-degree-of-freedom model of bicycle wheel buckling. The load point is assumed to move along a circular trajectory. \textbf{(c)} Buckling of a rigid, hinged, elastically restrained column. \textbf{(d)} Post-buckling curves for the rigid hinged column model. The post-buckling stability depends on the relative stiffness of the linear and rotational springs.}
\label{fig:hinged_column_model}
\end{figure}

The single degree-of-freedom wheel buckling model suggests an analogy with the system shown schematically in Fig. \ref{fig:hinged_column_model} (b). A rigid column, pinned at one end, is restrained by a linear spring at the tip and a torsional spring at the pivot. The deformation is completely characterized by the rotation angle, $\p$.

The total potential energy in the deformed configuration is
\begin{equation}
\label{eq:TotPot_hcm}
\Pi = \frac{1}{2}k_u(\R\sin{\p})^2 + \frac{1}{2}k_{\p}\p^2
    - PR(1-\cos{\p})
\end{equation}

Setting the first variation of \eqref{eq:TotPot_hcm} to zero gives the equilibrium condition.
\begin{equation}
\label{eq:equil_hcm}
k_u R^2 \sin{\p}\cos{\p} + k_{\p}\p - P\R\sin{\p} = 0
\end{equation}

The two possible solutions to \eqref{eq:equil_hcm} are
\begin{equation}
\label{eq:P_hcm}
\begin{cases}
\p = 0\\
P = \frac{k_{\p}}{\R}\left(\frac{\p}{\sin{\p}}\right) + k_u \R\cos{\p}
\end{cases}
\end{equation}

The torsional stiffness can be re-expressed in terms of the effective lateral stiffness of the column at its tip, $k'_{\p}=k_{\p}/\R^2$. The trivial solution bifurcates at the critical load $\gls{Pc}=(k'_{\p} + k_u) \R$. Note the similarity between this result and the critical load for the single degree-of-freedom wheel model, Eqn. \eqref{eq:Pc_1dof}.

Expanding \eqref{eq:P_hcm} about $\p=0$ and retaining only up to second-order terms, we obtain
\begin{equation}
\label{eq:P_series_hcm}
P = \gls{Pc} + \left(\frac{1}{6}k'_{\p} - \frac{1}{2}k_u\right)\R\p^2 + ...
\end{equation}

The post-buckling behavior of the column is stable (increasing load) only if the second term is positive. This condition can be expressed as
\begin{align}
\begin{split}
\label{eq:pb_stability_hcm}
k'_{\p} > 3k_u & \,\,\,\text{stable}\\
k'_{\p} \leq 3k_u & \,\,\,\text{unstable}
\end{split}
\end{align}

\subsection{Competing failure mode model}
\label{sec:Pc_nb}

\begin{figure}[t]
\centering
\includesvg{\rootdir/figs/buckling_ext_loads/}{radial_buckling_Pc_nb}
\caption{Comparison of competing failure modes model, Eqn. \eqref{eq:P_c_nb}, with ABAQUS radial buckling simulations. The markers represent the mean buckling load for a range of spoke tensions from $\T=0$ to $\T=0.95T_c$, while the error bars correspond to the full range of buckling loads for each wheel.}
\label{fig:Pc_rad_theory_comp}
\end{figure}

Since the load-displacement curve is linear up to the point of spoke buckling (assumption (\ref{assum:Psb_T})), the load required to buckle the bottom-most spoke is
\begin{equation}
\label{eq:P_sb}
P_{sb} = \frac{\gls{Krad} \T}{\gls{Ks} c_2}
\end{equation}

where \gls{Krad} is the radial stiffness of the wheel, $\gls{Ks}$ is the elastic stiffness of the spoke, and $c_2$ is the direction cosine of the spoke in the $\et$ (radial) direction. Unlike the lateral stiffness, the radial stiffness is not significantly sensitive to spoke tension.

Combining Eqns. \eqref{eq:P_sb} and \eqref{eq:P_c_T} and solving for $\T$ gives the tension at which spoke buckling and rim buckling will occur simultaneously:
\begin{equation}
\label{eq:T_nb}
\T^* = \frac{\T_c}{\left(1 + \frac{\gls{Krad} \T_c}{K_{lat}^0 \gls{Ks} \R c_2}\right)}
\end{equation}

Substituting \eqref{eq:T_nb} into \eqref{eq:P_c_T}, substituting the elastic stiffness for a straight-gauge spoke $\gls{Ks}=\gls{Espk}\gls{As}/\gls{ls}$, and noting that $\gls{ls} \approx R$ and $c_2\approx 1$ yields the critical load for simultaneous spoke and rim buckling:
\begin{equation}
\label{eq:P_c_nb}
\gls{Pc} = K_{lat}^0 \R \left(\frac{1}{1 + \frac{\gls{Espk}\gls{As}}
						{\T_c}\frac{K_{lat}^0}{\gls{Krad}}}\right)
\end{equation}

Equation \eqref{eq:P_c_nb} is now independent of spoke tension. The term in the parenthesis is always less than one, reflecting the fact that the wheel strength is reduced by spoke buckling. Despite the extreme simplicity of the model and the many assumptions employed, \eqref{eq:P_c_nb} gives a reasonable estimate of the strength of the seven types of bicycle wheels simulated. Furthermore, all the terms in \eqref{eq:P_c_nb} can either be directly measured or calculating using the formulas developed in this thesis.

Figure \ref{fig:Pc_rad_theory_comp} compares the theoretical radial strength (Eqn. \eqref{eq:P_c_nb}) against results from non-linear finite-element simulations of the wheels described in Appendix \ref{app:wheel_library}. In each simulation the wheel is first prestressed to a prescribed spoke tension by implicit integration in ABAQUS Standard. Next, the radial displacement is ramped linearly to a prescribed value which is beyond the peak load of the wheel. This segment is solved by explicit integration in ABAQUS Explicit. The error bars in Fig. \ref{fig:Pc_rad_theory_comp} represent the full range of peak loads from ABAQUS results for wheels with spoke tensions ranging from $\T=0$ to $\T=0.95T_c$. The black dashed line has a slope of one.

\subsection{Post-buckling behavior}

The single degree-of-freedom wheel model and the hinged-column model have the same critical behavior (buckling load equal to the lateral stiffness times radius). The post-critical behavior of the hinged-column model depends on the ratio of rotational stiffness to linear stiffness. The torsional spring stabilizes the post-critical path, while the linear spring destabilizes it. To extend the analogy to the wheel, I propose that the rotational spring is analogous to the rim stiffness, while the linear spring is analogous to the spoke stiffness. If this analogy holds, one would then expect wheels with stiffer rims (compared to the spoke stiffness) to trend towards stable post-buckling behavior.

From an analogy of the beam on elastic foundation model, a measure of the relative lateral stiffness of the spokes to the rim is
\begin{equation}
\lr^{lat} = \frac{\kuu \R^4}{\left(\frac{\EIl\,\GJ}{\EIl+\GJ}\right)}
\end{equation}

The combination $\EIl\,\GJ/(\EIl+\GJ)$ arises because the bending and torsion stiffnesses effectively act like springs in series. Higher $\lr^{lat}$ means high spoke stiffness (or low rim stiffness). Figure \ref{fig:post_buckling_comparison} shows the load-displacement curves for the seven wheels in this study, all tensioned to $\T=0.6\T_c$, ranked in order of increasing $\lr^{lat}$.

A clear but non-monotonic trend emerges in which the post-peak load drop is less severe for wheels with lower $\lr^{lat}$. One possible consequence of this trend is that wheels with higher relative spoke stiffness may be more susceptible to imperfections (e.g. a broken spoke) than wheels with stiff rims.

\begin{figure}[t]
\centering
\includesvg{\rootdir/figs/buckling_ext_loads/}{post_buckling_comparison}
\caption{Radial load-displacement curves from ABAQUS for the seven wheels in this study, ranked in order of increasing $\lr^{lat}$. Values of $\lr^{lat}$ calculated at $\T=0.6\T_c$ are shown in parentheses.}
\label{fig:post_buckling_comparison}
\end{figure}

\section{Radial buckling experiments}

Mechanical engineering undergraduate students at Northwestern complete a capstone design project as a degree requirement. Each capstone team, consisting of 5-7 students, works with a sponsoring client over the course of 20 weeks to design and implement a solution to a relevant problem. Project sponsors may come from academia, industry, or the non-profit sector. Teams are given broad latitude in the design of their project and work closely with their client to develop a rigorous set of product needs and metrics.

During the Winter/Spring 2018 quarters I sponsored a capstone project to design a mechanical testing machine capable of applying a combination of radial and lateral loads to a bicycle wheel. The team---Tina Dornbusch, Patrick Doyle, Duncan Lamb, Jonathan Sammon, Olivia Schneider, Spencer Simon, and Emma Wilgenbusch\footnote{Advised by Professor Michael Beltran, Professor J. Alex Birdwell, Ellen Owens, and me.}---designed a load frame which integrates into an existing United SFM-50KN Universal tension/compression testing machine.

The Northwestern (NU) capstone project followed a similar capstone project at Northeastern University (NEU) by Joseph Alim, Mehdi Lamnyi, Alexandra Koukhtieva, and Simon Tebbe sponsored by Professor Jim Papadopoulos \cite{Alim2016}. The NU project built on some of the successes of the NEU project---particularly the adjustable hub and LVDT radial displacement measurement---and addressed some of its shortcomings such as manual operation, lateral friction, and non-integrated measurement.

\subsection{Design of the wheel testing machine}

\begin{figure}
\centering
\includesvg{\rootdir/figs/buckling_ext_loads/}{NU_Machine}
\caption{NU Bicycle wheel testing machine. The pulleys on the left are for lateral load application. A weighted plate is hung from ropes routed over the pulleys and connected to the sliding plate. CAD and photo adapted from \cite{WheelCats2018}.}
\label{fig:NU_Machine}
\end{figure}

The mechanical engineering capstone design course at Northwestern is organized and taught in collaboration with the Segal Design Institute. Teams follow a rigorous design process including patent research, user observation, product specification, mock-up creation, design review, testing and validation, and prototype iteration phases. The Northwestern capstone experience is unique in its rigor: students submit 20 intermediate status reports and 6 product design and testing reports over the course of two academic quarters. A more complete description of the iteration of the project is given in \cite{WheelCats2018}.

A list of product needs developed by the team is given in Table \ref{tab:wheelcats_needs}. The team also developed a set of quantitative metrics to address these needs, described in \cite{WheelCats2018}. The testing apparatus is capable of performing a radial-displacement-controlled test on a bicycle wheel, with an optional lateral dead load, while simultaneously measuring radial loads (up to maximum rated load of \SI{12}{\kilo\newton}), and radial and lateral displacement of the load point relative to the hub (with a resolution of $<$\SI{10}{\micro\meter} and $<$\SI{100}{\micro\meter} respectively). The apparatus is designed to constrain the position and rotation of the hub, but allow unconstrained lateral motion of the load point.

\begin{table}
\caption{Product needs developed by the team. Adapted from \cite{WheelCats2018}.}
\label{tab:wheelcats_needs}
\begin{tabularx}{\linewidth}{r|Xc}
\hline
\bf{\#}& \bf{Need}& \bf{Importance}\\
\hline
\multicolumn{3}{c}{\emph{\textbf{Performance Requirements}}}\\
\hline
1 & Is capable of bringing a wheel to the "tacoed" buckling mode& 5\\
2 & Provides contact points that accurately simulate real-life boundary conditions of bike wheel& 5\\
3 & Can perform a radial displacement-controlled test& 5\\
4 & Can apply fixed lateral dead load to wheel& 4\\
5 & Is able to position and hold wheel in desired/specified orientation about axle& 4\\
6 & Is able to carry out automated testing programs& 3\\
7 & Can apply load to a wheel with or without a tire& 2\\
8 & Measures and records radial load while testing& 5\\
9 & Measures and records radial displacement while testing& 5\\
10& Measures and records lateral displacement while testing& 4\\
11& Facilitates semi-automated or automated live data collection& 4\\
12& Maintains full function after experiencing maximum loads& 5\\
13& Survives repeated use over an extended lifetime& 3\\
14& Accommodates a range of wheel diameters and widths& 3\\
15& Has significantly higher stiffness than test specimens& 5\\
\hline
\multicolumn{3}{c}{\emph{\textbf{Standards and Compliance}}}\\
\hline
16& Is able to interface with a variety of MTS machines with minimal reconfiguration/adjustment& 4\\
\hline
\multicolumn{3}{c}{\emph{\textbf{Manufacturing}}}\\
\hline
17& Can be manufactured in a university environment& 4\\
18& Can be manufactured within budget& 5\\
\hline
\multicolumn{3}{c}{\emph{\textbf{Assembly and Serviceability Considerations}}}\\
\hline
19& Can be moved/transported with two to three people& 4\\
20& Can be easily and accurately aligned for repeatable load application& 4\\
21& Can be maintained/repaired with common tools& 3\\
22& Allows easy access for maintenance and replacement of worn parts& 4\\
\hline
\multicolumn{3}{c}{\emph{\textbf{Assembly and Serviceability Considerations}}}\\
\hline
23& Maintains safety of observers, operators, and surroundings& 5\\
24& Prevents damage to MTS (both machine and sensors) when testing wheel& 5\\
\hline
\end{tabularx}
\end{table}

The testing apparatus, shown in Fig. \ref{fig:NU_Machine}, comprises the following subsystems: The \textbf{upper mechanical system} supports the hub, guides it along a vertical path, and interfaces with the MTS crosshead. The \textbf{lower mechanical system} contacts the rim, allows the application of lateral load, and allows lateral motion on precision linear bearings. The \textbf{safety subsystem} protects the operator and control system from projectiles\footnote{The importance of this subsystem was demonstrated during a test of a 20" wheel when multiple spoke nipples ruptured simultaneously, ejecting the spokes at high velocity like arrows from a bow.}. Finally the \textbf{control and instrumentation subsystem} measures displacements and loads in real time and controls the vertical motion of the MTS crosshead based on a preset control program.

\subsubsection*{Upper mechanical system}
The NEU machined allowed the hub to slide laterally while holding the load point fixed at the top of the load frame. During testing it was discovered that the load point would slip under relatively small lateral loads, potentially applying a bending moment to the load cell. The NU team chose to constrain the hub laterally, but allow vertical motion along linear bearings. The upper mechanical system comprises an external frame for lateral stiffness, vertical linear bearing rails, and a U-shaped carriage to guide the hub motion. Since friction in these rails would affect the measurement of radial load, the team measured the friction under a variety of lateral loads and determined that it would not affect the radial load measurement by more than \SI{50}{\newton} (\SI{11.24}{lb}) under a worst-case loading scenario.

\subsubsection*{Lower mechanical system}
Previously reported radial loading experiments on bicycle wheels constrained the lateral motion of the contact point \cite{Burgoyne1993,RARblog,McLundie2007}. This constraint artificially increases the radial stiffness and supresses the lateral buckling mode of interest. The lower mechanical system allows both rotation of the rim cross-section and lateral displacement of the contact point. The bare rim is supported by a smooth steel pin which is sized to be lightly press-fit between the rim flanges. The pin is supported below between adjustable V-jaws (Fig. \ref{fig:rad_buckling_NU_lateral_friction} (a)). The team measured a friction coefficient between the pin and jaws of less than 0.2, which they determined would have no appreciable effect on the test. A wheel with a tire can be tested by removing the pin and V-jaw assembly and placing the tire directly on the sliding plate.

The V-jaws are mounted on a steel plate riding on Thomson Linear SuperSmart pillow block bearings. The linear bearings run on fully-supported steel rails to minimize distortion under load. Lateral friction increases the apparent buckling load. The effect of lateral friction on the buckling load of a fixed-free column has been studied by Lazopoulos \cite{Lazopoulos1991}, who found a significant effect even at very small friction coefficients. Analogously modeling the bicycle wheel buckling problem as a straight column, Lazopoulos's model predicts that a friction coefficient of 0.01 would result in a \SI{10}{\percent} overprediction of the critical load\footnote{This is likely conservative due to the presence of additional lateral compliance in the load frame.}. The linear bearings used have a load-dependent friction coefficient, decreasing with increasing vertical load. The team experimentally measured the friction coefficient, achieving a minimum coefficient of 0.01 at a load of \SI{2.8}{\kilo\newton} (Fig. \ref{fig:rad_buckling_NU_lateral_friction} (b)). At higher loads, an even smaller friction coefficient is expected.

\begin{figure}[t]
\centering
\includesvg{\rootdir/figs/buckling_ext_loads/}{rad_buckling_NU_lateral_friction}
\caption{\textbf{(a)} Lateral carriage assembly without a bicycle wheel mounted. \textbf{(b)} Friction coefficient of the lateral bearings as a function of vertical load. The error bars represent the \SI{95}{\percent} confidence interval from 4 tests.}
\label{fig:rad_buckling_NU_lateral_friction}
\end{figure}

\subsubsection*{Control and instrumentation subsystem}
The control and instrumentation subsystem measures the radial displacement, lateral displacement, and radial load in real time and controls the position of the MTS crosshead. The vertical displacement of the U-carriage is measured relative to the load platten with a DC linear variable differential transformer (Schaevitz Sensors GPD-121-250 LVDT). The lateral displacement is measured by a string potentiometer (TE Connectivity SP1-12) connected to the lateral carriage. Optionally, the change in tension of a single spoke may be measured using a standard knife-edge extensometer with a 2-inch gauge length.

All sensors are read by the 24-bit analog-to-digital converter provided with the United Testing control hardware. The control software is capable of running preset testing routines with logic driven by sensor inputs (e.g. a program could be designed to load to \SI{100}{lb}, then unload to \SI{15}{lb}, then load until failure or a preset displacement).

\subsection{Experimental procedure}

Three bicycle wheels without tires were tested until lateral buckling failure. All three wheels were built from identical components to the same nominal specifications\footnote{See Appendix \ref{app:std_research_wheel} for complete wheel properties}. The maximum and minimum spoke tensions were kept to within \SI{+-10}{\percent} of the mean tension. No special effort was made to make the wheels laterally or radially true, however some adjustment was made to Wheel 2 to remove an excessive lateral wobble without sacrificing tension uniformity. The spokes were stress-relieved prior to testing by the method described by Brandt \cite{Brandt1993}.

The wheel was loaded directly at a spoke position. The wheel was first loaded to 100 lbs radial load to force the load pin to settle into the rim. Prior experiments showed that this load was sufficient to cause the pin to settle, but not sufficient to plastically deform the rim or spokes. Next, a small radial preload was put on the wheel and a lateral dead load was applied using a pulley and hanging weights connected to the lateral carriage. The dead load was applied in the direction opposite the side of the loaded spoke (i.e. if the wheel was loaded at a spoke connected to the ``left'' hub flange, the lateral load was acting to the right). Previous experiments had shown that the wheel naturally buckles in this direction without a bias load. The MTS crosshead was then moved downward at a fixed velocity until the radial load dropped below \SI{90}{\percent} of the peak load. The wheel was then unloaded until the radial load dropped below \SI{10}{lb}. After testing, each spoke tension was measured and the deformed lateral shape was measured using a dial indicator mounted to a bicycle wheel truing stand.

\subsection{Results}

Load-displacement and displacement-displacement curves for the three tests are shown in Fig. \ref{fig:rad_buckling_NU_experiments}. The load-displacement curves exhibit the same three failure points as the ABAQUS simulations: (1) non-linearity caused by spoke buckling, (2) rim buckling associated with a sharp increase in lateral displacement, and (3) unstable rim collapse. In every case, the rim buckled in the direction of the applied lateral load opposite the loaded spoke.

\begin{table}
\caption{Test parameters, selected results, and post-mortem properties.}
\label{tab:rad_buckling_tests}
\begin{tabular}{llccc}
\hline
&& \bf Wheel 1 & \bf Wheel 2 & \bf Wheel 3\\
\hline
\multicolumn{2}{l}{Lateral dead load [\si{N}]} & \num{58} & \num{58} & \num{220}\\
\multicolumn{2}{l}{Peak radial load [\si{kN}]} & \num{4.49} & \num{4.39} & \num{3.51}\\
\multicolumn{2}{l}{Radial stiffness [\si{kN/mm}]} & \num{1.944+-0.006} & \num{1.999+-0.014} & \num{2.003+-0.005}\\
\multicolumn{2}{l}{Onset of non-linearity (5\%) [\si{kN}]} & \num{2.79} & \num{1.85} & \num{2.14}\\
\multicolumn{5}{l}{Before test}\\
\,& Spoke tension [\si{N}]      & \num{852+-21} & \num{862+-52} & \num{867+-33}\\
\,& Lateral tolerance [\si{mm}] & \num{+-0.5} & \num{+-2.5} & \num{+-1}\\

\multicolumn{5}{l}{After test}\\
\,& Spoke tension [\si{N}]      & 638 (1475, 0)$^{\rm a}$ & 694 (1409, 84) & 722 (1150, 195)\\
\,& Lateral tolerance [\si{mm}] & \num{+-6.8} & \num{+-5.7} & \num{+-4.2}\\
\hline
$^{\rm a}$(max., min.) tension\\
\hline
\end{tabular}
\end{table}

\begin{figure}
\centering
\includesvg{\rootdir/figs/buckling_ext_loads/}{rad_buckling_NU_experiments}
\caption{Radial buckling experimental results. \textbf{(a)} Load-displacement curves. \textbf{(b)} Displacement diagrams. \textbf{(c)-(e)} Post-mortem inspection of tested wheels showing change in spoke tension (bars) and lateral deformation (line). Blue bars represent the spokes on the same side as the lateral load (and buckling direction), while orange bars represent spokes on the opposite side (see cartoon inset in (c). The spoke at the load is shown in red.}
\label{fig:rad_buckling_NU_experiments}
\end{figure}

The lateral dead load significantly decreased the peak load and radial displacement to rim collapse: a +\SI{162}{N} (\SI{36.4}{lb}) increase in lateral load decreased the peak radial load by \SI{1000}{N} (\SI{220}{lb}). There is no obvious trend in spoke buckling load. Radial displacement has a much larger effect on spoke tensions, per unit displacement, than lateral loads due to the orientation of the spokes.

The residual deformed shape (black lines in Fig. \ref{fig:rad_buckling_NU_experiments} (c)-(e)) is that of an asymmetric ``taco''. The average spoke tension has decreased, likely both due to radial yielding\footnote{by radial yielding I mean yielding of the rim by in-plane bending.} of the rim and yielding of some spokes. The three spokes near the load point (red bar and adjacent blue bars) for wheels 1 and 2 have lost most or all of their tension due to radial yielding. Away from the load point the pattern in residual spoke tensions follows the lateral displacement: at each ``wave'' the tensions are higher on the side opposite the direction of rim displacement. The influence of the radial deformation has completely disappeared at this distance. A wheel damaged in such a way is difficult or impossible to true by adjusting spoke tensions alone. The necessary adjustments to true the rim will exacerbate the difference in spoke tensions.

Wheel 3 shows less evidence of radial yielding of the rim than wheels 1 and 2: the spoke tension directly above the load maintained its original tension (the increase due to lateral rim motion is balanced by the decrease due to some possible radial yielding). No spokes in wheel 3 have gone completely slack.

All three tests were performed on wheels without tires in order to isolate the structural behavior of the wheel. The pin, set inside the rim cross-section and oriented with its axis in the plane of the wheel, effectively applied a point load to the rim. The localized loading is at least partially responsible for the localized yielding observed in these wheels. A tire would spread out the radial load to more spokes and possibly delay the onset of nonlinearity. However, since rim buckling and collapse are both global phenomena controlled more by the lateral stiffness of the rim, I speculate that a tire would not appreciably affect the peak load or collapse load.

\subsection{Comparison with theory}
In order to compare the experimentally-determined radial strength with Eqn. \eqref{eq:P_c_nb}, it is necessary to determine the rim radius $\R$, the effective spoke axial stiffness $\gls{Espk}\gls{As}$, the lateral stiffness at zero spoke tension $K_{lat}^0$, the radial stiffness \gls{Krad}, and the critical buckling tension $\T_c$. The effective axial stiffness of the butted spokes is calculated by modeling the spoke as a bar with three distinct cross-sections along its length, as described in Appendix \ref{app:std_research_wheel}. The remaining parameters are estimated from theory: the lateral and radial stiffnesses are calculated using the mode matrix method, Eqn. \eqref{eq:mm_Kd_f}, with 36 included modes and the smeared-spokes approximation (equivalent to Eqns. \eqref{eq:Klat_series} and \eqref{eq:Krad}). The buckling tension $\T_c$ is estimated using Eqn. \eqref{eq:Tcn_equiv_springs}.

Substituting these values into Eqn. \eqref{eq:P_c_nb} gives a predicted strength of \SI{4.87}{kN}, compared with the average strength of wheels 1 and 2, \SI{4.44}{kN}, a difference of less than \SI{10}{\percent}. As noted earlier, the experiment may slightly over-predict the true radial strength due to friction in the lateral carriage bearings. Evidently, the competing failure model gives a reasonable prediction for the strength in absence of lateral loads for a wheel with properties typical of those in wide use today.

\section{Concluding remarks}

The problem of buckling under external loads is significantly more complex than the problem of buckling under excessive spoke tension. Failure under external loads is controlled to some extent by a balance of two competing failure modes: buckling of the spokes (localized), and buckling of the rim (global). For lateral loads, an approximate treatment of this competition yields a rule-of-thumb of $\T_{opt} = 0.5\T_c$ for maximizing the lateral load the wheel can withstand without spokes buckling. For radial loads, the competing mode model gives a prediction of the buckling load which matches well with simulations and experiments.

The critical tension $\T_c$ plays an important role in wheel failure. Although the buckling tension for a wheel with a modern double-wall rim is generally too high to approach in practice, the effect of tension on lateral stiffness controls the failure under external loads. The critical tension is an important metric to be optimized through design even if the maximum tension will never be exceeded. The competing failure modes model assumes that the peak load is not sensitive to tension. This assumption has been validated against computational simulations but should also be validated experimentally. It is likely that the tension will have a strong effect on the dominant failure mode. A wheel built with low tension will likely experience significant rim yielding prior to lateral buckling, a failure mode which was ignored in this analysis.

It's difficult to imagine a realistic scenario in which a wheel would collapse under a purely radial load since any misalignment between the load and the plane of the wheel produces a lateral force. The testing apparatus built at Northwestern University has the ability to apply combined loads. The experiments described in this chapter showed a significant effect of lateral load: an increase in lateral force of \SI{162}{N} resulted in a reduction in radial strength of almost \SI{1}{kN}. With a few more tests, a failure diagram could be constructed which would define a safe region in lateral force---radial force space. Such a diagram for spoke buckling load rather than peak load could already be constructed with the linear theory described in Chapter \ref{chap:stress_analysis}.

Wheel failure may also depend on the details of the rim cross-section. It has been assumed here that the bending and torsion stiffness of the rim are sufficient for predicting its structural behavior. However, two rims with identical stiffness may vary considerably in their resistance to plastic yielding. A buckled wheel may yield under a combination of bending and torsional strain and the relative importance of these two modes may have implications for the ability to easily repair such a rim \cite{Wilson2004}.

\end{document}
