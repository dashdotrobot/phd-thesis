%!TEX root = ../../thesis.tex
\providecommand{\rootdir}{../..}
\documentclass[../../thesis.tex]{subfiles}

\begin{document}

The most general practical special case is that of the asymmetric wheel with tangential spoking and an arbitrary spoke offset vector. There are four spoke types: left-side leading, left-side trailing, right-side leading, and right-side spoke. The four independent spoke vectors are:
\begin{subequations}
\begin{align}
\n^l &= c_1^l \eo + c_2^l \et \pm c_3^l \eh\\
\n^r &= -c_1^r \eo + c_2^r \et \pm c_3^r \eh
\end{align}
\end{subequations}

The offset vector is assumed to be symmetric across the rim.
\begin{subequations}
\begin{align}
\bs^l &= b_1 \eo + b_2 \et\\
\bs^r &= -b_1 \eo + b_2 \et
\end{align}
\end{subequations}

The initial tension must be different on the two sides so that the lateral components balance. The left and right tensions are:
\begin{subequations}
\begin{align}
\T_p^l &= \frac{4\pi \R \Tb c_1^r}{\gls{ns}(c_1^lc_2^r + c_1^rc_2^l)}\\
\T_p^r &= \frac{4\pi \R \Tb c_1^l}{\gls{ns}(c_1^lc_2^r + c_1^rc_2^l)}
\end{align}
\end{subequations}
in terms of the average radial tension per unit circumference, $\Tb$. One further simplification is made by setting $\gls{ls}^l=\gls{ls}^r$. The spoke length generally differs by less than \SI{1}{\percent}. The smeared-spokes stiffness matrix is given below.

% \scriptsize
\footnotesize
% Assumption: Spoke lengths are approximately on left and right sides
\begin{multline}
\label{eqn:kbar_asymm_ncross}
\gls{kbar} = \frac{\gls{ns}\gls{Espk}\gls{As}}{4\pi \R \gls{ls}}
\begin{bmatrix}
(c_1^l)^2 + (c_1^r)^2   & c_1^lc_2^l - c_1^rc_2^r & 0 & b_1(c_1^lc_2^l + c_1^rc_2^r) - b_2\left[(c_1^l)^2 + (c_1^r)^2\right]\\
                        & (c_2^l)^2 + (c_2^r)^2   & 0 & b_2(c_1^rc_2^r - c_1^lc_2^l) - b_2\left[(c_2^r)^2 - c_2^l)^2\right]\\
                        &                         & (c_3^l)^2 + (c_3^r)^2 & 0\\
& & & (b_1c_2^l - b_2c_1^l)^2 + (b_1c_2^r - b_2c_1^r)^2
\end{bmatrix}\\
+ \frac{\Tb}{l_s(c_1^lc_2^r + c_1^rc_2^l)}
\begin{bmatrix}
   (c_1^l + c_1^r)(1 - c_1^lc_1^r)             & 0 & 0 & 0\\
&  c_1^l(1 - (c_2^r)^2) + c_1^r(1 - (c_2^l)^2) & 0 & 0\\
&& c_1^l(1-(c_3^r)^2) + c_1^r(1-(c_3^l)^2)     & 0\\
&&& 0
\end{bmatrix}\\
+ \frac{\Tb}{\gls{ls}(c_1^lc_2^r + c_1^rc_2^l)}
\begin{bmatrix}
0 & c_1^lc_1^r(c_2^r-c_2^l) & 0 & -b_1c_1^lc_1^r(c_2^l + c_2^r) - b_2\left[c_1^l(1 - (c_1^r)^2) + c_1^r(1- (c_1^l)^2)\right]\\
  & 0           & 0 &  b_2c_1^lc_1^r(c_2^l - c_2^r) + b_1\left[c_1^r(1 - (c_2^l)^2) - c_1^l(1-(c_2^r)^2)\right]\\
  &   & 0 & 0\\
  &   &   & b_1^2\left[c_1^l(1-(c_2^r)^2) + c_1^r(1-(c_2^l)^2)\right] + b_2^2\left[c_1^l(1-(c_1^r)^2) + c_1^r(1-(c_1^l)^2)\right]\\
  &   &   & + 2b_1b_2c_1^lc_1^r(c_2^l + c_2^r)
\end{bmatrix}
\end{multline}
\normalsize

\end{document}
