%!TEX root = ../../thesis.tex
\providecommand{\rootdir}{../..}
\documentclass[../../thesis.tex]{subfiles}

\begin{document}

\section{Standard research wheel}
\label{app:std_research_wheel}

Many of the experiments and theoretical calculations described in this thesis use a wheel with standardized components and geometry for ease of comparison. The ``standard research wheel'' comprises a Sun-Ringle CR18-700C 36-hole rim laced to a custom hub with double-butted Wheelsmith DB14 \SI{276}{mm} spokes. Unless otherwise noted, the spokes are all oriented with the elbows on the inside of the hub flange.

\begin{figure}[h]
\centering
\includesvg{\rootdir/figs/buckling_ext_loads/}{shear_center_estimation}
\caption{\textbf{(a)} Diameter measurement of the Sun-Ringle CR18 rim. \textbf{(b)} Centroid and shear center of a thin-walled C-channel beam \cite{Timoshenko1961}. \textbf{(c)} Assumed location of the shear center of the CR18 rim cross-section.}
\label{fig:shear_center_estimation}
\end{figure}

\subsubsection*{Rim}

The Sun-Ringle CR18-700C aluminum rim has a shallow double-wall construction whose shape approximates a C-channel beam. A C-channel beam with a fully-open cross-section has a shear center outside the cross-section below the web \cite{Timoshenko1961} (Fig. \ref{fig:shear_center_estimation} (b)). The behavior of the CR18 rim should be somewhere between that of a fully-closed, symmetic cross-section and that of a fully-open C cross-section. Therefore I measure the radius at the ``bottom'' of the cross-section (Fig. \ref{fig:shear_center_estimation} (c)). By this method, the effective radius is \SI{304}{mm}. The cross-section stiffness parameters are estimated using the acoustic method described in Chapter \ref{chap:acoustic_testing}.

\begin{table}
\caption{Properties of the standard research wheel.}
\label{tab:standard_wheels}
\begin{tabular}{llcc}
\hline
&& \textbf{Standard research wheel} & \textbf{Fat bike wheel}\\
\hline
\multicolumn{2}{l}{Rim}& Sun-Ringle CR18-700C\\
\,& $\R$ [\si{mm}]         & \num{304} & 300\\
\,& \gls{mrim} [\si{g}]    & \num{538}\\
\,& \gls{rimdens} [\si{kg/m^3}] & \num{2700}\\
\,& $E$ [\si{N/m^2}]       & \num{69}        & \num{69}\\
\,& $G$ [\si{N/m^2}]       & \num{26}        & \num{26}\\
\,& \gls{Arim} [\si{mm^2}] & \num{104}       & \num{170}\\
\,& $\EIr$ [\si{N.m^2}]    & \num{111+-9}    & \num{72}\\
\,& $\EIl$ [\si{N.m^2}]    & \num{219+-40}   & \num{6995}\\
\,& $\GJ$   [\si{N.m^2}]   & \num{26.3+-1.2} & \num{98}\\
\multicolumn{2}{l}{Hub} & custom adjustable hub\\
\,& width [\si{mm}]              & 50 & 50\\
\,& effective width (elbows in)  & 53\\
\,& effective width (elbows out) & 48\\
\,& diameter [\si{mm}]           & 58 & 50\\
\multicolumn{2}{l}{Spokes} & Wheelsmith DB14 \SI{276}{mm}\\
\,& number                 & 36      & 36\\
\,& arrangement            & radial  & radial\\
\,& diameter [\si{mm}]     & 1.7/2.0 & 2.0\\
\,& effective diameter [\si{mm}]     & 1.78\\
\,& $EA_{eff}$ [\si{kN}]   & 522     & 628\\
\hline
\end{tabular}
\end{table}

\subsubsection*{Hub}
The custom research hub used in this thesis was designed by Joseph Alim, Mehdi Lamnyi, Alexandra Koukhtieva, and Simon Tebbe for an undergraduate capstone project sponsored by Jim Papadopoulos \cite{Alim2016}. The hub comprises a solid chromoly steel \sfrac{3}{4}-inch diameter ``axle'' with oppositely threaded ends, and two mild steel hub flanges threaded onto the axle and secured with jam nuts. With this turnbuckle design, the hub flange spacing can be changed by simply rotating the axle relative to the flanges. The research hub does not have a bearing so it cannot be rolled like a standard bicycle wheel. The purpose of the solid axle is to make the hub very stiff relative to the rim and spokes system.

\subsubsection*{Spokes}
The wheel is built with stainless-steel double-butted Wheelsmith DB14 \SI{276}{mm} J-bend spokes. The thick section at the top and bottom of the spoke has a diameter of \SI{1.99}{mm} while the thin swaged middle section has an average diameter \SI{1.72}{mm} and a length of \SI{205}{mm}. The effective cross-sectional area is calculated using the series-springs rule for the thin and thick sections:
\begin{equation}
\frac{l_s}{EA_{eff}} = \frac{l_1}{EA_1} + \frac{l_2}{EA_2}
\end{equation}

By this method, the effective spoke diameter is \SI{1.78}{mm}. For stainless steel ($E = \SI{210}{GPa}$), this gives an effective axial stiffness of $EA_{eff} = \SI{521}{kN}$.

The complete wheel properties are given in Table \ref{tab:standard_wheels}.


\section{Fat bike wheel}
\label{app:fat_bike_wheel}

The example fat bike rim is modeled as a prismatic aluminum ($\E=\SI{69}{GPa}$, $\G=\SI{26}{GPa}$) beam with a hollow, thin-walled, rectangular cross-section (width: \SI{80}{mm}, height: \SI{5}{mm}, wall thickness: \SI{1}{mm}). The relevant properties are given in Table \ref{tab:standard_wheels}.


\section{Example wheel library}
\label{app:wheel_library}

Several example wheels are defined for the purposes of illustrative calculations and simulations. The ``high wheel'' represents the kind of wheel that would be found on an Ordinary, or ``penny-farthing,'' bicycle popular in the 1880s. The ``small'' wheel might be found on an adult folding bicycle or a youth bicycle. All of the properties are hypothetical and are not meant to correspond to any particular wheel or product.

All the wheels have aluminum rims ($\E=\SI{69}{GPa}$, $\G=\SI{26}{GPa}$, $\gls{rimdens}=\SI{2700}{kg/m^3}$) except for the high wheel rim, which is constructed of steel ($\E=\SI{200}{GPa}$, $G=\SI{80}{GPa}$, $\gls{rimdens}=\SI{8000}{kg/m^3}$). The cross-sectional area of all rims is \SI{100}{mm^2}. The remaining properties are given in Table \ref{tab:example_wheel_props}.

\begin{table}
\caption{Properties of the wheels in this study. Linear dimensions are given in millimeters. $\EIr$, $\EIl$, and $\GJ$ are given in \si{N.m^2}.}
\label{tab:example_wheel_props}
\begin{tabular}{l|cccc|cc|cc}
\hline
&\multicolumn{4}{c|}{Rim} & \multicolumn{2}{c|}{Hub} & \multicolumn{2}{c}{Spokes}\\
\bf{Wheel} & $\R$ & $\EIr$ & $\EIl$ & $\GJ$ & Diam. & Width & Number & Diam.\\
\hline
High wheel   & 600 & 300 & 300 & 100 & 50 & 90 & 64 & 2.5\\
Vintage road & 315 & 150 & 150 & 15  & 50 & 50 & 36 & 1.8\\
Modern road  & 300 & 200 & 200 & 80  & 50 & 50 & 24 & 1.8\\
Cheap MTB    & 280 & 200 & 200 & 12  & 50 & 60 & 32 & 2.0\\
Tandem       & 300 & 200 & 200 & 80  & 50 & 70 & 40 & 2.0\\
Small        & 225 & 150 & 150 & 40  & 50 & 50 & 32 & 1.8\\
Track        & 300 & 200 & 200 & 150 & 50 & 60 & 32 & 1.8\\
\hline
\end{tabular}
\end{table}

\end{document}
