%!TEX root = ../../thesis.tex
\providecommand{\rootdir}{../..}
\documentclass[../../thesis.tex]{subfiles}

\begin{document}



\section{Frequency response of smartphone microphone}
\label{app:mic_cal}

\begin{figure}[h]
\centering
\includesvg{\rootdir/figs/acoustic_testing/}{mic_sensitivity}
\caption{Frequency response of the iPhone SE built-in microphone used in this study.}
\label{fig:mic_sensitivity}
\end{figure}

The frequency response of the built-in microphone in an Apple iPhone SE (model A1662) was measured in an anechoic chamber. The smartphone and a calibrated reference microphone with a flat frequency response (Etymotic Research ER-7C Probe Mic System) were placed on a foam block 66 inches from a single mono speaker (Roland MA-12C Micro Monitor). Approximated pink noise was generated from an online source (https://mynoise.net/NoiseMachines/whiteNoiseGenerator.php) and played through the speaker. A spectral average was obtained from both microphones using the Fast Fourier Transform with a buffer size of 8192 samples at \SI{44.1}{kHz} sample rate with 50 averaging windows and discarding the phase. The iPhone microphone relative sensitivity was calculated by taking the ratio of the iPhone spectrum to the reference spectrum and normalizing by the amplitude at \SI{5.38}{Hz}.

\section{Peak identification procedure}
\label{app:peak_fits}

The following procedure was used to identify the radial and lateral mode frequencies for each rim: First, the two spectra were compared with the noise spectrum to identify any peaks with a signal-to-noise ratio of at least 10 (note the first two apparent peaks at \SI{27}{Hz} and \SI{60}{Hz} are both present in the noise spectrum and can therefore be discarded). Next, the lowest peaks were compared between the lateral and radial spectra to find duplicates. In the case of duplicates, the peak was assigned to the spectrum with the greater relative magnitude.

After identifying the approximate location of each peak, the precise peak parameters were determined by fitting a Lorentzian peak function of the form:
\begin{equation}
\label{eq:ema_peak_fit}
F(f) = \left(\frac{1}{2\pi}\right) \frac{\Gamma}{(f-f_0)^2 + \Gamma^2}
\end{equation}

Fitted curves for $f_2^{rad}$, $f_3^{rad}$, $f_2^{lat}$, $f_3^{lat}$, and $f_4^{lat}$ are shown below:

\begin{figure}
\centering
\includesvg{\rootdir/figs/acoustic_testing/}{fit_all_peaks}
\caption{Fitted peaks in frequency spectra}
\label{fig:ema_all_peaks}
\end{figure}

\end{document}
