%!TEX root = ../../thesis.tex
\providecommand{\rootdir}{../..}
\documentclass[../../thesis.tex]{subfiles}

\begin{document}

In the four-point bending test, the rim is supported at 3 and 9 o'clock and loaded at 12 and 6 o'clock with an out-of-plane force $P$. A dummy torque $Q$ is applied at each point in the same sense as the rotation of the cross-section. Free-body diagrams of the complete rim and upper section are shown below:

\begin{figure}[h]
\centering
\includesvg{\rootdir/figs/acoustic_testing/}{four_pt_analysis}
\caption[Schematic of rim four-point bending test]{\textbf{(a)} Force diagram for the four-point bend test. \textbf{(b)} Free-body diagram of the top half of the rim. \textbf{(c)} Free-body diagram of an arbitrary section of rim.}
\label{fig:four_pt_bend_sections}
\end{figure}

The internal shear, lateral bending moment, and twisting moment are $F_1$, $M_2$, and $M_3$, respectively. The symmetry of the problem gives us the conditions
\begin{subequations}
\begin{gather}
F_1^a = -F_1^b\\
M_2^a = M_2^b\\
M_3^a = M_3^b
\end{gather}
\end{subequations}

Equilibrium of forces and moments gives
\begin{subequations}
\begin{gather}
F_1^a = -F_1^b = \frac{P}{2}\\
M_2^a = M_2^b = \frac{P\R}{2}+\frac{Q}{2}\\
M_3^a = M_3^b = 0
\end{gather}
\end{subequations}

The internal forces can now be determined by making a cut at an arbitrary location $\gls{ang}$, as shown in Fig. \ref{fig:four_pt_bend_sections} (c). Equilibrium of forces and moments gives
\begin{subequations}
\label{eq:eq_section}
\begin{gather}
F_1 = \frac{P}{2}\\
M_2\cos{\gls{ang}} - M_3\sin⁡{\gls{ang}} = \frac{P\R}{2}(1 -\sin{\gls{ang}}) + \frac{Q}{2}\\
M_2\sin{\gls{ang}} + M_3\cos{\gls{ang}} = \frac{P\R}{2}(\cos⁡{\gls{ang}} - 1)
\end{gather}
\end{subequations}

Solving \eqref{eq:eq_section} for $M_2$ and $M_3$ gives
\begin{subequations}
\begin{gather}
M_2 = P\R(\cos⁡{\gls{ang}} - \sin⁡{\gls{ang}}) + \frac{Q}{2}\cos{\gls{ang}}\\
M_3 = -P\R(\sin⁡{\gls{ang}} + \cos{\gls{ang}}) + \frac{P\R}{2} - \frac{Q}{2}\sin⁡{\gls{ang}}
\end{gather}
\end{subequations}

The strain energy in the upper half of the rim is given by
\begin{equation}
U = 2\int_0^{\pi/2} \left( \frac{M_2^2}{2\EIl} +\frac{M_3^2}{2\GJ} \right) R d\gls{ang}
\end{equation}

The displacement and rotation at the load point is determined using Castigliano's theorem: $u_0 = \frac{\partial U}{\partial P}$, $\p_0 = \frac{\partial U}{\partial Q}$. This is the ``balanced'' deflection, i.e. the vertical deflections at each load point assuming that the slope $du/d\gls{ang}$ is zero at the supports. In the un-balanced four-point bending test (three points are constrained and the third is loaded), the displacement will be $u_l=2u_0$.
\begin{gather}
\begin{align}
u_l  &= -\frac{P\R^3}{2\GJ}[2(3-\pi) + \m(2-\pi)]\\
\p_l &= -\frac{P\R^2}{8\GJ}(1+\m)(2-\pi)
\end{align}
\end{gather}

\end{document}
