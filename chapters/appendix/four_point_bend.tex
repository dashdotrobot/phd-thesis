%!TEX root = ../../thesis.tex
\providecommand{\rootdir}{../..}
\documentclass[../../thesis.tex]{subfiles}

\begin{document}

In the four-point bending test, the rim is supported at 3- and 9-o'clock and loaded at 12- and 6-o'clock with an out-of-plane force $P$. A dummy torque $Q$ is applied at each point in the same sense as the rotation of the cross-section. Free-body diagrams of the complete rim and upper section are shown below:

\begin{figure}[h]
\centering
% \includesvg{\rootdir/figs/acoustic_testing/}{4pt_bend_FBD}
\caption{...}
\label{fig:four_pt_bend_sections}
\end{figure}

The symmetry of the problem gives us the condition
\begin{equation}M_1=M_2\end{equation}

Equilibrium of forces and moments gives
\begin{gather}
V_1=\frac{P}{2}\\
F_1=F_2=0\\
M_1=\frac{PR}{2}+\frac{Q}{2}
\end{gather}

The internal forces can now be determined by making a cut at an arbitrary location $\gls{ang}$, as shown in Fig. \ref{fig:four_pt_bend_sections} (c). Equilibrium of forces and moments gives
\begin{subequations}
\label{eq:eq_section}
\begin{gather}
V = \frac{P}{2}\\
M_1 + F \sin⁡{\gls{ang}} - M \cos{\gls{ang}} - \frac{PR}{2}\sin{\gls{ang}} = 0\\
F\cos{\gls{ang}} + M \sin{\gls{ang}} + \frac{PR}{2} (1-\cos⁡{\gls{ang}}) = 0
\end{gather}
\end{subequations}

Solving \ref{eq:eq_section}for $F$ and $M$ gives
\begin{subequations}
\begin{gather}
F = -PR(\sin⁡{\gls{ang}} + \cos{\gls{ang}}) + \frac{PR}{2} - \frac{Q}{2}\sin⁡{\gls{ang}}\\
M = PR(\cos⁡{\gls{ang}} - \sin⁡{\gls{ang}}) + \frac{Q}{2}\cos{\gls{ang}}
\end{gather}
\end{subequations}

The strain energy in the upper half of the rim is given by
\begin{equation}
U = 2\int_0^{\pi/2} \left( \frac{M^2}{2\EIl} +\frac{T^2}{2\GJ} \right) R d\gls{ang}
\end{equation}

The displacement and rotation at the load point is determined using Castigliano's theorem:
\begin{equation}
u_0 = \frac{\partial U}{\partial P}, \,\,\, \p_0 = \frac{\partial U}{\partial Q}
\end{equation}

This is the ``balanced'' deflection, i.e. the vertical deflections at each load point assuming that the slope $du/d\gls{ang}$ is zero at the supports. In the un-balanced four-point bending test (three points are constrained and the third is loaded), the displacement will be $u_l=2u_0$.
\begin{gather}
\begin{align}
u_l  &= -\frac{PR^3}{2\GJ}[2(3-\pi) + \m(2-\pi)]\\
\p_l &= -\frac{PR^2}{8\GJ}(1+\m)(2-\pi)
\end{align}
\end{gather}

\end{document}
