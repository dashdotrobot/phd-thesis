%!TEX root = ../../thesis.tex
\providecommand{\rootdir}{../..}
\documentclass[../../thesis.tex]{subfiles}

\begin{document}

I measured spoke tensions using the WheelFanatyk analog tensiometer and the calibration table supplied with the instrument. The tensiometer works on the principle that the spoke under tension behaves like a guitar string, i.e. the lateral stiffness is proportional to the tension. The spoke is loaded by a linear spring with a precisely known stiffness and the displacement is measured on the opposite side as where the load is applied. The instrument is shipped with a calibration table giving the tension for a range of displacements.

A straightforward analysis of the mechanics involved, assuming that the tension remains constant and the bending stiffness is negligible, leads to the following relationship for the tension as a function of the measured displacement:
\begin{equation}
\T = a\left(\frac{1}{\delta}\right) + b
\end{equation}

where $\T$ is the spoke tension and $\delta$ is the measured displacement measured in millimeters. The parameters $a$ and $b$ differ slightly for different spoke diameters because the preloaded spring contacts the spoke at a different point along its stroke, and thick spokes may be more affected by bending stiffness. By fitting the model above to the calibration data, I determined the following parameters for different spoke diameters:

\begin{table}[h]
\begin{tabular}{l|ccccccc}
\hline
Spoke diameter & \bf 0.9 mm & \bf 0.95 mm & \bf 1.2 mm & \bf 1.5 mm & \bf 1.7 mm & \bf 1.8 mm & \bf 2.0 mm\\
\hline
$a$ [\si{N.mm}] &  480.8 &  466.4 & 434.1 &  466.4 &  423.5 &  428.2 &  427.2\\
$b$ [\si{N}]    & -156.8 & -189.8 & -93.3 & -189.8 & -219.4 & -273.6 & -323.5\\
\hline
\end{tabular}
\end{table}

The spoke tensions reported in this thesis are calculated using the model above to effectively interpolate the values supplied by the manufacturer.

\end{document}
