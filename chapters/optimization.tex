%!TEX root = ../thesis.tex
\providecommand{\rootdir}{..}
\documentclass[\rootdir/thesis.tex]{subfiles}

\newcolumntype{L}{>{\raggedright\arraybackslash}X}

\begin{document}

A thesis entitled ``Reinventing the Wheel'' would not be complete without delving into the topic of optimization. For 150 years, bicycle designers and wheelbuilders have sought to improve the wheel across multiple performance metrics by changing dimensions, materials, and construction methods. This process occurred largely by trial-and-error, although in recent years component design has been greatly aided by finite-element simulation. My intent in this chapter is not to give an exhaustive optimization routine which can be followed to give the perfect wheel under any set of performance criteria, but rather to explore general trends which emerge when optimizing the wheel under a realistic set of constraints.

Optimization requires at minimum a space of tunable design parameters, an objective function defining the performance characteristic or characteristics to be optimized, and a model which predicts the performance characteristic as a function of the design parameters. Some design parameters are tunable by the wheelbuilder such as lacing pattern, spoke type, and component selection, while others are tunable by the component manufacturer such as rim cross-section shape, material, and hub flange spacing.

Keller tackled the problem of optimizing the spoke geometry in order to maximize a weighted sum of the lateral stiffness and torsional stiffness \cite{Keller2013}. With this objective function, most design parameters are trivial (e.g. spoke diameter will always optimize to the maximum possible value) so he limited the design space to the number of spokes on the left and right sides (with the total number fixed) and the in-plane spoke inclination angles ($\beta$ in this thesis). He used formulas for the lateral and torsional stiffness from Goldberg \cite{Goldberg1984} which are deeply flawed in several respects\footnote{First, Goldberg treats the rim as a rigid body. This is a reasonable approximation for the torsional stiffness (Goldberg's formula is approximately equivalent to Equation \ref{eq:Ktan} in this thesis), however it is grossly incorrect for the lateral stiffness. Second, even if the rigid-rim approximation is used, Goldberg's formula gives a value which is approximately 50\% higher than the correct result (Equation \ref{eq:Klat_stiff_rim} in this thesis). This is apparently due to Goldberg's assumption that the rim translates as a rigid body in the lateral direction without rotation under a point load, which violates equilibrium as well as good sense.} although they trend in the correct direction. Keller's optimized wheels have different numbers of spokes on the left and right sides even for symmetrically-dished wheels, which suggests flaws in implementation\footnote{In calculating the lateral stiffness, he occasionally came across negative values which he incorrectly attributed to buckling. Goldberg's formulas do not capture elastic instability and the error is more likely due to a misinterpretation of signs.}. However, it is common to use spokes of different thicknesses on the left and right sides of an asymmetrically-dished wheel.

\begin{table}
\caption{Comparison of selected references on wheel optimization.}
\begin{tabularx}{\textwidth}{l L L L L L}
\hline
\bf Ref. & \bf Subject & \bf Design space & \bf Objective & \bf Model & \bf Algorithm\\
\hline
\cite{Svensson2015} & bicycle wheel    & spoke lacing pattern  & Multi-objective: stiffness and peak spoke force &
	Rigid rim, linear-elastic spokes$^{\rm a}$ & Genetic algorithm NSGA-II and NSGA-III \cite{Deb2002,Deb2014}\\

\cite{Keller2013}   & bicycle wheel    & spoke lacing pattern  & Weighted average of lateral stiffness and rotational stiffness &
	Rigid rim, linear-elastic spokes$^{\rm b}$ & brute-force search over hypercube\\

\cite{Zuo2011}      & automotive wheel & material distribution & Mean compliance$^{\rm c}$ &
	2D (plane stress) finite-element method & Bi-directional evolutionary structural optimization (BESO) \cite{Huang2007}\\
\hline
\end{tabularx}
\raggedright{$^{\rm a}$Using a video-game multiphysics engine.}\\
\raggedright{$^{\rm b}$Using equations developed by Goldberg \cite{Goldberg1984}.}\\
\raggedright{$^{\rm c}$Minimized the stored strain energy under a fixed loading scenario.}
\end{table}

% Intro
%   What are the most relevant performance criteria?
%   What choices will optimize those criteria?
%   What design parameters have the most significant effect on performance?

% Lit review
%   Models: rigid rim, elastic spoke models (Svensson, Keller)
%           finite-element (Zuo)
%           meta-model from ABAQUS (Czech)
%   Techniques: brute-force calculation over hypercube (Keller)
%               genetic algorithm (Svensson)
%               Topology optimization (Zuo)
%               Genetic algorithm with meta-model -> topology optimization of metamaterial (Czech)

% Missing in the literature: Analytical-model-based optimization

\section{Design space}

\subsection{Relevant design parameters}
% Describe design parameters relevant to the theory

\subsection{Narrowing the design space}
% Many design parameters are constrained, trivial, or redundant
%   Rim radius trivially optimizes to zero. Obviously needs to be constrained
%   Number of spokes and EA_s are linked
%   Hub width trivially optimizes to approx 45 degree spoke angle, usually constrained anyway
%   Rim xc radius optimizes to infinity, unless otherwise constrained
%   Rim cross-section shape could be a parameter, or constrain to either circular or fixed mu
%   n_s*EA and rim(r, t) add up to total mass

\subsection{Rim mass fraction}
% Describe rim mass fraction parameter.
% Develop expressions for t and n_s*A_s as a function of:
%   f_rim, R, M, w/R, r/R, rho_rim/rho_spokes


\section{One-parameter optimization}
% Figure: Trends in 1-D optimization.
%   For 4 different masses: (a) K_lat vs. f_rim (b) T_c vs. f_rim (c) P_c vs f_rim
%   Compare against ABAQUS results for one mass

\begin{figure}
\centering
\includesvg{\rootdir/figs/optimization/}{1D_trends}
\caption{...}
\label{fig:opt_1D_trends}
\end{figure}

\inprogress

\end{document}
