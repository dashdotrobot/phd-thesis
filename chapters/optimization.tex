%!TEX root = ../thesis.tex
\providecommand{\rootdir}{..}
\documentclass[\rootdir/thesis.tex]{subfiles}

\begin{document}

% Intro
%   What are the most relevant performance criteria?
%   What choices will optimize those criteria?
%   What design parameters have the most significant effect on performance?

% Lit review
%   Models: rigid rim, elastic spoke models (Svensson, Keller)
%           finite-element (Zuo)
%           meta-model from ABAQUS (Czech)
%   Techniques: brute-force calculation over hypercube (Keller)
%               genetic algorithm (Svensson)
%               Topology optimization (Zuo)
%               Genetic algorithm with meta-model -> topology optimization of metamaterial (Czech)

% Missing in the literature: Analytical-model-based optimization

\section{Design space}

\subsection{Relevant design parameters}
% Describe design parameters relevant to the theory

\subsection{Narrowing the design space}
% Many design parameters are constrained, trivial, or redundant
%   Rim radius trivially optimizes to zero. Obviously needs to be constrained
%   Number of spokes and EA_s are linked
%   Hub width trivially optimizes to approx 45 degree spoke angle, usually constrained anyway
%   Rim xc radius optimizes to infinity, unless otherwise constrained
%   Rim cross-section shape could be a parameter, or constrain to either circular or fixed mu
%   n_s*EA and rim(r, t) add up to total mass

\subsection{Rim mass fraction}
% Describe rim mass fraction parameter.
% Develop expressions for t and n_s*A_s as a function of:
%   f_rim, R, M, w/R, r/R, rho_rim/rho_spokes


\section{One-parameter optimization}
% Figure: Trends in 1-D optimization.
%   For 4 different masses: (a) K_lat vs. f_rim (b) T_c vs. f_rim (c) P_c vs f_rim
%   Compare against ABAQUS results for one mass

\begin{figure}
\centering
\includesvg{\rootdir/figs/optimization/}{1D_trends}
\caption{...}
\label{fig:opt_1D_trends}
\end{figure}

\inprogress

\end{document}
