%!TEX root = ../thesis.tex
\providecommand{\rootdir}{..}
\documentclass[\rootdir/thesis.tex]{subfiles}

\begin{document}

\section{Kinematics and strain energy}
A schematic of a typical bicycle wheel is shown in Figure \ref{fig:wheel_schematic}. The structure consists of a hub, rim, and spokes. The spokes are connected to two parallel flanges on the hub and the resulting projected bracing angle, $\alpha$, stabilizes the rim laterally, like guy-wires. Since the introduction of the tangent-spoked wheel by Starley in 1874 \cite{Hadland2014}, the spokes on most wheels are inclined by an angle $\beta$ in the plane of the rim relative to the radial vector in order to efficiently transmit torque between the hub and rim. Conventional spokes are threaded into nipples set into the rim which can be tightened and loosened independently. The spokes are tensioned during construction to prevent them from buckling when the wheel carries load. Bicycle rims today are typically constructed from thin-walled extruded sections, while rims on older bicycles were constructed from roll-formed metal strip, or sometimes solid wood.

\begin{figure}[h]
\centering
\includesvg{\rootdir/figs/stress_analysis/}{wheel_schematic}
\caption{Schematic of a typical bicycle wheel. \textbf{(a)} Side view, looking at the hub. \textbf{(b)} Rim cross-section showing local coordinate system at the shear center and vector spoke offset $\mathbf{b}_s$. \textbf{(c)} Rim cross-section after deformation. Tangential displacement $w$ is not shown.}
\label{fig:wheel_schematic}
\end{figure}

To describe forces and deformations of a wheel, we employ a local coordinate triad whose origin is at the cross-section centroid, $C$. The basis vector $\mathbf{e}_2$ points radially inwards, the basis vector $\mathbf{e}_3$ points in the circumferential direction of increasing arc length $s$, and the lateral basis vector $\mathbf{e}_1$ completes a right-hand triad.

\subfile{\rootdir/chapters/stress_analysis/rim_strain_energy}

\subsection{Deformation of the spoke system}
\subfile{\rootdir/chapters/stress_analysis/spoke_stiffness}


\section{Stress and deformation of the pretensioned wheel}
\label{sec:radial_bulging}
\subfile{\rootdir/chapters/stress_analysis/radial_bulging}


\section{Stresses in the wheel under external loads}

We assume the rim is loaded by distributed forces $f_u,f_v,f_w$ and a distributed moment $m$. The total potential energy in the deformed configuration under this system of loads is
\begin{equation}
\label{eq:TotPot}
\Pi = \Pi_{rim} + U_{spokes} - \int_0^{2\pi R} (f_uu+f_vv+f_ww+m\phi)\, ds
\end{equation}

The solution for the displacements is found by minimizing \eqref{eq:TotPot}. The Euler-Lagrange equations for \eqref{eq:TotPot} give a set of four coupled differential equations for the displacement variables:
\begin{equation}
\label{eq:EulerLagrange}
\begin{bmatrix}
 \D_{vv} + \kvv & \D_{vw} + \kvw & \kuv              & \kvp\\
-\D_{vw} + \kvw & \D_{ww} + \kww & \kuw              & \kwp\\
\kuv            & \kuw           & \D_{uu} + \kuu    & \D_{u\phi} + \kup\\
\kvp            & \kwp           & \D_{u\phi} + \kup & \D_{\phi\phi} + \kpp
\end{bmatrix}
\begin{bmatrix}
v\\w\\u\\\phi
\end{bmatrix}=
\begin{bmatrix}
f_v\\f_w\\f_u\\m
\end{bmatrix}
\end{equation}

where
\begin{align*}
\D_{vv} &= (EI_1 + EAy_0^2)\D^4 - 2EA\left(\frac{y_0}{R}\right)\D^2 + \frac{EA}{R^2}\\
\D_{ww} &= -\left(\frac{EI_1}{R^2} + EA\left(1+\frac{y_0}{R}\right)^2\right)\D^2\\
\D_{vw} &= \left(\frac{EI_1}{R} + EAy_0\left(1+\frac{y_0}{R}\right)\right)\D^3 - \frac{EA}{R}\left(1+\frac{y_0}{R}\right)\D\\
\D_{uu} &= \left(EI_2 + \frac{EI_w}{R^2}\right)\D^4 - \frac{GJ}{R^2}\D^2
    + R\bar{T}\left(1 + \frac{r_y^2}{R^2}\right)\D^2\\
\D_{\phi\phi} &= EI_w\D^4 - GJ\,\D^2 + \frac{EI_2}{R^2}
    + Rr_0^2\bar{T}\D^2 + y_0\bar{T}\\
\D_{u\phi} &= -\frac{EI_w}{R}\D^4 + \left(\frac{EI_2}{R} + \frac{GJ}{R}\right)\D^2
    - \bar{T}(r_y^2 + Ry_0)\D^2
\end{align*}

and $\D^n \equiv (d/ds)^n$, $r_0^2 = r_x^2 + r_y^2 + y_0^2$.

Due to the spoke stiffness parameters $\bar{k}_{ij}$, all of the displacement variables appear in each of the equilibrium equations. In the most general case with no further simplifications or symmetries, the combined governing equation is a 14\textsuperscript{th} order linear ordinary differential equation with constant coefficients.


\section{Mode stiffness matrix method}
\label{sec:ModeMatrix}
\subfile{\rootdir/chapters/stress_analysis/mode_matrix}


\section{Stresses in wheels with neglegible radial-lateral coupling}
In most cases relevant to real wheels many of the coupling terms $\bar{k}_{ij}$ $(i \neq j)$ are identically zero or are small compared to other relevant quantities. For example, in a wheel with either mirror symmetry or mirror-rotational symmetry across the plane of the wheel, all the off-diagonal terms $\bar{k}_{ij}$ $(i \neq j)$ are identically zero, except $\bar{k}_{u\phi}$. In the case where $\bar{k}_{uv}=\bar{k}_{uw}=\bar{k}_{v\phi}=\bar{k}_{w\phi}=0$, the equilibrium equations decouple into a pair of equations for radial-tangential deformations and a pair of equations for lateral-torsional deformations. Previous theoretical studies on the bicycle wheel by Pippard, et. al. \cite{Pippard1931,Pippard1932a,Pippard1932b}, Smith \cite{Smith1901}, and Burgoyne and Dilmaghanian \cite{Burgoyne1993} are derived by implicitly assuming a decoupled form of \eqref{eq:EulerLagrange}.

\subsection{Loads in the plane of the wheel}
\label{sec:RadTan}
\subfile{\rootdir/chapters/stress_analysis/radial_tangential}

\subsection{Loads out of the plane of the wheel}
\label{sec:Lateral}
\subfile{\rootdir/chapters/stress_analysis/lateral_torsional}

\section{Limits of the continuum approximation}
\inprogress




\end{document}
