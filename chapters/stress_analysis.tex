\providecommand{\rootdir}{..}
\documentclass[\rootdir/thesis.tex]{subfiles}

\begin{document}

\section{Kinematics and strain energy}
A schematic of a typical bicycle wheel is shown in Figure \ref{fig:wheel_schematic}. The structure consists of a hub, rim, and spokes. The spokes are connected to two parallel flanges on the hub and the resulting projected bracing angle, $\alpha$,  stabilizes the rim laterally. In this way, the spokes are analogous to guy-wires. Conventional spokes are threaded into nipples set into the rim which can be tightened and loosened independently. The spokes are tensioned during construction to prevent them from buckling when the wheel carries load.

\begin{figure}[h]
\centering
\includesvg{\rootdir/figs/stress_analysis/}{wheel_schematic}
\caption{Schematic of a typical bicycle wheel. \textbf{(a)} Side view, looking at the hub. \textbf{(b)} Rim cross-section showing local coordinate system at the shear center and vector spoke offset $\mathbf{b}_s$. \textbf{(c)} Rim cross-section after deformation. Tangential displacement $w$ is not shown.}
\label{fig:wheel_schematic}
\end{figure}

Bicycle rims today are typically constructed from thin-walled extruded sections, while rims on older bicycles were constructed from roll-formed metal strip, or sometimes solid wood. To describe forces and deformations of a wheel, we employ a local coordinate triad whose origin is at the cross section shear center, $S$. (The shear center is that unique point in a beam cross section at which a shear load applied to a beam with no intrinsic curvature causes no twist.) The basis vector $\mathbf{e}_2$ points radially inwards, the basis vector $\mathbf{e}_3$ points in the circumferential direction of increasing arc length $s$, and the lateral basis vector $\mathbf{e}_1$ completes a right-hand triad.

\subsection{Deformation of the rim}

The rim is modeled as a circular beam of constant cross-section with an axis of symmetry in the plane of the wheel. The height of the shear center $S$ relative to the centroid is $y_0$. The displacement of the rim is described in terms of the cross-section displacement $u\mathbf{e}_1 + v\mathbf{e}_2 + w\mathbf{e}_3$ in polar coordinates and rotation angle $\phi$ of the cross-section about $\mathbf{e}_3$.

The deformation of the rim is then defined by the change in curvatures and twist of its line of shear centers, and longitudinal extension of its centerline (which differs from the line of shear centers if the centroid is offset). To compute these quantities, we begin with the rotation of each cross-section, defined as a vector in terms of the cylindrical basis vectors. According to Euler-Bernoulli theory, the cross sections do not shear relative to the beam axis, so rotation is defined by slopes relative to the axis, plus a twisting rotation. Thus the cross section rotation vector is $(v + w/R) \mathbf{e}_1 + u' \mathbf{e}_2 + \phi\mathbf{e}_3$, where $()'$ denotes the derivative with respect to circumferential arc length $s$. We differentiate with respect to $s$ to get a bending and twist vector, recalling that $\et' = -\eh/R$ and $\eh' = \et/R$. The result is:
\begin{subequations}
\label{eq:kappa}
\begin{align}
\mathbf{\kappa} = \kappa_1\eo + \kappa_2\et + \kappa_3\eh\\
\kappa_1 = \left( v'' + \frac{w'}{R} \right) \label{eq:kappa1}\\
\kappa_2 = \left( u'' + \frac{\phi}{R} \right) \label{eq:kappa2}\\
\kappa_3 = \left( \phi' - \frac{u'}{R} \right) \label{eq:kappa3}
\end{align}
\end{subequations}

We define these three components as in-plane bending, out-of-plane bending, and twisting. For understanding rim deformation, it is worth remarking that $\phi=$ constant creates pure bending (ring eversion), while $u'=$ constant creates pure torsion (analogous to a helical spring).

To determine strain of the centroidal line, we need the displacement vector of the centroid: $\mathbf{u}_c = (u+y_0\phi)\eo + v\et + (w+y_0 (v'+w/R))\eh$. We differentiate with respect to $s$ to determine the $\eh$ component or extensional strain:
\begin{equation}
\label{eq:mem_strain}
\varepsilon_m = w' - \frac{v}{R} + y_0\left(v'' + \frac{w'}{R} \right)
\end{equation}

Due to the prestressing process, the reference configuration is not a state of zero stress. In addition to the tension in the spokes, the rim is under compression\cite{Sharp} (proportional to the tension in the spokes) and may also support a uniform radial bending moment about $\eo$ depending on the construction method of the rim, as well as a periodic bending moment from the separation of discrete spokes. Most modern rims are constructed from profiles permanently deformed into a circle close to the final radius, so the uniform moment can generally be neglected. We will show that the periodic moment can be neglected provided the wheel has a sufficient number of spokes compared with its in-plane bending stiffness.

The increase in strain energy in the rim in moving from the reference configuration to a deformed configuration can be decomposed into components for centroidal axial stretching, radial and lateral bending, twisting, and varied warping due to twist gradient:
\begin{equation}
\label{eq:U_rim}
U_{rim} = \frac{1}{2} \int_0^{2\pi R}[EA\varepsilon_m^2 + EI_1\kappa_1^2+EI_2\kappa_2^2 + GJ\kappa_3^2 + EI_w(\kappa_3')^2]ds
\end{equation}

The warping energy $EI_w(\kappa_3')^2$ is primarily related to the bending energy in the rim sidewalls (which deform in a similar manner as the flanges of an I-beam) of the rim due to varying torsion. For a single-wall rim, the resistance to warping can account for most of the effective torsional stiffness.

When the wheel is significantly prestressed and the rim undergoes lateral deformations, the strain energy \eqref{eq:U_rim} obtained from the linearized curvatures \eqref{eq:kappa} significantly overestimates the increase in strain energy. The axial compressive stress in the rim reduces the lateral stiffness of the wheel and can result in buckling under excessive spoke tension with no externally-applied loads. This phenomenon will be examined in detail in Chapter \ref{tension_buckling}.

We consider a rim deformation from a uniformly prestressed planar configuration $(u,v,w,\phi) = (0,v_0,w_0,0)$  under a constant compressive axial load $N_r$ to a non-planer configuration $(u,v,w,\phi) = (u,v_0,w_0,\phi)$. Trahair and Papangelis show that the total potential energy for a curved beam undergoing this lateral-torsional deformation with an appropriate non-linear strain measure has the form \cite{TrahairPapangelis}
\begin{equation}
\label{V_rim}
\Pi_{rim} = U_{rim} - \frac{1}{2}N_r \int_0^{2\pi R}
	\left[u'^2 + r_x^2\phi'^2 + r_y^2\left(\frac{u'}{R} - \phi'\right)^2
	      + y_0\left(2u''\phi - \frac{\phi^2}{R}\right) + y_0^2\phi'^2 \right]ds
\end{equation}

Without simplifying assumptions, the total potential energy under an arbitrary deformation $(u,v,w,\phi)$ includes up to 4th order terms in the displacements and their first and second derivatives. Therefore the resulting equilibrium equations would be non-linear (including hundreds of terms) and impossible to solve by analytical techniques (and extremely impractical by numerical techniques). The approximation \eqref{V_rim} is obtained by expanding the fully-nonlinear form of the potential energy, factoring out terms having the form of Equation \eqref{eq:mem_strain} and replacing with $N_r/EA$ (axial strain), and neglecting second-order and higher terms in the planar displacements $v,w$. The strain energy term $U_{rim}$ in \eqref{V_rim} is given by \eqref{eq:U_rim}. This simplification can be justified by the observation that the lateral stiffness of the rim is significantly smaller than the radial stiffness due to the small lateral projection of the spokes. In pre-stressed rings without spokes, the in-plane displacements must be considered which can lead to the appearance of in-plane buckling modes which are supressed in the bicycle wheel due to the elastic restraint provided by the spokes.

\subsection{Deformation of the spoke system}
\subfile{\rootdir/chapters/stress_analysis/spoke_stiffness}


\section{Stress and deformation of the pretensioned wheel}
\label{sec:radial_bulging}
\subfile{\rootdir/chapters/stress_analysis/radial_bulging}


\section{Stresses in the wheel under external loads}

We assume the rim is loaded by distributed forces $f_u,f_v,f_w$ and a distributed moment $m$. The total potential energy in the deformed configuration under this system of loads is
\begin{equation}
\label{eq:TotPot}
\Pi = \Pi_{rim} + U_{spokes} - (f_uu+f_vv+f_ww+m\phi)
\end{equation}

The solution for the displacements is found by minimizing \eqref{eq:TotPot}. The Euler-Lagrange equations for \eqref{eq:TotPot} give a set of four coupled differential equations for the displacement variables:
\begin{equation}
\label{eq:EulerLagrange}
\begin{bmatrix}
 \D_{vv} + \kvv & \D_{vw} + \kvw & \kuv              & \kvp\\
-\D_{vw} + \kvw & \D_{ww} + \kww & \kuw              & \kwp\\
\kuv            & \kuw           & \D_{uu} + \kuu    & \D_{u\phi} - Ry_0\bar{T}\D^3+ \kup\\
\kvp            & \kwp           & \D_{u\phi} + Ry_0\bar{T}\D^3 + \kup & \D_{\phi\phi} + \kpp
\end{bmatrix}
\begin{bmatrix}
v\\w\\u\\\phi
\end{bmatrix}=
\begin{bmatrix}
f_v\\f_w\\f_u\\m
\end{bmatrix}
\end{equation}

where
\begin{align*}
\D_{vv} &= (EI_1 + EAy_0^2)\D^4 - 2EA\left(\frac{y_0}{R}\right)\D^2 + \frac{EA}{R^2}\\
\D_{ww} &= -\left(\frac{EI_1}{R^2} + EA\left(1+\frac{y_0}{R}\right)^2\right)\D^2\\
\D_{vw} &= \left(\frac{EI_1}{R} + EAy_0\left(1+\frac{y_0}{R}\right)\right)\D^3 - \frac{EA}{R}\left(1+\frac{y_0}{R}\right)\D\\
\D_{uu} &= \left(EA_2 + \frac{EI_w}{R^2}\right)\D^4 - \frac{GJ}{R^2}\D^2
    + R\bar{T}\left(1 + \frac{r_y^2}{R^2}\right)\D^2\\
\D_{\phi\phi} &= EI_w\D^4 - GJ\,\D^2 + \frac{EI_2}{R^2}
    + Rr_0^2\bar{T}\D^2 + y_0\bar{T}\\
\D_{u\phi} &= -\frac{EI_w}{R}\D^4 + \left(\frac{EI_2}{R} + \frac{GJ}{R}\right)\D^2
    - r_y^2\bar{T}\D^2
\end{align*}

and $\D^n \equiv (d/ds)^n$, $r_0^2 = r_x^2 + r_y^2 + y_0^2$.

Due to the spoke stiffness parameters $\bar{k}_{ij}$, all of the displacement variables appear in each of the equilibrium equations. In the most general case with no further simplifications or symmetries, the combined governing equation is a 14\textsuperscript{th} order linear ordinary differential equation with constant coefficients.


\section{Mode stiffness matrix method}
\label{sec:ModeMatrix}
\subfile{\rootdir/chapters/stress_analysis/mode_matrix}


\section{Stresses in wheels with neglegible radial-lateral coupling}
In most cases relevant to real wheels many of the coupling terms $\bar{k}_{ij}$ $(i \neq j)$ are identically zero or are small compared to other relevant quantities. For example, in a wheel with either mirror symmetry or mirror-rotational symmetry across the plane of the wheel, all the off-diagonal terms $\bar{k}_{ij}$ $(i \neq j)$ are identically zero, except $\bar{k}_{u\phi}$. In the case where $\bar{k}_{uv}=\bar{k}_{uw}=\bar{k}_{v\phi}=\bar{k}_{w\phi}=0$, the equilibrium equations decouple into a pair of equations for radial-tangential deformations and a pair of equations for lateral-torsional deformations. Previous theoretical studies on the bicycle wheel by Pippard, et. al. \cite{Pippard}, Smith \cite{Smith}, and Burgoyne and Dilmaghanian \cite{Burgoyne} are derived by implicitly assuming a decoupled form of \eqref{eq:EulerLagrange}.

\subsection{Loads in the plane of the wheel}
\label{sec:RadTan}
\subfile{\rootdir/chapters/stress_analysis/radial_tangential}

\subsection{Loads out of the plane of the wheel}
\label{sec:Lateral}
\subfile{\rootdir/chapters/stress_analysis/lateral_torsional}

\section{Limits of the continuum approximation}





\end{document}
