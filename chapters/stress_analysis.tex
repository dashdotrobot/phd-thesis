\providecommand{\rootdir}{..}
\documentclass[\rootdir/thesis.tex]{subfiles}

\begin{document}

\section{Kinematics and strain energy}
A schematic of a typical bicycle wheel is shown in Figure \ref{fig}. The structure consists of a hub, rim, and spokes. The spokes are connected to two parallel flanges on the hub and the resulting projected bracing angle, $\alpha$,  stabilizes the rim laterally. In this way, the spokes are analogous to guy-wires. Conventional spokes are threaded into nipples set into the rim which can be tightened and loosened independently. The spokes are tensioned during construction to prevent them from buckling when the wheel carries load.

Bicycle rims today are typically constructed from thin-walled extruded sections, while rims on older bicycles were constructed from roll-formed metal strip, or sometimes solid wood. To describe forces and deformations of a wheel, we employ a local coordinate triad whose origin is at the cross section shear center, $S$. (The shear center is that unique point in a beam cross section at which a shear load causes no twist.) The basis vector $\mathbf{e}_2$ is radially inwards, the basis vector $\mathbf{e}_3$ is in the circumferential direction of increasing arc length $s$, and the lateral basis vector $\mathbf{e}_1$ completes a right-hand triad.

\subsection{Deformation of the rim}

The rim is modeled as a circular beam of constant cross-section with an axis of symmetry in the plane of the wheel. The height of the shear center $s$ relative to the centroid is $y_0$. The displacement of the rim is described in terms of the cross-section displacement $u\mathbf{e}_1 + v\mathbf{e}_2 + w\mathbf{e}_3$ in polar coordinates and rotation angle $\phi$ of the cross-section about $\mathbf{e}_3$.

The deformation of the rim is then defined by the change in curvatures and twist of its line of shear centers, and longitudinal extension of its centerline (which differs from the line of shear centers if the centroid is offset). To compute these quantities, we begin with the rotation of each cross-section, defined as a vector in terms of the cylindrical basis vectors. According to Euler-Bernoulli theory, the cross sections do not shear relative to the beam axis, so rotation is defined by slopes relative to the axis, plus a twisting rotation. Thus the cross section rotation vector is $(v + w/R) \mathbf{e}_1 + u' \mathbf{e}_2 + \phi\mathbf{e}_3$, where $()'$ denotes the derivative with respect to circumferential arc length $s$. We differentiate with respect to $s$ to get a bending and twist vector, recalling that $\et' = -\eh/R$ and $\eh' = \et/R$. The result is:
\begin{subequations}
\label{eq:kappa}
\begin{align}
\mathbf{\kappa} = \kappa_1\eo + \kappa_2\et + \kappa_3\eh\\
\kappa_1 = \left( v'' + \frac{w'}{R} \right) \label{eq:kappa1}\\
\kappa_2 = \left( u'' + \frac{\phi}{R} \right) \label{eq:kappa2}\\
\kappa_3 = \left( \phi' - \frac{u'}{R} \right) \label{eq:kappa3}
\end{align}
\end{subequations}

We define these three components as in-plane bending, out-of-plane bending, and twisting. For understanding rim deformation, it is worth remarking that $\phi=$ constant creates pure bending (ring eversion), while $u'=$ constant creates pure torsion (analogous to a helical spring).

To determine strain of the centroidal line, we need the displacement vector of the centroid: $\mathbf{u}_c = (u+y_0\phi)\eo + v\et + (w+y_0 (v'+w/R))\eh$. We differentiate with respect to $s$ to determine the $\eh$ component or extensional strain:
\begin{equation}
\label{eq:mem_strain}
\varepsilon_m = w' - \frac{v}{R} + y_0\left(v'' + \frac{w'}{R} \right)
\end{equation}

Due to the prestressing process, the reference configuration is not a state of zero stress. In addition to the tension in the spokes, the rim is under compression\cite{Sharp} (proportional to the tension in the spokes) and may also support a uniform radial bending moment about $\eo$ depending on the construction method of the rim, as well as a periodic bending moment from the separation of discrete spokes. Most modern rims are constructed from profiles permanently deformed into a circle close to its final radius, so the uniform moment can generally be neglected. We will show that the periodic moment can be neglected provided the wheel has a sufficient number of spokes compared with its in-plane bending stiffness.

The increase in strain energy in the rim in moving from the reference configuration to a deformed configuration can be decomposed into components for centroidal axial stretching, radial and lateral bending, twisting, and varied warping due to twist gradient:
\begin{equation}
\label{eq:U_rim}
U_{rim} = \frac{1}{2} \int_0^{2\pi R}[EA\varepsilon_m^2 + EI_1\kappa_1^2+EI_2\kappa_2^2 + GJ\kappa_3^2 + EI_w(\kappa_3')^2]ds
\end{equation}

The warping energy $EI_w(\kappa_3')^2$ is primarily related to the bending energy in the rim sidewalls (which deform in a similar manner as the flanges of an I-beam) of the rim due to varying torsion. For a single-wall rim, the resistance to warping can account for most of the effective torsional stiffness.

\subsection{Deformation of the spoke system}

\subfile{\rootdir/chapters/stress_analysis/spoke_stiffness}


\section{Stress and deformation of the pretensioned wheel}

In the absence of external loads, the bicycle rim is loaded radially by the system of spokes, and to a lesser extent by the tire air pressure if the tire casing is not bonded to the rim\cite{Burgoyne}. The rim shrinks due to the compressive hoop stress induced by the pull of the spokes and bows inwards at each spoke due to the bending moment introduced by the spacing between spokes. The average radial tension per unit length exerted by the spokes is
\begin{equation}
\label{eq:Tbar}
\bar{T} = \frac{1}{2\pi R} \sum_i^{n_s} \mathbf{T}_i\cdot\et
\end{equation}

Consider a rim segment containing a single spoke, as shown in Figure \ref{fig:radial_bulging} (a). By symmetry, the axial force and moment must be equal at the two ends. Equilibrium of forces in the horizontal direction immediately requires $V=0$. Sum of forces in the vertical direction gives
\begin{equation}
\label{eq:Nr}
N_r = \frac{\pi R\bar{T}}{n_s\sin{(\pi/n_s)}} \approx R\bar{T}
\end{equation}

where the second result, first derived by Sharp\cite{Sharp}, is obtained by noting that $\sin{\pi/n_s}\approx \pi/n_s$ for sufficiently large $n_s$. The internal forces at an arbitrary section at $\theta < \pi/n_s$ are obtained from equilibrium of the segment shown in Figure \ref{fig:radial_bulging} (a). The unknown end moment $M$ is determined from the condition that there can be no rotation of the cross-section at the symmetry point between spokes.
\begin{subequations}
\begin{align}
\label{eq:NVM}
N_r' &= R\bar{T}\cos{\theta}\\
V'   &= R\bar{T}\sin{\theta}\\
M'   &= R^2\bar{T} \left( \frac{\sin{\pi/n_s}}{\pi/n_s} - \cos{\theta} \right)
\end{align}
\end{subequations}

\begin{figure}
\centering
\includesvg{\rootdir/figs/stress_analysis/}{radial_bulging}
\caption{Radial deformation of the wheel under uniform tension. \textbf{(a)} Segment of rim containing a single spoke. \textbf{(b)} Bending moment induced by radial spoke pull. solid line = 36 spokes, dashed line = 24, dot-dash line = 16. \textbf{(c)} Ratio of deflection at a spoke due to bending and due to circumferential shrinkage.}
\label{fig:radial_bulging}
\end{figure}

Castigliano's method can be used to determine the displacement at the spoke due to axial compression alone and bending
\begin{align}
v_C &= \frac{R^2\bar{T}}{2EA} \left( \left(\frac{\pi/n_s}{\sin{\pi/n_s}}\right)^2 +
    \frac{\pi/n_s}{\tan{\pi/n_s}}\right)
    \approx \frac{R^2\bar{T}}{EA}\label{eq:vC}\\
v_M &= \frac{R^4\bar{T}}{2EI} \left( \left(\frac{\pi/n_s}{\sin{\pi/n_s}}\right)^2 +
    \frac{\pi/n_s}{\tan{\pi/n_s}} - 2\right)\label{eq:vM}
\end{align}

The relative contribution of $v_M$ is generally very small compared to $v_C$. To a very close approximation, $v_M/v_C=(1/45)(R/r_y)^2(\pi/n_s)^4$, where $r_y$ is the radius of gyration of the rim in the radial direction. Bending deformations become significant if the number of spokes is very low or the rim bending stiffness is very small in the radial direction. Low spoke-count wheels generally have very deep cross-sections to minimize bending deformation between spokes.


\section{Stresses in the wheel under external loads}

We assume the rim is loaded by distributed forces $f_u,f_v,f_w$ and a distributed moment $m$. The total potential energy in the deformed configuration under this system of loads is
\begin{equation}
\label{eq:TotPot}
\Pi = U_{rim} + U_{spokes} - (f_uu+f_vv+f_ww+m\phi)
\end{equation}

The solution for the displacements is found by minimizing \eqref{eq:TotPot}. The Euler-Lagrange equations for \eqref{eq:TotPot} give a set of four coupled differential equations for the displacement variables:
\begin{equation}
\label{eq:EulerLagrange}
\begin{bmatrix}
 \D_{vv} + \bar{k}_{vv} & \D_{vw} + \bar{k}_{vw} & \bar{k}_{uv} & \bar{k}_{v\phi}\\
-\D_{vw} + \bar{k}_{vw} & \D_{ww} + \bar{k}_{ww} & \bar{k}_{uw} & \bar{k}_{w\phi}\\
\bar{k}_{uv}    & \bar{k}_{uw}    & \D_{uu} + \bar{k}_{uu} & \D_{u\phi} + \bar{k}_{u\phi}\\
\bar{k}_{v\phi} & \bar{k}_{w\phi} & \D_{u\phi} + \bar{k}_{u\phi} & \D_{\phi\phi} + \bar{k}_{\phi\phi}
\end{bmatrix}
\begin{bmatrix}
v\\w\\u\\\phi
\end{bmatrix}=
\begin{bmatrix}
f_v\\f_w\\f_u\\m
\end{bmatrix}
\end{equation}

where
\begin{align*}
\D_{vv} &= (EI_1 + EAy_0^2)\D^4 - 2EA\left(\frac{y_0}{R}\right)\D^2 + \frac{EA}{R^2}\\
\D_{ww} &= -\left(\frac{EI_1}{R^2} + EA\left(1+\frac{y_0}{R}\right)^2\right)\D^2\\
\D_{vw} &= \left(\frac{EI_1}{R} + EAy_0\left(1+\frac{y_0}{R}\right)\right)\D^3 - \frac{EA}{R}\left(1+\frac{y_0}{R}\right)\D\\
\D_{uu} &= \left(EA_2 + \frac{EI_w}{R^2}\right)\D^4 - \frac{GJ}{R^2}\D^2\\
\D {u\phi} &= \frac{EI_w}{R}\D^4 - \left(\frac{EI_2}{R} + \frac{GJ}{R}\right)\D^2\\
\D_{\phi\phi} &= EI_w\D^4 - GJ\,\D^2 + \frac{EI_2}{R^2}
\end{align*}

and $\D^n \equiv (d/ds)^n$.

Due to the spoke stiffness parameters $\bar{k}_{ij}$, all of the displacement variables appear in each of the equilibrium equations. In the most general case with no further simplifications or symmetries, the combined governing equation is a 14\textsuperscript{th} order linear ordinary differential equation with constant coefficients. In most cases relevant to real wheels many of the coupling terms $\bar{k}_{ij}$ $(i \neq j)$ are identically zero or are small compared to other relevant quantities. For example, in a wheel with either mirror symmetry or mirror-rotational symmetry across the plane of the wheel, all the off-diagonal terms $\bar{k}_{ij}$ $(i \neq j)$ are identically zero, except $\bar{k}_{u\phi}$. In the case where $\bar{k}_{uv}=\bar{k}_{uw}=\bar{k}_{v\phi}=\bar{k}_{w\phi}=0$, the equilibrium equations decouple into a pair of equations for radial-tangential deformations and a pair of equations for lateral-torsional deformations.

\subsection{Loads in the plane of the wheel}

The radial-tangential equations are
\begin{subequations}
\begin{align}
\label{eq:rad_tan_2}
(EI_1 + EAy_0^2)\left( \ds{v}{4} + \frac{1}{R}\ds{w}{3} \right) -
    \frac{EA}{R}\left( \frac{dw}{ds} - \frac{v}{R} +y_0\left(2\ds{v}{2} + \frac{1}{R}\frac{dw}{ds} -
    R\ds{w}{3}\right) \right) +\bar{k}_{vv}v &= f_v\\
-\left( \frac{EA_1}{R} + EAy_0\left( 1 + \frac{y_0}{R} \right) \right)
    \left(\ds{v}{3} + \frac{1}{R}\ds{w}{2} \right) -
    EA\left(1+\frac{y_0}{R} \right) \left(\ds{w}{2} - \frac{1}{R}\frac{dv}{ds} \right) + \bar{k}_{ww}w &= f_w
\end{align}
\end{subequations}

Combining these two equations (equivalent to taking the determinant of the $v,w$ submatrix in Equation \eqref{eq:rad_tan_2}) results in a single governing equation for $v$:
\begin{multline}
\label{eq:rad_tan_ry}
\dt{v}{6} + \left[2-\lambda_{ww}\left(\left(\frac{y_0}{R}\right)^2 +
                                        \left(\frac{r_y}{R}\right)^2 \right) \right] \dt{v}{4}\\
          + \left[1+\lambda_{vv}\left(\left(1+\frac{y_0}{R}\right)^2+\left(\frac{r_y}{R}\right)^2\right)
                   +2\lambda_{ww}\left(\frac{y_0}{R}\right)\right]\dt{v}{2}
          - \lambda_{ww}\left[1+\lambda_{vv}\left(\frac{r_y}{R}\right)^2\right] v = 0
\end{multline}

where $\lambda_{vv}=\bar{k}_{vv}R^4/EI_1$ and $\lambda_{ww}=\bar{k}_{ww}R^4/EI_1$. The tangential spoke stiffness $\bar{k}_{ww}$ is related to the projection of the spoke stiffness along the tangential direction. For practical wheels, $\bar{k}_{ww}$ is at least 2 orders of magnitude smaller than $\bar{k}_{vv}$. Analytical solutions to Equation \eqref{eq:rad_tan_ry} are possible because the roots of the characteristic equation come in three pairs, $\pm r_i$.

Further simplification is possible for most practical cases by noting that $y_0,r_y \ll R$. These conditions are equivalent to assuming that the beam is doubly-symmetric, and that extension of the centerline can be neglected, respectively.
\begin{equation}
\label{eq:rad_tan}
\dt{v}{6} + 2\dt{v}{4}+(1+\lambda_{vv})\dt{v}{2} - \lambda_{ww}v=0
\end{equation}

Equation \eqref{eq:rad_tan} is the same as Pippard’s result obtained from equilibrium of a differential element of the rim24. The boundary conditions and solution procedure is identical for Equations \eqref{eq:rad_tan} and \eqref{eq:rad_tan_ry}.

\subsubsection{Loading case I: radial point load}

The reaction force from the road on the wheel may be represented by a radial point load at $\theta=0$. The radial displacement closely resembles the classical solution of a point load acting on a beam supported by an elastic foundation\cite{Hetenyi}. The radial deformation is confined to a narrow arc around the load point in which spokes lose tension, bounded by an ``overshoot'' region where spokes tensions increase and reach a maximum within about 30$^{\circ}$ of the load point. Far from the load, spoke tensions generally change by a very small amount on the order of 5\% of the applied load. This has led some authors to claim that ``the hub stands on the spokes beneath it,'' despite the counter-intuitive image this conjures\cite{Brandt}. Others insist that the hub ``hangs from the spokes above it'' due to the fact that the spoke tensions above the hub are higher than those below it.  Both statements are mathematically equivalent, and it is clear that the lower spokes play the most significant dynamic role in supporting the bicycle and are most prone to loosening or buckling under load.

\begin{figure}
\centering
\includesvg{\rootdir/figs/stress_analysis/}{rad_tan_grid}
\caption{(a)-(c) Deformation of a wheel subject to radial point load, normalized by $P/\pi R\bar{k}_{vv}$. For the dark lines $\lambda_{vv}=1000$ and for the light lines $\lambda_{vv}=10$. (d)-(f) Deformation of a wheel subject to a tangential point load.}
\label{fig:rad_tan_grid}
\end{figure}

If, as is generally the case for practical wheels, $\lambda_{ww} \ll \lambda_{ww}$, or more rigorously for the case of purely radial spokes, Equation \eqref{eq:rad_tan} simplifies to
\begin{equation}
\label{eq:rad}
\dt{v}{4} + 2\dt{v}{2} + (1+\lambda_{vv})v=0
\end{equation}

Since all the derivatives have even order, a relatively simple analytical solution to \eqref{eq:rad} exists. The most relevant loading case for the bicycle wheel is approximated by a point load $P$ located at $\theta=0$ corresponding to the reaction force from the road. In this case the radial displacement is given by
\begin{equation}
\label{eq:rad_soln}
v = \frac{PR^3}{4abEI_1} \left( \frac{2ab}{\pi\eta^2} + \frac{b\sinh{a\theta}\cos{b\theta}}{\eta} 
                               -\frac{a\cosh{a\theta}\sin{b\theta}}{\eta}
                               +A\cosh{a\theta}\cos{b\theta} + B\sinh{a\theta}\sin{b\theta}\right)
\end{equation}

where
\begin{gather*}
\eta=\sqrt{\lambda_{vv} + 1}, \,\,\,\, a=\sqrt{\frac{\eta-1}{2}}, \,\,\,\, b=\sqrt{\frac{\eta+1}{2}}\\
A = -\frac{a\sin{2\pi b} + b\sinh{2\pi a}}{2\eta(\sinh^2{\pi a} + \sin^2{\pi b})}\\
B =  \frac{a\sinh{2\pi a} - b\sin{2\pi b}}{2\eta(\sinh^2{\pi a} + \sin^2{\pi b})}
\end{gather*}

This solution for radial spokes was first given in slightly different form by Smith\cite{Smith} in 1901 and later by Pippard\cite{Pippard} in 1931, who also performed experiments on spoked wheels. The non-dimensionalized radial stiffness depends only on the stiffness ratio $\lambda_{vv}$:
\begin{equation}
\label{eq:Krad}
\frac{K_{rad}}{\pi R \bar{k}_{vv}} =
    \frac{1}{\pi\lambda_{vv}} \left(\frac{1}{2\pi\eta^2}
                                    -\frac{b\sinh{a\pi}\cosh{a\pi} + a\sin{b\pi}\cos{b\pi}}
                                     {4ab\eta (\sinh^2{a\pi} + \sin^2{b\pi})} \right)^{-1}
\end{equation}

\subsubsection{Loading case II: tangential point load}

During acceleration (or braking using disc brakes), the reaction force from the road has a component in the tangential direction. The tangential displacement is primarily controlled by the stiffness $\bar{k}_{ww}$, while the ratio $\lambda_{ww}/\lambda_{vv}$ controls the degree to which radial displacement is also involved.

A very satisfactory approximation to the problem of a tangential load can be obtained by noting that $\lambda_{vv}\gg\lambda_{ww}$ and therefore the radial displacement is very small compared with the tangential displacement. Under this approximation, the rim rotates about the axle as a rigid body and the tangential stiffness is simply
\begin{equation}
\label{eq:Ktan}
K_{tan} = 2\pi R \bar{k}_{ww}
\end{equation}

\subsubsection{The role of spoke tension in supporting radial and tangential loads}

The model considered here depends on the assumption that a pretensioned spoke can equally support tension or compression (or rather, loss of tension) and that the elastic properties remain constant. In order for this assumption to be valid, all of the spokes must maintain a non-zero tension at all times. This condition may be violated if an excessive load is applied to the wheel.

Figure \ref{fig:radtan_Tinf} shows the change in spoke tension under different loading scenarios. In this example, the most critical spoke supports 40\% of the applied load, while the load sharing fractions for the nearest-neighbor and next-nearest-neighbor spokes are about 24\% and 3\%, respectively. A properly tensioned wheel should not admit spokes go slack under typical loads.

On a typical wheel with ``cross'' laced spokes, half of the spokes are inclined forward in the plane of the wheel, while the other half are inclined backwards. These are referred to as ``pushing'' and ``pulling'' or ``leading'' and ``trailing'' spokes due to their behavior under load. Pulling spokes increase their tension under acceleration torque while pushing spokes decrease their tension, as shown in Figure \ref{fig:radtan_Tinf} (b). Under a combined radial load and acceleration torque, the primary factor causing spokes to slacken is the radial load, while the primary factor causing spokes to tighten is the tangential load. Therefore, both types of loads should be accounted for when making fatigue calculations.

\begin{figure}
\centering
\includesvg{\rootdir/figs/stress_analysis/}{rad_tan_Tinf}
\caption{Change in spoke tension under (a) a unit radial load and (b) a unit tangential load. (c) Change in spoke tension for a typical loading scenario: 500 N radial load and 50 N tangential load.}
\label{fig:radtan_Tinf}
\end{figure}

\subsection{Loads out of the plane of the wheel}

Under lateral loads, the rim bends and twists into a non-planar shape. The wheel is considerably more flexible in the lateral direction than in the radial direction due to the small lateral projection of the spokes. When the rim undergoes lateral deformation, the potential energy of the compressive load induced by the spoke pretension is reduced. Due to the large lateral compliance, this reduction in potential energy can be significant compared to the increase in strain energy due to lateral bending and twisting. This leads to larger lateral deflections than would be predicted by the linear theory in Eqn. \eqref{eq:EulerLagrange}, with the possibility of lateral-torsional instability at a sufficiently high spoke tension.

We consider a variation of the displacements from the prestressed reference configuration defined by $(u,v,w,\phi)=(0,v_0,w_0,0)$ to a new deformed state defined by $(u,v,w,\phi)=(0+u,v_0,w_0,0+\phi)$. Trahair and Papangelis\cite{Trahair} give the virtual work due to a uniform compressive stress in a monosymmetric curved beam, assuming that the displacements $v_0,w_0$ are small compared to $u,r\phi$

\begin{equation}
\label{eq:V_rim}
V_{rim} = \frac{1}{2}R\bar{T} \int_0^{2\pi R}
    \left[u'^2 + r_x^2\phi'^2+r_y^2\left(\frac{u'}{R}-\phi'\right)^2
          +y_0\left(2u''\phi-\frac{\phi^2}{R}\right) + y_0^2\phi'^2 \right] ds
\end{equation}

Then the total potential energy, Equation \ref{eq:TotPot} is augmented by the virtual work due to compressive stress in the rim:

\begin{equation}
\label{eq:TotPot_V}
\Pi = U_{rim} + U_{spokes} - V_{rim} - (f_uu+m\phi)
\end{equation}

The Euler-Lagrange equations that guarantee the minimization of \eqref{eq:TotPot_V} are

\todo{Full lateral-torsional equations with y0, rx, ry}
\begin{subequations}
\label{eq:lat_ode_full}
\begin{align}
...
\end{align}
\end{subequations}

\section{Limits of the continuum approximation}


\section{Mode stiffness matrix method}

\end{document}
