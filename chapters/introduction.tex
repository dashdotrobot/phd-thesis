\documentclass[../thesis.tex]{subfiles}

\begin{document}

% The bicycle will celebrate its 200\textsuperscript{th} anniversary in 2018 if we define a bicycle as a steerable, two-wheeled, human-powered contraption.

In its heyday in the 1890's, the bicycle was regarded as one of the most advanced products of the Industrial Age. Inventors and industrialists spent considerable effort mastering the manufacture of lighter, more efficient machines. Many ubiquitous technologies made their debut on or were perfected for the bicycle including ball-bearings, chain drives, and tensioned wheels. Although most people know that the Wright brothers were bicycle mechanics, it is less commonly known that Henry Ford and Horace Dodge \cite{}, fathers of their eponymous automobile empires, got their start in the bicycle industry. Although these may be regarded as humble beginnings to a modern observer, at the time the bicycle was considered high technology.

The bicycle subsequently ceded this title to the auto industry, the aerospace industry, and finally to the digital electronics and nanotechnology industries. Bicycle technology has not radically changed since the introduction of derailleur gears in the 1920’s. There is no longer a crowd of technical minds bent on the perfection of ``the mechanical horse,'' but this should be seen as the inevitable maturation of a truly civilized technology. The bicycle is humanity's most efficient means of transportation---a technology for the age of climate change---and will be a key component of the modern, sustainable city.

Bicycles have not become significantly more complex over time, which has enabled widespread participation in bicycle design and maintenance by amateurs. A few industry standards dominate each major component, allowing parts to be easily swapped, upgraded, or modified. Most cities have bicycle co-ops where community members can collaborate, share tools and learn about bicycle repair. A classic example of “user innovation” was the invention of the mountain bike in California by a loose group of cycling enthusiasts. Using older steel-frame bicycles (mostly used Schwinns from the 1950’s), these pioneers hacked together rugged bikes to meet the demands of off-road cycling. Eventually, the industry picked up the trend and mountain bicycles constitute the majority of bicycles sold in America today\cite{NoHands}.


\section{History of the bicycle wheel}
With its elegant system of slender spokes, the wheel is the most recognizable feature of the bicycle---an invention whose success it critically enabled. Indeed, the term ``wheel'' was once used to refer to the entire bicycle; the League of American Bicyclists was once the League of American Wheelmen before changing their name in 1994.

The vast majority of bicycle wheels produced today are pretensioned ``wire-spoked'' wheels in which a slender system of pre-tensioned spokes resist buckling when loaded. Originally conceived for aircraft landing gear by Sir George Cayley in 1808, the prestressed wire-spoked wheel saw its first practical application on the bicycle and the design was later utilized in early automobiles and aircraft\cite{HadlandLessing}. Although materials and manufacturing methods have evolved over time, the fundamental design, a system of individually prestressed spokes in tension stabilized by a rim in compression, has not significantly changed since the 1870's.

For the first 50 years of the bicycle, wheels generally consisted of a small number of stout wooden spokes fitted into individual wooden rim sections, known as felloes, fitted together with mortice and tenon joints. An iron tire was heated and placed around the circumference of the rim and allowed to cool and contract, putting the spokes in compression and the tire in tension\cite{Sharp}. The structure was held together by this prestress; there was no need to bond the rim sections together. The spokes, necessarily large to resist buckling, made these wheels extremely heavy. Eugene Meyer obtained a patent for a tensioned wire-spoked wheel in 1869, which enabled the construction of much larger wheels\cite{Clayton}. In the tension-spoke wheel the method of prestress was reversed. Each slender iron spoke could be individually pre-tensioned, stabilized by an iron rim in compression.

Several other prestressing methods were developed around the same time. W. F. Reynolds and J. A. Mays introduced the Phantom bicycle featuring looped wire spokes secured to one flange of the hub, laced through an eyelet on the rim, and secured to the other hub flange\cite{Herlihy}. The spokes were all tensioned simultaneously by spreading apart the hub flanges. The spokes of James Starley's successful Ariel bicycle, which emerged radially from the hub, were pre-stressed simultaneously by rotating the hub relative to the rim by means of a pair of levers connected to the hub and secured to the rim by adjustable tensioning rods\cite{Caunter}. Although the spokes were attached radially (and therefore could not transmit appreciable torque), the prestress resulted in a small offset which conferred rotational stiffness. The tangent-spoked wheel, developed by Starley in 1874, achieved rotational stiffness by connecting the spokes tangent to the hub. This design has survived, in large part, up until the present day.

Prestressing of the spokes allowed the use of thin wire, rather than stout wooden spokes, enabling larger and lighter wheels. The gear of a direct-drive pedal-driven bicycle (the linear distance traveled per rotation of the cranks) is simply equal to the circumference of the wheel, so a larger wheel confers a considerable speed advantage. The high-wheel bicycle enjoyed popularity amongst primarily young men of means thoughout the 1880's and into the 1890's \cite{SocialHistory}. The development of practical and lightweight chain drives brought the ``safety bicycle'' to the masses---so named because the gear ratio afforded by the chain enabled the use of a smaller front wheel, reducing the risk of pitching over the handlebars.

\section{The suspension bridge and the Ferris Wheel}

The first prestressed wheels were commonly called ``suspension wheels'' at the time, perhaps due to an analogy with the suspension bridge, itself a relatively new structural innovation\cite{SciAm}.

\subsection{Materials and methods}

\section{Technical literature on wheels}

\end{document}
