%!TEX root = ../thesis.tex
\documentclass[../thesis.tex]{subfiles}

\begin{document}

In its heyday in the 1890s, the bicycle was regarded as one of the most advanced products of the Industrial Age. Inventors and industrialists spent considerable effort mastering the manufacture of lighter, more efficient machines. Many now-ubiquitous technologies made their debut on or were perfected for the bicycle including ball-bearings, chain drives, and tensioned wheels. Although most people know that the Wright brothers were bicycle mechanics, it is less commonly known that Henry Ford and Horace Dodge \cite{Reid2015}, fathers of their eponymous automobile empires, got their starts in the bicycle industry. Although these may be regarded as humble beginnings to a modern observer, at the time the bicycle was considered high technology.

The bicycle subsequently ceded this title to the auto industry, the aerospace industry, and finally to the digital electronics and nanotechnology industries. Bicycle technology has not radically changed since the introduction of derailleur gears in the 1920’s. There is no longer a crowd of technical minds bent on the perfection of ``the mechanical horse,'' but this may be seen as the inevitable maturation of a truly civilized technology. The bicycle is humanity's most efficient means of transportation\todo{Add a footnote and citation here}---a technology for the age of climate change---and will be a key component of the modern, sustainable city.

Bicycles have not become significantly more complex over time, which has enabled widespread participation in bicycle design and maintenance by amateurs. A few industry standards dominate each major component, allowing parts to be easily swapped, upgraded, or modified. Most cities have bicycle co-ops where community members can collaborate, share tools, and learn about bicycle repair. A classic example of “user innovation” was the invention of the mountain bike in California by a loose group of cycling enthusiasts. Using older steel-frame bicycles (mostly used Schwinns from the 1950’s), these pioneers hacked together rugged bikes to meet the demands of off-road cycling. Eventually, the industry picked up the trend and mountain bicycles constitute the majority of bicycles sold in America today \cite{Crown1996}.


\section{History of the bicycle wheel}
With its elegant system of slender spokes, the wheel is the most recognizable component of the bicycle---an invention whose success it critically enabled. Indeed, the term ``wheel'' was once used to refer to the entire bicycle; the League of American Bicyclists was once the League of American Wheelmen before changing their name in 1994\todo{Add a citation here}.

The vast majority of bicycle wheels produced today are pretensioned ``wire-spoked'' wheels in which a system of slender, pre-tensioned spokes resist buckling when loaded. Originally conceived for aircraft landing gear by Sir George Cayley in 1808, the prestressed wire-spoked wheel saw its first practical application on the bicycle and the design was later utilized in early automobiles and aircraft \cite{Hadland2014}. Although materials and manufacturing methods have evolved over time, the fundamental design, a system of individually prestressed spokes in tension stabilized by a rim in compression, has not significantly changed since the 1870's.

For the first 50 years of the bicycle, wheels generally consisted of a small number of stout wooden spokes fitted into individual wooden rim sections, or felloes, fitted together with mortice and tenon joints. An iron tire was heated and placed around the circumference of the rim and allowed to cool and contract, putting the spokes in compression and the tire in tension \cite{Sharp1977}. The structure was held together by this prestress; there was no need to bond the rim sections together. The spokes, which were made quite wide to prevent lateral buckling, made these wheels extremely heavy. Eugene Meyer obtained a patent for a tensioned wire-spoked wheel in 1869 which enabled the construction of much larger wheels \cite{Clayton1991}. In the tension-spoke wheel the method of prestress was reversed: each slender iron spoke could be individually pre-tensioned, stabilized by an iron rim in compression.

Several other prestressing methods were developed around the same time. W. F. Reynolds and J. A. Mays introduced the Phantom bicycle featuring looped wire spokes secured to one flange of the hub, laced through an eyelet on the rim, and secured to the other hub flange \cite{Herlihy2004}. The spokes were all tensioned simultaneously by spreading apart the hub flanges. The spokes of James Starley's successful Ariel bicycle, which emerged radially from the hub, were pre-stressed simultaneously by rotating the hub relative to the rim by means of a pair of levers connected to the hub and secured to the rim by adjustable tensioning rods \cite{Caunter1955}. Although the spokes were attached radially (and therefore could not efficiently transmit torque), the prestress resulted in a small offset which conferred rotational stiffness. The tangent-spoked wheel, developed by Starley in 1874, achieved rotational stiffness by connecting the spokes tangent to the hub. This remains the most common spoke configuration used today.

Prestressing the spokes allowed the use of thin wire, rather than stout wooden spokes, enabling larger and lighter wheels. The gear of a direct-drive pedal-driven bicycle (the linear distance traveled per rotation of the cranks) is equal to the circumference of the wheel, so a larger wheel confers a considerable speed advantage. The high-wheel bicycle enjoyed popularity amongst primarily young men of means thoughout the 1880s and into the 1890s \cite{Smith1972}. The development of practical and lightweight chain drives brought the ``safety bicycle''---so named because the gear ratio afforded by the chain enabled the use of a smaller front wheel, reducing the risk of pitching over the handlebars---to the masses.

\section{The Eiffel Tower and the Ferris Wheel}

The tensioned bicycle wheel arrived during a time of rapid progress in the use of iron and steel in lightweight structures. The Home Insurance Building in Chicago, completed in 1885, was built around a steel skeleton-frame and is widely recognized as the first ``sky-scraper'' \todo{Add citation}. The Eiffel Tower, constructed in 1889 for the World's Fair, became the tallest human-built structure in the world. Its sparse, wrought-iron truss frame impressed visitors and initially drew sharp criticism from the engineering community\todo{Add citation}. The first prestressed bicycle wheels were commonly called ``suspension wheels,'' perhaps due to an analogy with the suspension bridge, itself a relatively new structural innovation \cite{Sewall1896}.

Perhaps the most striking structural analogy to the bicycle wheel, the Ferris Wheel, was completed in 1893 for the Columbian World Exposition in Chicago. Crafted as Chicago's answer to the Eiffel tower, George Washington Gale Ferris' wheel drew frequent comparisons to the bicycle, such as the following observation by Julian Hawthorne, (son of the novelist, Nathaniel) \cite{Larson2004}:

\begin{quote}
\emph{``...it has no visible means of support---none that appear adequate. The spokes look like cobwebs; they are after the fashion of those on the newest make of bicycles.''}
\end{quote}

Of course the means of support \emph{were} adequate, and the ``Chicago Wheel'' derived its remarkable stability from the fact that the iron spokes were pretensioned by turnbuckles so as not to lose tension (and therefore stiffness) when they came to be at the top of the wheel. The construction of the Ferris Wheel, and perhaps the reluctance of its builders to satisfy the engineering community with details of its analysis, spurred considerable interest and debate \cite{Searles1893}. These early analyses relied heavily on intuition or assumed that the rim was stiff enough that all deformation was confined to the spokes. But the technical marvel of the Ferris Wheel, and the bicycle wheel that preceded it, may have inspired a long but sparse effort to understand the mechanics of pretensioned wheels.


\section{Technical literature on wheels}

% Importance of wheel mechanics
%   Dynamics and stability
%   Wheel failure
%   Crashworthiness

% Early analytical studies
%   Smith
%   Pippard, various studies
%   Hetenyi analogy
In his 1896 treatise on the mechanics and design of bicycles and tricycles \cite{Sharp1977}, Archibald Sharp gave a brief, qualitative treatment on the deformation of a tension-spoked wheel. He correctly noted that the bottom spokes play the most dynamic role in supporting loads applied to the hub. He then motivated the development of a set of equations for the tensions in the spokes using a polygonal approximation for the rim and correctly deduced that they form a statically indeterminate system.

\subsection{Analytical studies}

In 1901 Bernard Smith \cite{Smith1901} published an analysis of the deformation of a pretensioned wheel with purely radial spokes analytically by assuming that the number of spokes is great enough such that the spoke tension is continuously distributed about the rim and proportional to radial displacement. This clever method transforms a discrete system of coupled equations for the spoke tensions into a linear, ordinary differential equation for the radial displacement of the rim. Through an analytical solution, he came to the same conclusion as Sharp---namely, that the bottom spokes play the most direct role in supporting loads---and gave a table for the influence function (change in spoke tensions per unit applied radial load) for a typical 32-spoke wheel.\todo{include diagram from Sharp?}

The most complete theoretical treatment of the deformations of tension-spoked wheels came in a series of investigations in 1931-32 by Alfred J. Sutton Pippard and various coauthors. Pippard, a civil engineer with expertise in elasticity of lightweight structures, was engaged by the British Royal Air Force to undertake an investigation of lightweight wheels for aircraft\footnote{After his retirement in 1956, Pippard took up a yearlong visiting lecturer position at Northwestern University where he taught undergraduate and graduate courses in theory of structures.}. Pippard and Francis derived a set of coupled equations for the spoke tensions in radially-spoked wheels under radial loads, and gave tables for the calculation of wheels with up to six spokes \cite{Pippard1931}. Apparently unaware of Smith's earlier contribution, they also gave a general analytical solution for the problem under the smeared-spokes approximation. Recognizing the power of this approach, Pippard and White extended the method to analysis of wheels with non-radial spokes \cite{Pippard1932b}. Pippard and Francis analyzed the wheel under lateral loads \cite{Pippard1932a}, although they neglected the effects of spoke tension both in the effective lateral stiffness of the spokes and in the mechanics of the rim (which is under compression).

For the purposes of validation, Pippard and Francis performed radial extension tests on specially-constructed wheels with pin-jointed radial spokes to compare the change in spoke tension to the smeared-spokes model. They achieved good agreement in the spoke tension at the loaded point (maximum difference less than 9\%) between theory and experiments, and noted that accuracy increased with the number of spokes, as is expected when assuming that the spoke stiffness is continuously distributed. The rims used in their experiments, cut from solid steel plate, were quite stiff in bending compared to the axial stiffness of the spokes which would have increased the accuracy of the smeared-spokes model.

% Hetenyi
The Smith-Pippard approximation---taking the limit in which the number of spokes goes to infinity---transforms the bicycle wheel into a curved beam resting on an elastic foundation. The foundation produces a reaction at each point proportional to the local deflection of the beam. The stiffness of the beam effectively spreads out point loads with a characteristic decay length equal to $(4EI/k)^{1/4}$, where $EI$ is the bending stiffness of the beam around the relevant axis and $k$ is the stiffness constant of the foundation (in units of force per unit length, per unit deflection). The theory of beams on elastic foundations was broadly summarized and formalized by Hetenyi \cite{Hetenyi1946} in his 1946 monograph. He included a treatment of radial deflection of rings on elastic foundations, citing Pippard et. al., but not Smith.

% Note by Papadopoulos in Human Power
The significance of the beam-on-elastic-foundation analogy was noted by Papadopoulos in a note on wheel mechanics in the now-defunct journal \emph{Human Power} \cite{Papadopoulos1986}:

\begin{quote}
\emph{``...it may be most helpful to think of a bicycle wheel as a long, bendable, twistable, curved rod (or beam) held in place by 36 springs archored in a firmly-held hub. Forces in any direction applied to a point on the rim always produce the greatest effects in spokes nearby.''}
\end{quote}

The ratio of spoke stiffness to rim stiffness (very large for radial deflections, very small for tangential deflections, moderate for lateral deflections) determines the extent of the affected zone near the load. Papadopoulos also noted the importance of torsional stiffness of the rim. Comparing a double-wall rim (whose cross-section contains a large hollow cavity) to a single-wall rim with a much greater bending stiffness, the single-wall rim will deform laterally much more readily due to its low torsional stiffness.

\subsection{Modern bicycle wheel literature}

After Pippard there has been scant development of broadly applicable analytical theories for stress analysis of bicycle wheels, to the best of my knowledge. Instead, modern studies have employed finite-element analysis and experimental techniques.

% Brandt
Jobst Brandt, a mechanical engineer and cycling enthusiast, published \emph{The Bicycle Wheel} in 1981 as a practical manual for wheelbuilding and brief treatise on wheel mechanics drawn from his own experience and analysis \cite{Brandt1993}. To illustrate some key points of wheel mechanics, Brandt calculated radial and tangential deformations on a 2-dimensional model of a typical wheel using the linear finite-element method. As in Smith and Pippard's analyses, the dominant role of the lower spokes is apparent. Finite-element results also exhibit a phenomenon supressed by the continuum approximation of Pippard and Smith: a drive torque applied to the hub and reacted at the road contact point produces a small, periodic, radial distortion of the wheel with a period of four spokes. This effect is due to the local increase and decrease in tensions from leading and trailing spokes in a tangent-spoked wheel. The same periodic deformation would be observed in the lateral deflection, which was not considered in Brandt's 2-dimensional analysis.

% Salamon and Oldham, finite-element
Salamon and Oldham published the first comparative finite-element study of the bicycle wheel \cite{Salamon1992}. They compared wheels with radial spokes to wheels with tangent spokes under purely radial loading, finding that the radially-spoked wheel was only 5\% stiffer than the tangent-spoked wheel, and the maximum stresses were substantially similar. They also analyzed wheels with 3, 7, and 9 spokes (which would typically not be built with tensile prestress) and found that the bending stress varied by up to an order of magnitude when the load was moved from directly at a spoke to a point halfway between spokes.

% Burgoyne and Dilmaghanian
Most experimental and theoretical studies have neglected the effect of the tire and inner tube due to the considerable complexity introduced. Burgoyne and Dilmaghanian \cite{Burgoyne1993} studied the entire spokes-rim-tire-tube assembly as an engineering system and demonstrated that the tire effectively spreads out radial loads along a segment of rim near the contact point. The load distribution has a significant effect on the bending moment sustained by the rim (hence the danger of damaging the rim when hitting a pothole at low inflation pressure), but only has a small effect on spoke strain. They identified three prestessing systems in the wheel: (1) the spokes prestressed against the rim, (2) the tire casing prestressed by the air pressure, and (3) the prestressing of the tire bead, which they incorrectly claim is prestressed against the rim due to a difference in unstressed circumference \cite{Papadopoulos1995}.

% Gavin and spoke fatigue
Wheel failure is most commonly preceded by the gradual accumulation of fatigue damage, especially in the spokes, which are repeatedly stressed each time the wheel rotates or a lateral load is encountered. Henri Gavin constructed an instrumented wheel for measuring strain on a single spoke during naturalistic riding conditions in order to make fatigue life predictions \cite{Gavin1996}. Averaged road testing results from three wheels with different spoke configurations showed negligible differences in strain history between the three wheels, however, it should be noted that possible strains from lateral or braking loads which produced non-periodic strains were omitted from the analysis. Gavin also describes an experimental method to obtain the lateral bending stiffness $EI_2$ and torsional stiffness $GJ$ of a bicycle rim, but he did not conduct lateral stiffness tests on a wheel to compare with theory or his finite-element calculations.

% Minguez and Vogwell
The rim and spokes are structurally coupled and their contributions to wheel stiffness cannot be easily decoupled. In an attempt to separately quantify the contributions of the spokes and rim, Minguez and Vogwell \cite{Minguez2008} derived a model for the radial stiffness of bicycle wheels by assuming that the top half of the rim remains perfectly rigid, and the bottom half deforms from a circular arc to an ellipsoidal (squashed) arc. They calculated the stiffness using Castigliano's theorem on the assumed displacement field. This assumption contradicts the qualitative observations by Sharp and calculations by Smith for typical wheels (that the distortion of the rim is limited to a narrow arc about the bottom spoke), but they nevertheless achieved close (within 10\%) agreement due to their choice of very stiff rims on wheels with few ($<18$) spokes.

% Undergraduate and masters theses, and hobbyist projects
%   Mariappan? (reference in Keller, apparently mentions buckling?)
%   Ng, Hartz, Keller, Svensson, Kern
The structural simplicity of the bicycle wheel makes finite-element analysis straightforward: if localized stresses are not of great interest the spokes and rim can be readily modeled with structural beam elements, and no delicate choices must be made regarding meshing except to choose an appropriate discretization for the curvature of the rim. Due to its structural simplicity and broad appeal, the bicycle wheel is a popular subject for undergraduate and masters theses and  hobbyist projects. These authors have focused on stiffness and stress analysis \cite{Hartz2002,Ng2012}, optimization \cite{Keller2013,Svensson2015}, and buckling \cite{Kern2016}.

% \subsection{Mixed-technique studies}
% Recent mixed-technique studies:
%   Guibilato, et. al. (wheelset "reactivity") ??

\subsection{The ``gray'' literature}
% "Gray" literature
%   Damon Rinard
%   "Great wheel stiffness test" - Roue Artisanales

Other than the studies mentioned above, scant attention has been paid to the bicycle wheel in the peer-reviewed technical literature. However, a number of hobbyists and specialists have published experimental results on the wheel with varying levels of technical rigor and documentation. Stiffness, being the most intuitive mechanical property to the non-specialist, is generally the focus of these studies. It is generally agreed that lateral stiffness plays a much more significant role than radial stiffness in the performance and qualitative experience of a wheel \cite{Kopecky2013}. The radial stiffness is generally about two orders of magnitude larger than the lateral stiffness, and is completely obscured by the flexibility of the inflated tire.

Damon Rinard devised a simple setup for measuring lateral stiffness of wheels by gripping the axle between specially-machined aluminum blocks held in a milling machine and loading the rim with hanging weights \cite{Rinard}. In addition to simply publishing the lateral stiffness for a variety of commercially-available wheels, he also addressed several questions about stiffness including variation with spoke tension (discussed in detail in Section \ref{sec:Lateral}), difference in stiffness between the right and left directions, and some specialized spoke configurations. Because he measured stiffness with a fixed 25 lb test load, he is not able to distinguish between the infintesimal stiffness of the wheel and the nonlinear deflection due to potentially buckled spokes.

The industry blog, Roues Artisanales \cite{RARblog} (affiliated with RAR, manufacturer of high-end wheels and components) has published a number of investigations on wheels, mostly involving lateral stiffness. As part of their ``Great Wheel Test,'' they published lateral and radial stiffnesses of 44 wheels. In smaller studies, they performed systematic tests, with some wheel parameters held constant, to determine the effects of spoke count, spoke diameter, and spoke tension. They have not compared these results with theory.

\section{Outline of this thesis}

The purpose of this thesis is to develop a consistent theoretical framework for stress analysis of the bicycle wheel. Previous theoretical models emerge as special cases within this framework. In Chapter \ref{chap:stress_analysis} I develop the analysis framework and use it for stress analysis of the wheel under external loads. In Chapter \ref{chap:acoustic_testing}, I demonstrate an experimental method for determining rim stiffness parameters necessary for theoretical analysis. In Chapter \ref{chap:tension_buckling}, I show that the equations developed in Chapter \ref{chap:stress_analysis} lead to an elastic instability of the wheel under spoke tension, and derive equations for the critical tension under several approximations. In Chapter \ref{chap:buckling_ext_loads}, I analyze buckling failure of the wheel under external loads and derive an approximate formula for the critical radial load based on a simplified model of the wheel. Finally, in Chapter \ref{chap:optimization}, I use the models derived in this thesis to design wheels optimized for specific performance parameters. \todo{Add references to new experimental sections when completed}.

\end{document}
